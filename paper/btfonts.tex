%%% btfonts.tex -- BHRID TeX fonts

% Author: Michael Grünewald
% Date: Wed Sep 27 00:00:01 CEST 2006

% Copyright (C) 2006, 2013 Michael Grünewald
% All rights reserved.
%
% This file is part of Bhrìd TeX.
%
% Bhrìd TeX is free software: you can redistribute it and/or modify
% it under the terms of the GNU General Public License as published by
% the Free Software Foundation, either version 3 of the License, or
% (at your option) any later version.
%
% Bhrìd TeX is distributed in the hope that it will be useful,
% but WITHOUT ANY WARRANTY; without even the implied warranty of
% MERCHANTABILITY or FITNESS FOR A PARTICULAR PURPOSE.  See the
% GNU General Public License for more details.
%
% You should have received a copy of the GNU General Public License
% along with Bhrìd TeX.  If not, see <http://www.gnu.org/licenses/>.

\class bhridpaper
\enablehverbatim

\title les Fontes de caractères de BHRID \TeX
\author Michael Grünewald

Lorsqu'on rédige un mansucrit avec BHRID \TeX, il est utile de savoir
quels symboles sont à disposition, et comment les insérer dans le
manuscrit. Si la plupart des conventions établies par le {\sl \TeX
book} pour la saisie des documents dans le chapitre~$9$ {\it \TeX's
Roman Fonts} restent valables pour BHRID \TeX, des modifications ont
été nécessaires pour utiliser les fontes Latin Modern à la place des
fontes Computer Modern.


\section Rappels

\formalpar Insérer un caractère par d'après son numéro

Les commandes |\fontchar| et ses spécialisations peuvent être
utilisées pour insérer un caractère arbitraire d'une fonte
particulière. On écrit par exemple |\fontchar\mainfontselect"65| pour
insérer le caractère situé à la position héxadécimale~65, les
quantités numériques peuvent être exprimées en~\em{octal} ou
en~\em{décimal}, dans la syntaxe habituelle pour les constantes
numériques de \TeX. On peut également
écrire~|\fontchar\mainfontselect\the\rnA\relax|,
ou~|\fontchar\mainfontselect\rnA|; en l'absence du |\relax| dans la
version avec |\the|, les \em{tokens} suivants sont examinés pour
rechercher des chiffres supplémentaires.

Les spécialisations de la commande |\fontchar| sont obtenues par des
définitions ressemblant~à |\def\tlchar{\fontchar\tlfontselect}|,
on peut alors dire~|\tlchar\rnA|, \em{etc.}, pour accéder directement
à n'importe quel caractère de la fonte choisie par |\tlfontselect|.


\formalpar Nom symbolique des caractères

Il peut être utile de nommer symboliquement des caractères, tout comme
on le fait avec |\chardef|, on peut par exemple
dire~|\def\tlchardef{\fontchardef\tlfontselect}|;
et~|\tlchardef\theTextLatinLetterA=65|.


\section Fontes de caractères romains

Dans votre manuscrit, les caractères |A-Z|, |a-z| et |0-9| jouent leur
propre rôle, ainsi que certaines marques de ponctuation
|( ) [ ] ` ' - * / . , @|.
Les marques |; : ! ?| sont traitées spécialement par BHRID
\TeX\ qui place automatiquement les espaces nécessaires avant ces
caractères, en accord avec les usages linguistiques en vigueur; pour
obtenir le bon résultat, ces marques ne doivent être précédées par
aucun blanc.

\formalpar Ligatures

\TeX\ reconnaît certaines combinaisons particulières comme des
\em{ligatures}; ainsi
\begindisplay
\settabs 3\columns
\def\\#1{\li{#1} donne #1;}%
\+\\{ff}&\\{ffi}&\li{`{}`} donne ``\cr
\+\\{fi}&\\{ffl}&\li{'{}'} donne ''\cr
\+\\{fl}&\li{-{}-} donne {--}&\li{-{}-{}-} donne ---\cr
\+\li{!{}`} donne !`;&\li{?{}`} donne ?`\cr
\enddisplay
On peut cacher une ligature à \TeX, on écrit par exemple pour
cela~|f{}f| ou~|-{}-{}-|; ceci est peu commode, mais il est rare que
les ligatures soient indésirables, un exemple de cette situation est
le mot anglais \hbox{\qu{shelf{}ful} qui ne doit pas être écrit
\hbox{\qu{shelfful}}}.

Si vous utilisez d'autres fontes que celles de la famille Latin
Modern, il se peut qu'elles présentent un autre jeu de
ligatures, consultez alors la documentation apropriée.


\formalpar Sur le mode mathématique

Les caractères \li{+} et \li{=} jouent également leur propre rôle,
mais il vaut mieux ne jamais les utiliser hors du \em{mode
mathématique} où ils sont respectivement considérés comme une
opération et comme une relation, et les espaces sont automatiquement
positionnées autour d'eux. En mode texte on obtient par
exemple~\hbox{3+2=5} contre~$3+2 = 5$ avec le mode mathématique.

Remarquons finalement que les chiffres peuvent produire un résultat
différent si vous utilisez le mode mathématique ou le mode texte pour
les saisir. Cela n'est pas le cas avec BHRID \TeX, mais des fichiers de
macros peuvent créer ce comportement dissymétrique; dans ce cas suivez
les recommendations de la documentation sur ce point.

Les caractères \li{-} et \li{/} peuvent être utilisés dans et hors du
mode mathématique, ils acquièrent alors des sens différents. La
soustraction et la division ne devraient appraraître qu'au sein du
mode mathématique; le tiret et l'oblique ne devraient apparaître que
dans le mode texte.


\formalpar Autres caractères du jeu ASCII

Les caractères |$ # % &| %$
jouent un rôle particulier pour BHRID \TeX, on peut les représenter
dans un manuscrit par les séquences respectives~|\$ \# \% \&|. Les
caractères |\ { } ^ _ ~| jouent également un rôle spécial, mais leur
prence dans un texte est inhabituelle; on peut les obtenir par des
noms longs comme expliqué dans la suite. Enfin les quatre symboles
\disablehverbatim\li{" < > |}\enablehverbatim\ n'ont pas à être
utilisés lorsqu'on prépare un texte, la double apostrophe droite
doit être remplacée par des guillemets ou des \em{quotes} ouvrants ou
fermants; les trois autres sont à utiliser en mode mathématique.


\formalpar Sur les tirets

On utilise usuellement trois longueurs de tirets dans le texte, comme
on l'a déjà vu, il sont reprentés dans le manuscrit par les séquences
\li{-}, \li{-{}-} et \li{-{-}-}. Le tiret le plus court est utilisé
dans les noms composés, ou lorsque la grammaire le demande, par
exemple on écrit
\beginlist\inset\literal
\item{franc-tireur} pour \qu{franc-tireur};
\item{a-t-il} pour \qu{a-t-il};
\endlist
et c'est cette longueur de tiret qui est insérée automatiquement par
le mécanisme des césures. Le tiret de longueur intermédiaire est
utilisé dans les intervalles, comme par exemple
\beginlist\inset\literal
\item{pp.\tstilde55-{}-61}
 pour \qu{pp.~55--61};
\item{durant l'hiver 1789-{}-1790}
 pour \qu{durant l'hiver 1789--1790}.
\endlist
Les tirets de longueur supérieure sont quant à eux utilisés pour
délimiter certaines incises, comme dans
\beginlist\inset\literal
\item{une -{-}- ou plusieurs -{-}- d'entr'elles}
 pour \qu{une --- ou plusieurs --- d'entr'elles};
\endlist
les règles comcernant les espaces autour de ces tirets ne sont pas les
mêmes pour toutes les langues, ainsi les anglais ne placent aucune
espace autour de ces tirets. On utilise aussi ces tirets au début d'un
paragraphe pour indiquer le commencement d'un dialogue, puis chaque
fois que la parole change de personnage dans ce dialogue.


\formalpar Sur le code Latin 9

BHRID \TeX\ accepte comme entrée un fichier codé avec le jeu Latin~9,
ou ISO~8859~15, ce format interprète alors correctement les codes de
caractères hors du domaine ASCII, ocuppant les positions dans
l'intervalle~128--255.

Les caractères dans ces positions sont \em{actifs}, ils jouent le rôle
d'une séquence de contrôle qui s'arrange pour insérer le caractère
souhaité dans le document. On peut donc entrer |«éléphant»|,
|Lårs Øre| et |süßkind| pour obtenir \em{«éléphant»}, \em{Lårs Øre}
et~\em{süßkind}.

\em{Remarque. Les caractères dans les positions 128--255 ont un
comportement indéfini dans le mode mathématique, ils ont aussi
un comportement indéfini lorsque la fonte courante n'a pas~\li{T1}
comme paramètre \li{fontpage}.}

Le code Latin~9 ne correspond à aucun \em{codepage}, dans la
terminologie propre aux systèmes Microsoft, mais il est assez proche
du code Latin~1 qui correspond lui au~\em{codepage} CP819, ou
IBM819. Les utilisateurs de systèmes Microsoft peuvent éditer et
enregistrer des fichiers dans le code Latin~9 au moyen d'éditeurs
appropriés, comme par exemple GNU~Emacs. Dans tous les cas, n'importe
que caractère de l'entrée peut être obtenu au moyen de la \em{caret
notation}, comme expliqué dans le {\sl\TeX book}~pp.~46--48.

\formalpar Accents

Si le code Latin~9 comprend bon nombre de caractères accentués,
certains peuvent manquer, et BHRID \TeX\ permet d'utiliser le mécanisme
familier aux utilisateurs de PLAIN \TeX:
\begindisplay
\relax
\+Type\qquad&to get\cr
\+|\`o|&\`o\cr
\+|\'o|&\'o\cr
\+|\^o|&\^o\cr
\+|\"o|&\"o\cr
\+|\~o|&\~o\cr
\+|\=o|&\=o\cr
\+|\.o|&\.o\cr
\+|\u o|&\u o\cr
\+|\v o|&\v o\cr
\+|\H o|&\H o\cr
\+|\t oo|&\t oo\cr
\+|\c o|&\c o\cr
\+|\d o|&\d o\cr
\+|\b o|&\b o\cr
\enddisplay
Deux caractères spéciaux sont destinés à être utilisés avec les
commandes d'accent, il s'agit des `i sans point' et `j sans point',
que l'on obtient respectivement par |\i| et~|\j|. On écrit~|\`\i| et
non~|\`i| pour obtenir~`\`\i'.

\formalpar Caractères spéciaux

Certaines ligatures ont un nom, comme avec PLAIN \TeX, ce sont
|\oe|, |\OE|, |\ae|, |\AE|, |\ss| (es-zet) et~|\SS|. Les caractères
`\aa' et `\AA' peuvent être obtenus par~|\aa| et~|\AA|. En revanche,
PLAIN \TeX\ prévoit que l'on puisse saisir `Ø' avec |\O|, ce qui est
abandonné ici au profit du \em{groupe orthogonal} des géomètres
désigné par cette macro, les conventions voisines pour |\o|, |\L| et
|\l| sont également abandonnées.

Les signes de ponctuation sont des caractères spéciaux à plus d'un
titre, d'une part certains d'entre eux sont actifs et ajoutent
automatiquement certaines espaces autour d'eux, d'autre part certains
ne figurent pas dans la table ASCII, mais seulement avec le jeu
Latin~9.
\beginchartable
3a pncolon
3b pnsemicolon
3f pnquestion
21 pnexclam
a1 pnexclamdown
bf pnquestiondown
ab pnguillemetleft
bb pnguillemetright
xx pndblquoteleft
xx pndblquoteright
xx pndblquotebase
\endchartable
Si vous pouvez produire certains de ces caractères par une ligature,
ce mécanisme peut biaiser l'ajout automatique des espaces autour de
certains de ces signes.

Le placement des espaces autour des signes de ponctuation est
subordonné aux usages linguistiques, vous sélectionnez tel ou tel
usage avec la commande |\selectpunct{whatever}|, ou avec
|\selectlocale{whatever}| qui ajuste tous les paramètres dépendant des
usages linguistiques.

\smallskip
Liste des symboles pour le texte.
\beginchartable
b6 P
a7 S
aa ordfeminine
ba ordmasculine
b0 degree
xx centigrade
xx No
xx dag
xx ddag
xx dblverticalbar
xx percent
xx perthousand
xx permyriad
\endchartable
Rappelons que vous pouvez aussi obtenir le caractère~\% avec la
séquence courte~|\%|.


\smallskip
Liste des symboles utilisés pour le commerce.
\beginchartable
a2 cent
a3 sterling
a4 euro
a5 yen
xx dollar
ae registered
a9 copyright
xx copyleft
xx published
xx trademark
\endchartable
Vous obtenez également le symbole~\$ par~|\$|. Dans PLAIN \TeX\ le
symbole \sterling\ peut être obtenu avec un signe dollar lorsqu'une
fonte italique est active, ceci n'est pas vrai dans BHRID \TeX.


\formalpar Caractères hors liste

Les fontes Latin Modern dans les pages \li{T1} et \li{S1} comportent
beaucoup d'autres symboles, qui ont des noms comme~|\tl...|
et~|\ts...|; ils n'ont pas de nom adéquat pour figurer dans vos
manuscrits mais il est facile d'en ajouter selon vos besoins.

Supposons par exemple que vous souhaitiez utiliser les symboles
\beginchartable
xx tsborn
xx tsdied
xx tsmarried
xx tsdivorced
xx tsleaf
\endchartable
pour préparer des documents généalogiques. Les noms |\ts...| sont
inadéquats pour votre manuscrit parce qu'ils commandent très
précisément l'insertion d'un caractère dans une fonte fixée, alors que
d'une part vous voudriez peut-être ajouter des espaces à ces
caractères ou leur prêter un comportement particulier, et que d'autre
part vous voudriez peut-être un jour utiliser des symboles provenant
d'une fonte différente. La bonne façon de faire est de choisir des
noms symboliques pour ces caractères, comme par exemple \li{born},
\li{died}, \li{married}, \li{divorced}, \li{leaf} s'il ne sont pas
déjà utilisés; et d'ajouter les définitions
\begincode
\def\born{\tsborn}
\def\died{\tsdied}
...
\endcode
à votre manuscrit. Pour modifier le comportement de ces séquences il
suffit alors d'altérer ces définitions, au lieu de modifier les noms
|\tsborn| \em{etc.} utilisés par BHRID~\TeX, ce qui provoque un
comportement indéfini et ne doit être effectué sous aucun prétexte:
les macros peuvent donc utiliser les noms |\ts...| et |\tl...| avec la
certitude que ceux-cis font ce qui leur est demandé.


\formalpar  Fontes Latin Modern

\begincode
\csdef{selectfont@Latin Modern}{%
  \selecttl{Latin Modern}%
  \selectts{Latin Modern}%
  \selectmit{Latin Modern}%
  \selectms{Latin Modern}%
  \selectmx{Latin Modern}%
  \selectmrm{Latin Modern}%
  \selectmbf{Latin Modern}%
  \selectmsf{Latin Modern}%
  %
  % Final Setup
  %
  \def\fontrmfamily{lmr}%
  \def\fontttfamily{lmtt}%
  \def\fontsffamily{lmss}%
  \def\mainfontselect{\mainfont\fontselect}%
  \def\mainfont{\rmfamily\fontpage{T1}\fontseries{m}\fontshape{n}}%
  \def\mainsize{\tenpoint}%
}
\endcode

\bye

%%% End of file `btfonts.tex'
