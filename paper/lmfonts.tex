%%% lmfonts.tex -- Fontes Latin Modern

% Copyright © 2001–2015 Michael Grünewald
% All rights reserved.
%
% This file is part of Cadet TeX.
%
% Cadet TeX is free software: you can redistribute it and/or modify
% it under the terms of the GNU General Public License as published by
% the Free Software Foundation, either version 3 of the License, or
% (at your option) any later version.
%
% Cadet TeX is distributed in the hope that it will be useful,
% but WITHOUT ANY WARRANTY; without even the implied warranty of
% MERCHANTABILITY or FITNESS FOR A PARTICULAR PURPOSE.  See the
% GNU General Public License for more details.
%
% You should have received a copy of the GNU General Public License
% along with Cadet TeX.  If not, see <http://www.gnu.org/licenses/>.


\class cadetpaper
\title Fontes Latin Modern

Ce document présente la façon dont sont utilisées les fontes \em{Latin
Modern} dans \brTeX. L'article \em{An exploration of the Latin Modern
fonts}, par~Will Robertson~[1] peut être consulté pour une
présentation de ces fontes.
\begindescription
\item{1. Un échantillon de fontes}
Ce paragraphe montre toutes les formes et toutes les familles de
caractères disponibles dans les fontes Latin Modern.
\item{2. PFSS torture test}
Ce paragraphe montre le résultat de toutes les requêtes PFSS.
\enddescription
\item{[1]} Robertson, Will: \em{An exploration of the Latin Modern
fonts}, TPJ 2006 No 01, 2006--02--21.

\def\pangramme{«Dès Noël où un zéphyr haï me vêt de glaçons würm\-iens
je dîne d'exquis rôtis de bœuf au kir à l'aÿ d'âge mûr \&~cætera!»}


\section 1. Un échantillon de fontes

\def\setsamplesize{\setfontsize{12}}
\def\sample#1 #2#3#4{%
  \begingroup
  \setfontfamily{#1}%
  \setfontshape{#2}%
  \setfontweight{#3}%
  \setfontwidth{#4}%
  \readline\to\rtA\then
    \twelvepoint
    \edef\rmA{\fontname}%
    \testfontfamily{lmtt}\iffontfamily\raggedright\fi
    \testfontfamily{lmvtt}\iffontfamily\raggedright\fi
    \noindent\pangramme\par
    \nobreak
    \centerline{\setnormalfont\sevenpoint
      \the\rtA\space (\rmA)}%
    \endgroup
    \bigskip
  \done
}

\begindisplay
\sample lm rmm LMRoman10-Regular
\sample lm omm LMRoman10-Oblique
\sample lm imm LMRoman10-Italic
\sample lm rbx LMRoman10-Bold
\sample lm obx LMRoman10-BoldOblique
\sample lm ibx LMRoman10-BoldItalic
\sample lm rbm LMRoman10-Demi
\sample lm obm LMRoman10-DemiOblique
\sample lm Rmm LMRoman10-CapsRegular
\sample lm Omm LMRoman10-CapsOblique
\sample lmd rmm LMRoman10-Dunhill
\sample lmd omm LMRoman10-DunhillOblique
\sample lmtt rmm LMTypewriter10-Regular
\sample lmtt imm LMTypewriter10-Italic
\sample lmtt omm LMTypewriter10-Oblique
\sample lmtt Rmm LMTypewriter10-CapsRegular
\sample lmtt Omm LMTypewriter10-CapsOblique
\sample lmtt rbm LMTypewriter10-Dark
\sample lmtt obm LMTypewriter10-DarkOblique
\sample lmtt rlm LMTypewriter10-Light
\sample lmtt olm LMTypewriter10-LightOblique
\sample lmtt rlc LMTypewriter10-LightCondensed
\sample lmtt olc LMTypewriter10-LightCondensedOblique
\sample lmss rmm LMSans10-Regular
\sample lmss omm LMSans10-Oblique
\sample lmss rbm LMSans10-Bold
\sample lmss obm LMSans10-BoldOblique
\sample lmss rbc LMSans10-DemiCondensed
\sample lmss obc LMSans10-DemiCondensedOblique
\sample lmss rmx LMSansQuotation8-Regular
\sample lmss omx LMSansQuotation8-Oblique
\sample lmss rbx LMSansQuotation8-Bold
\sample lmss obx LMSansQuotation8-BoldOblique
\sample lmvtt rmm LMTypewriterVarWd10-Regular
\sample lmvtt rbm LMTypewriterVarWd10-Dark
\sample lmvtt rlm LMTypewriterVarWd10-Light
\sample lmvtt omm LMTypewriterVarWd10-Oblique
\sample lmvtt obm LMTypewriterVarWd10-DarkOblique
\sample lmvtt olm LMTypewriterVarWd10-LightOblique
\enddisplay

\section 2. PFSS torture test

Dans ce paragraphe, on affiche les résultats de toutes les requêtes
PFSS. Chaque élément de réponse est formé des arguments de la requête,
des paramètres modifiés et d'un échantillon de la fonte obtenue.
\bigskip

\enableprivatenames
\def\pfsssample{%
  \medskip
  \begingroup
  \fontswitch
  \toksloadcsfontname\to\rtA
  \toksloadfontswitch\to\rtB
  \hskip 10pt\vtop{\hsize=120pt\parindent\z@
    \systemfont
    \expandafter\string\the\rtA
    \hfil\break
    \expandafter\string\the\rtB
  }%
  \hskip 10pt\relax
  \vtop{%
    \advance\hsize by -150pt
    \parindent=\z@
    \raggedright
    \pangramme\par
  }%
  \hskip 10pt
  \par
  \medskip
  \endgroup
}

\def\pfsssample@family{\\{lm}\\{lmd}\\{lmss}\\{lmtt}\\{lmvtt}}
\def\pfsssample@shape{\\{r}\\{o}\\{i}\\{R}\\{O}\\{I}}
\def\pfsssample@weight{\\{m}\\{b}\\{l}\\{d}}
\def\pfsssample@width{\\{m}\\{c}\\{x}}

\def\pfsssample@L#1#2#3{%
  \toksloadcsname pfssloop@#1\to\rtA
  \expandafter\ifx\the\rtA\empty
    \let\rmN\relax
  \else
    \toksloadcsname font#1\to\rtB
    \edef\rmN{\noexpand\listlop\the\rtA\noexpand\to\the\rtB}%
    \rmN
    \toksloadcsname pfssloop@#2\to\rtA
    \toksloadcsname pfsssample@#2\to\rtB
    \edef\rmN{\let\the\rtA\the\rtB}%
    \rmN
    \def\rmN{#3\pfsssample@L{#1}{#2}{#3}}%
  \fi
  \rmN
}

\def\pfssloop@A{\pfsssample@L{family}{shape}\pfssloop@B}
\def\pfssloop@B{\pfsssample@L{shape}{weight}\pfssloop@C}
\def\pfssloop@C{\pfsssample@L{weight}{width}\pfssloop@D}
\def\pfssloop@D{\pfsssample@L{width}{none}\pfssloop@E}
\def\pfssloop@E{\pfsssample}
\def\pfssloop{\let\pfssloop@family=\pfsssample@family\pfssloop@A}

\begingroup
\pfssloop
\endgroup


\bye

%%% End of file `pfssintro.tex'
