%%% text.tex -- Préparation du texte

% Copyright © 2001–2015 Michael Grünewald
% All rights reserved.
%
% This file is part of Cadet TeX.
%
% Cadet TeX is free software: you can redistribute it and/or modify
% it under the terms of the GNU General Public License as published by
% the Free Software Foundation, either version 3 of the License, or
% (at your option) any later version.
%
% Cadet TeX is distributed in the hope that it will be useful,
% but WITHOUT ANY WARRANTY; without even the implied warranty of
% MERCHANTABILITY or FITNESS FOR A PARTICULAR PURPOSE.  See the
% GNU General Public License for more details.
%
% You should have received a copy of the GNU General Public License
% along with Cadet TeX.  If not, see <http://www.gnu.org/licenses/>.


\class paper
\enablehverbatim

\title Préparation du texte du paragraphe
\author Michael \nm{Grünewald}

Nous décrivons les services que rend CADET~\TeX\ dans la préparation du
texte du paragraphe.


\section Algorithme de découpe des paragraphes

L'algorithme de de découpe des paragraphes est décrit en détails dans
le~{\sl\TeX book}, pp.~97--100; les nombreux paramètres influant sur
son comportement y sont également~discutés. Pour mémoire, des
descriptions approximatives de leurs rôles sont rassemblées
ci-dessous.

\begincslist\relax
\item\pretolerance
Seuil de laideur pour les lignes. S'il existe un découpage du
paragraphe qui donne des lignes dont la laideur est chaque fois
inférieure à |\pretolerance|, \TeX\ n'a pas besoin d'essayer la césure
des mots du paragraphe. La recherche d'un découpage du paragraphe
respectant cette contrainte de laideur est la première passe de
l'algorithme de découpe des paragraphes.

\beginvalues
\item{100} plain \TeX\ (texte en anglais)
\item{500} \nm{R.~Séroul} (texte en français)
\item{-1} ignorer la première passe
\item{10000} la première passe réussit toujours
\endvalues

\item\tolerance
Seuil de laideur pour la deuxième passe de l'algorithme de découpe
des paragraphes.

\beginvalues
\item{200} plain TeX (texte en anglais)
\item{1000} \nm{R.~Séroul} (texte en français)
\endvalues

\item\hyphenpenalty
Penalité pour une coupure de ligne dans un mot.

\beginvalues
\item{50} plain TeX
\endvalues

\item\exhyphenpenalty
Pénalité pour une coupure de ligne dans un mot, dans le cas ou le
\em{pre-text} de l'objet discretionary associé est vide. Pour les
césures usuelles, le~|\-| signifie |\discretionary{-}{}{}| et le
\em{pre-break} n'est pas vide, mais après chaque caractère |-| est
automatiquement inséré un |\discretionary{}{}{}| ou le \em{pre-break}
est vide. Le \em{pre-break} peut également se trouver vide lorsque
les~|\discretionary| sont insérés~explicitement.

\beginvalues
\item{50} plain \TeX
\endvalues

\item\doublehyphendemerits
Pénalité pour les césures sur des lignes consécutives.

\beginvalues
\item{10000} plain \TeX
\endvalues

\item\finalhyphendemerits
Pénalité pour une césure sur l'avant-dernière ligne d'un
paragraphe.

\beginvalues
\item{5000} plain \TeX
\endvalues

\item\adjdemerits
Pénalité appliquée lorsque des lignes de catégorie distantes se
suivent. (catégories: \em{tight}, \em{loose}, \em{very loose},~\em{descent}\/)

\beginvalues
\item{10000} plain \TeX
\endvalues

\endcslist


\formalpar Paramètres de l'algorithme

L'algorithme de coupure des lignes de \TeX\ a son comportement régi
par de nombreux paramètres, |\pretolerance|, |\tolerance|,
|\hyphenpenalty|, |\exyphenpenalty|, |\doublehyphendemerits|,
|\finalhyphendemerits|, |\adjdemerits|. Pour la description de chacun
de ces paramètres, voir le {\sl\TeX book}.

La taille des espaces entre les mots d'un paragraphe est contrôlée par
deux paramètres |\spaceskip| et~|\xspaceskip|. Il est parfois
nécessaire d'ajouter une \em{glue} importante à ces paramètres pour
que \TeX\ arrive à préparer un paragraphe sans \em{overfull hbox}.

De façon générale les paragraphes dont les lignes sont longues et
nombreuses sont plus faciles à préparer que ceux dont les quelques
lignes sont très courtes. Aussi, si vous préparez une classe de
document dont certains paragraphes doivent être très étroits, il vous
est recommandé de rechercher des paramètres permettant à l'algorithme
de découpe des lignes de fonctionner dans ces conditions
particulières.

Certaines caractéristiques linguistiques, comme la longueur des mots
couramment employés, ont une incidence indirecte sur le travail soumis
à \TeX\ pour la préparation d'un paragraphe. Les paramètres
coutumiers~|\selectparagraph| modifient les paramètres de découpe des
lignes du paragraphe de façon à refléter ces caractéritiques. La
commande |\selectparagraph| est appelée automatiquement
par~|\selectlocale|.


\formalpar Modification temporaire des paramètres de l'algorithme

CADET~\TeX\ offre une commande |\pargroup| qui ouvre un groupe et le
referme à la fin d'un paragraphe; il est ainsi possible de modifier
temporairement les paramètres de l'algorithme de découpe des lignes en
vue de traiter un paragraphe particulièrement difficile. De façon
complémentaire, CADET~\TeX\ définit trois commandes~|\paragraphtricky|,
|\paragraphhard|, |\paragraphnasty| qui altèrent ces paramètres de
façon légèrement permissive pour~|\paragraphtricky|, jusqu'à être très
permissif avec~|\paragraphnasty|. On écrit par exemple
\begincode
\pargroup
\pretolerance=-1
\finalhyphendemerits=-500
Dans ce paragraphe, un coupure de ligne à la fin de
l'avant-dernière ligne est encouragée.
\endcode
La commande~|\pargroup| a un deuxième nom,~|\thisparagraph|, et les
trois commandes précédentes ont des noms
spécialisés~|\thisparagraphtricky|, \em{etc.} pour permettre la
modification rapide des paramètres de l'algorithme de découpe des
lignes.


\formalpar Composition en drapeau

On peut composer les paragraphes en drapeau déchiquetés sur la droite
avec~|\raggedright|; dont on peut éventuellement limiter la portée à
un paragraphe en ouvrant celui-ci
par~|\thisparagraph\raggedright|. Des macros |\raggedleft| et
|\raggedboth| sont également disponibles.


\formalpar Composition étroite

Les règlages~|\paragraphnarrower| (ou |\narrower|)
et~|\paragraphindent| ajustent les paramètres de~\TeX\ pour qu'il
compose un paragraphe aux lignes plus étroites, ou un paragraphe en
retrait vers la droite. Ces macros ont des versions spécialisées
|\thisparagraphnarrower| et~|\thisparagraphindent|.

La largeur des espaces réservés égale |\marginleft| ou~|\marginright|,
selon la position de cet espace. Les macros qui présentent un contenu
centré ou mis en retrait d'un côté ou de l'autre devraient utiliser
ces valeurs pour calculer la position du contenu qu'elles présentent.



\section Affichages

Une certaine quantité de matériel vertical peut être mise en exergue
au moyen de la commande~|\begindisplay...\enddisplay|, dont le
résultat est un \em{affichage}. Les caractéristiques de ces affichages
et leurs options de préparation sont discutées ci-dessous


\formalpar Premiers exemples

Pour mettre en exergue une citation longue dans un paragraphe, on peut
écrire dans le manuscrit
\begincode
\begindisplay
Bratteaux, dans son grenier, découpait des pantins tandis que
le père Longuemare, en face de lui, assemblait par des fils
leurs membres épars, et il souriait en voyant ainsi naître
sous ses doigts le rythme et~l'harmonie.
\enddisplay
\endcode
pour obtenir
\begindisplay
Bratteaux, dans son grenier, découpait des pantins tandis que le père
Longuemare, en face de lui, assemblait par des fils leurs membres
épars, et il souriait en voyant ainsi naître sous ses doigts le rythme
et~l'harmonie.
\enddisplay
Les \em{tokens} situés entre l'ouverture du \em{display} et sa
fermeture sont assemblés dans une boîte verticale; dans l'exemple le
premier caractère ouvre un paragraphe, mais on peut très bien insérer
une liste de boîtes préparées individuellement, ou bien un
tableau.
\begindisplay
\selectlocale{en}
\leftline{\symbolic the Poultry Example}
\settabs
\+\em{Weight}\quad&\em{Servings}\quad&\qquad\em{Approximate Cooking Time}\cr
\+\em{Weight}&\em{Servings}&\em{Approximate Cooking Time}\cr
\+8 lbs&6&1 hour and 50 to 55 minutes\cr
\+9 lbs&7 to 8&About 2 hours\cr
\enddisplay
apparaît de la façon suivante dans le manuscrit:
\begincode
\begindisplay
\selectlocale{en}
\leftline{\symbolic the Poultry Example}
\settabs
\+\em{Weight}\quad&\em{Servings}\quad
 &\qquad\em{Approximate Cooking Time}\cr
\+\em{Weight}&\em{Servings}&\em{Approximate Cooking Time}\cr
\+8 lbs&6&1 hour and 50 to 55 minutes\cr
\+9 lbs&7 to 8&About 2 hours\cr
\enddisplay
\endcode
Les affichages sont des éléments insécables, c'est-à-dire qu'un saut
de page ne peut avoir lieu dans un affichage. Il est par conséquent,
souhaitable de n'insérer dans le texte que des affichages de hauteur
réduite, ou bien de placer ceuc-cis dans des insertions, comme le
permettent les commandes~|\midinsert| et~|\topinsert|.


\formalpar Paramètres

La commande |\begindisplay| est une commande de la structure du
document. On peut utiliser avec elle les \em{marques} suivantes:
\begincslist
\item\compact
 diminue l'importance des espaces verticaux précédant et
 succédant~l'affichage.
\endcslist
On peut altérer la préparation de l'affichage au moyen de toutes les
commandes habituelles, notamment~|\emphasis|, |\literal|,
et~|\raggedleft|; ainsi que les commandes modifiant la taille des
caractères.

La commande interne~|\display@I| est utilisée comme séquence
d'initialisation pour le matériel de l'affichage. Sa signification~est
\begincode
\def\display@I{\displayfontsize\displayfont\fontselect}
\endcode
qui bascule vers une taille et une fonte appropriée. On pourrait fort
bien définir~|\display@I| de la façon~suivante:
\begincode
\def\display@I{\tenpoint\mainfont\fontselect}
\endcode
mais l'utilisation des paramètres~|\displayfontsize| et~|\displayfont|
permet de voir ces valeurs utilisées par d'autres macros, comme des
spécialisation des affichages. On peut modifier~|\displayfontsize|
et~|\displayfont| de la façon~suivante
\begincode
\let\displayfontsize=\ninepoint
\let\displayfont=\mainfont
\endcode
Pour des affichages présentant du texte ordinaire, comme pour une
citation ou un exemple, l'utilisation d'une taille légèrement plus
petite que celle du texte courant peut faciliter une lecture inambigüe
du~texte.


Les espaces situés avant et après l'affichage sont placés
par~|\displayopenskip| et~|\displaycloseskip|, ils peuvent être
modifiés temporairement ou de façon permanente pour certaines
applications.

\formalpar Classes de documents

Des classes de documents peuvent redéfinir~|\display@I| pour leurs
besoins spécifiques.
On peut ajouter des marques à celles comprises par la
commande~|\begindisplay| comme dans l'exemple~suivant.
\begincode
\displayoption\compact{%
  \offinterlineskip
  \def\displayopenskip{\nobreak\vskip 0pt minus 3em\relax}%
  \def\displaycloseskip{\vskip 0pt minus 3em\relax}%
}
\endcode



\formalpar Spécialisation des commandes d'affichage

Les séquences~|\displayopen| et~|\displayclose| qui placent les
espaces précédents et succédants à un affichage et ouverent et ferment
une boîte de largeur appropriée. Le texte situé entre~|\displayopen|
et~|\displayclose| doit être bien parenthésé vis-à-vis des deux sortes
de~groupes, et les affichages peuvent être emboîtés.

Ces commandes peuvent être utilisées par des commandes utilisant des
affichages spécialisés, par exemple pour afficher une citation, une
image, un exemple~de~code.



\section Listes

Une liste est un affichage spécialisé formé de courts paragraphes
introduits usuellement par une \em{marque}, comme un tiret ou une
pastille, ou par un~\em{tag}, ces dernières listes sont fort courantes
dans les manuels techniques. CADET~\TeX\ propose des commandes pour la
préparation des listes qui tentent de reproduire les facilités
introduites par \xr{groff_mdoc(7)} et utilisées pour la rédaction
des pages du manuel en ligne de~FreeBSD.

\formalpar Premiers exemples

Les listes les plus courantes ont leurs entrées introduites par un
marqueur, telles que celle apparaissant dans
\begindisplay
% p.59
On retrouve ici les trois degrés de la spécification tels qu'ils ont
été définis au premier chapitre~(cf.~p.~23):
\beginlist\marker
\item valeur spécifique lexicalisée: \em{aller à Paris, presser sur le
      bouton};
\item valeur générique: \em{obéir à son père, c'est de ma faute,
      s'ennuyer de sa fille};
\item marque zéro: \em{écrire une lettre, manger une pomme}.
\endlist
\enddisplay
qui est produit par
\begincode
\begindisplay
On retrouve ici les trois degrés de la spécification tels qu'ils ont
été définis au premier chapitre~(cf.~p.~23):
\beginlist
\item valeur spécifique lexicalisée: \em{aller à Paris, presser sur le
      bouton};
\item valeur générique: \em{obéir à son père, c'est de ma faute,
      s'ennuyer de sa fille};
\item marque zéro: \em{écrire une lettre, manger une pomme}.
\endlist
\enddisplay
\endcode
La marque utilisée dépend de la coutume, par exemple
\begindisplay
\selectlocale{en}
\beginlist
\item This is the first of several cases that are being enumerated,
      with hanging indentation applied to entire paragraphs.
\item The second case is smilar.
\endlist
\enddisplay
est produit par
\begincode
\selectlocale{en}
\beginlist
\item This is the first of several cases that are being enumerated,
      with hanging indentation applied to entire paragraphs.
\item The second case is smilar.
\endlist
\endcode
\endgraf
\medskip

Une autre sorte de liste que l'on rencontre fréquemment est la liste
énumérative, dont les entrées sont introduites par un compteur,
généralement un nombre ou une lettre. Par exemple
\begindisplay
\selectlocale{en}
\beginlist\enum\1
\item This is the first of several cases that are being enumerated,
      with hanging indentation applied to entire paragraphs.
\beginlist\enum\a
\item This is the first subcase.
\item And this is the second. Notice that subcases have twice as much
      hanging indentation.
\endlist
\item The second case is smilar.
\endlist
\enddisplay
est produit par
\begincode
\selectlocale{en}
\beginlist\enum\1
\item This is the first of several cases that are being enumerated,
      with hanging indentation applied to entire paragraphs.
\beginlist\enum\a
\item This is the first subcase.
\item And this is the second. Notice that subcases have twice as much
      hanging indentation.
\endlist
\item The second case is smilar.
\endlist
\endcode
La dernière présentation de liste proposée par CADET~\TeX\ introduit
une entrée par une étiquette, un~\em{tag}; ces listes apparaissent
notamment dans les manuels techniques. Par exemple
\begindisplay
\selectlocale{en}
\beginlist\tag
\item{EX\_USAGE (64)}
The command was used incorrectly, e.g., with the wrong number of
arguments, a bad flag, a bad syntax in a parameter, or whatever.

\item{EX\_DATAERR (65)}
The input data was incorrect in some way.  This should only be used
for user's data and not system files.

\item{EX\_NOINPUT (66)}
An input file (not a system file) did not exist or was not readable.
This could also include errors like ``No message'' to a mailer (if it
cared to catch it).

\item{EX\_NOUSER (67)}
The user specified did not exist.  This might be used for mail
addresses or remote logins.

\endlist
\enddisplay
est le produit de
\begincode
\selectlocale{en}
\beginlist\tag
\item{EX\_USAGE (64)}
The command was used incorrectly, e.g., with the wrong number of
arguments, a bad flag, a bad syntax in a parameter, or whatever.

\item{EX\_DATAERR (65)}
The input data was incorrect in some way.  This should only be used
for user's data and not system files.

\item{EX\_NOINPUT (66)}
An input file (not a system file) did not exist or was not readable.
This could also include errors like ``No message'' to a mailer (if it
cared to catch it).

\item{EX\_NOUSER (67)}
The user specified did not exist.  This might be used for mail
addresses or remote logins.
\endlist
\endcode

\formalpar Paramètres

La commande |\beginlist| est une commande de la structure du
document. On peut utiliser avec elle les \em{marques} suivantes:
\begincslist
\item\compact
 diminue l'importance des espaces verticaux précédant et
 succédant~l'affichage.
\medskip
\item\marker
 prépare une liste dont les entrées sont introduites par un marqueur
 prédéfini (comme le premier~exemple), le marqueur est préparé avec la
 sélection de fonte~|\listmarkfont|;
\item\item
 prépare une liste dont les entrées sont introduites par un marqueuer
 précisé par l'utilisateur, pour chaque entrée, ce ce marqueur est le
 premier argument de la commande~|\item|, le marqueur est préparé avec la
 sélection de fonte~|\listmarkfont|;
\item\hyphen
 sélectionne un marqueur prédéfini en forme de tiret court;
\item\dash
 synonyme de~|\hyphen|;
\item\bullet
 sélectionne un marqueur prédéfini en forme de puce noire;
\medskip
\item\enum
 prépare une liste dont les entrées sont introduites par un compteur
 (comme le second~exemple), le compteur est préparé avec la
 fonte~|\listmarkfont|;
\item\1
 utilise pour représenter le compteur les nombres décimaux;
\item\i
 utilise pour représenter le compteur les nombres romains en minuscules;
\item\I
 utilise pour représenter le compteur les nombres romains en majuscules;
\item\a
 utilise pour représenter le compteur les lettres de l'alphabet latin
 en~minuscules;
\item\A
 utilise pour représenter le compteur les lettres de l'alphabet latin
 en~majuscules;
\item\alpha
 utilise pour représenter le compteur les lettres de l'alphabet grec
 en~minuscules;
\item\Alpha
 utilise pour représenter le compteur les lettres de l'alphabet grec
 en~majuscules;
\medskip
\item\tag
 prépare une liste dont les entrées sont introduites par un~\em{tag},
 si ce dernier est trop large pour tenir dans la réserve située à
 gauche de l'entrée il apparaît seul sur la ligne précédent l'entrée;
\item\hang
 prépare une liste dont les entrées sont introduites par un~\em{tag},
 si ce dernier est trop large pour tenir dans la réserve située à
 gauche de l'entrée il dépasse sur l'entrée;
\item\ohang
 prépare une liste dont les entrées sont introduites par un~\em{tag},
 situé sur la ligne précédant l'entrée;
\item\diag
 prépare une liste dont les entrées sont introduites par un~\em{tag},
 ces listes peuvent être utilisées pour les diagnostiques des
 programmes (messages~d'erreur), le~\em{tag} n'est pas placé dans une
 réserve, il ouvre l'entrée;
\item\inset
 prépare une liste dont les entrées sont introduites par un~\em{tag},
 qui ouvre le paragraphe, le~\em{tag} est composé avec~|\listtagfont|;
\medskip
\item\literal
 utilise |\literalfont| pour~|\listtagfont|;
\item\symbolic
 utilise |\symbolicfont| pour~|\listtagfont|;
\item\emphasis
 utilise |\emphasisfont| pour~|\listtagfont|;
\item\natural
 utilise |\mainfont| pour~|\listtagfont|, la forme~|\main|
 a la même mnémotechnie que~les marques~précédentes;
\item\main
 synonyme de~|\natural|;
\endcslist

Des paramètres globaux affectent également la préparation des listes,
en voici le détail.
\begincslist
\item\listtagfont
 Paramètres pour la fonte utilisée dans la préparation des~\em{tags} pour
 les listes de type approprié;
\item\listmarkfont
 Paramètres pour la fonte utilisée dans la préparation des marques
 pour les listes marquées et les listes énumérées;
\item\listtagwidthadjust
 Commande lancée à chaque changement de la taille des fontes au moyen
 des commandes~|\tenpoint|,~\em{etc.}, elle ajuste la largeur
 des~\em{tags} pour les listes de ce type; on dit par exemple
 |\def\listtagwidthadjust{\listtagwidth=4em\relax}|, rappelons que
 l'unité~\li{em} voit sa valeur exacte dépendre de la fonte en cours
 d'utilisation.
\endcslist


\formalpar Classes de documents

Des classes de documents peuvent redéfinir~|\list@I| pour leurs
besoins spécifiques.

On peut ajouter des marques à celles comprises par la
commande~|\beginlist| comme dans l'exemple~suivant.
\begincode
\listoption\compact{%
  \offinterlineskip
  \def\displayopenskip{\nobreak\vskip 0pt minus 3em\relax}%
  \def\displaycloseskip{\vskip 0pt minus 3em\relax}%
}
\endcode


\formalpar Spécialisations des listes

On peut spécialiser les listes comme le montre l'exemple suivant. Il
est utilisé pour préparer les listes de diagnositiques pour la pager
de manuel~\em{jot(1)}.
\begincode
\pdef\begindiagnostics#1#2{\pdef@@beginlist{\inset\symbolic#1}{#2}}
\def\enddiagnostics{\endlist}
\let\tag@@begindisagnostics=\tag@@beginlist

\selectlocale{en}
The following diagnostic messages deserve special explanation:
\begindiagnostics
\item{illegal or unsupported format \li{'\%s'}}
 The requested conversion format specifier for \em{printf(3)} was not
 of the form |%[#][ ][{+,-}][0-9]*[.[0-9]*]?|
 where \qu{\li{?}} must be one of |[l]{d,i,o,u,x}| or
 |{c,e,f,g,D,E,G,O,U,X}|.

\item{range error in conversion}
A value to be printed fell outside the range of the data type
associated with the requested output format.

\item{too many conversions}
More than one conversion format specifier has been supplied, but only
one is allowed.
\enddiagnostics
\endcode
donne le résultat suivant
\begingroup
\enableprivate
\pdef\begindiagnostics#1#2{\pdef@@beginlist{\inset\symbolic#1}{#2}}
\def\enddiagnostics{\endlist}
\let\tag@@begindisagnostics=\tag@@beginlist
\disableprivate

\begindisplay
\selectlocale{en}
The following diagnostic messages deserve special explanation:
\begindiagnostics
\item{illegal or unsupported format \li{'\%s'}}
 The requested conversion format specifier for \em{printf(3)} was not
 of the form |%[#][ ][{+,-}][0-9]*[.[0-9]*]?|
 where \qu{\li{?}} must be one of |[l]{d,i,o,u,x}| or
 |{c,e,f,g,D,E,G,O,U,X}|.

\item{range error in conversion}
A value to be printed fell outside the range of the data type
associated with the requested output format.

\item{too many conversions}
More than one conversion format specifier has been supplied, but only
one is allowed.
\enddiagnostics
\enddisplay
\endgroup


\section Marquage de groupes de mots

On marque un groupe de mot au moyen d'une séquence appropriée pour
obtenir un effet visuel approprié. Nous avons déjà rencontré un
marquage de ce type, en la personne de~\em{la mise en relief}, on
écrit par exemple
\begincode
en la personne de~\em{la mise en relief},
\endcode
pour indiquer à \TeX\ que l'on souhaite mettre en relief le texte en
argument de la commande~|\em|, ce type de marquage est
dit~\em{abstrait} car on n'indique pas quelle fonte ou quel effet
visuel doit être obtenu. On peut marquer le texte de
façon~\em{concrète} en écrivant par exemple
\begincode
en la personne de~{\it la mise en relief},
\endcode
mais il est en général préférable d'utiliser un marquage~\em{abstrait}
plutôt qu'un marquage~\em{concret} lorsqu'on prépare un texte, pour
plusieurs raisons:
\beginlist
\item
l'aspect visuel adéquat peut dépendre du contexte, pour mettre en
relief un groupe de mot dans un texte en italiques, il est souhaitable
d'utiliser des romaines et non des italiques; dans ce cas le marquage
concret est difficile puisque l'utilisateur doit savoir précisément
quelle fonte est utilisée, en revanche le marquage abstrait avec~|\em|
se charge de composer le texte avec une fonte adaptée;

\item
l'aspect visuel adéquat peut dépendre de la coutume, dans ce cas il
faut apprendre les règles pour chaque coutume, et lorsque plusieur
règles coexistent, celles-cis doivent être employées de façon
cohérente à travers le manuscrit; l'utilisation d'un marquage abstrait
permet de déléguer ces détails au système de préparation;

\item
des traitements supplémentaires peuvent profiter des marquages
abstraits, comme par exemple la préparation d'un index des noms cités,
si tous les noms sont marqués de façon convenable.
\endlist
CADET \TeX\ rend quelques services pour le marquage abstrait que nous
discutons ci-dessous.


\formalpar Mise en relief d'un groupe de mots

On peut mettre en relief un groupe de mots au moyen de~|\em|, on
obtient par exemple
\begindisplay
\parindent=0pt
\def\example{Isaïe avait déjà demandé avant vous: \em{Quomodo
cecidisti de c½lo, Lucifer, qui maneo riebaris!}\par}
\thisparagraph\rm\example
\smallskip
%\thisparagraph\sl\example
\thisparagraph\it\example
%\thisparagraph\bf\example
%\thisparagraph\sf\example
\enddisplay
à partir de l'extrait de manuscrit
\begincode
\def\example{Isaïe avait déjà demandé avant vous: \em{Quomodo
cecidisti de c½lo, Lucifer, qui maneo riebaris!}\par}
\thisparagraph\rm\example
\smallskip
\thisparagraph\it\example
\endcode

\formalpar Mise en relief symbolique

On obtient une mise en relief symbolique grâce à~|\sy|, comme dans
\begindisplay
\sy{Note importante.} Plop plop fizz fizz.
\enddisplay
XXX ce nom curieux vient de groff\_man(7).


\formalpar Expression littérales

Une expression littérale est à prendre telle quelle, on utilise le
marquage littéral pour indiquer une saisie pour un système
informatique, etc. |\li|
\begindisplay
\li{literal}
\enddisplay

\formalpar Marquage des noms propres

On marque les noms propres avec~|\nm|, comme dans
\begindisplay
\noindent
\nm{Mme des Aubels}, Maurice \nm{d'Esparvieu}, ou~D.~\nm{Hilbert}.
\enddisplay
\begincode
\nm{Mme des Aubels}, Maurice \nm{d'Esparvieu}, ou D.~\nm{Hilbert}.
\endcode
Les prénoms, les initiales et les dignités doivent faire partir du
marquage. Lorsque plusieurs noms sont accolés, comme par exemple dans
les noms de théorèmes, chaque nom est marqué, mais lorsqu'il s'agit
d'un nom composé, il n'y a qu'une seule marque. On écrit donc
\begindisplay
\parindent=0pt
Théorème de \nm{Hahn}-\nm{Banach}\par
\nm{Mme Royal-Hollande}\par
\enddisplay
\begincode
Théorème de \nm{Hahn}-\nm{Banach}\par
\nm{Mme Royal-Hollande}\par
\endcode

\formalpar Guillemets

On obtient des guillemets avec~|\qu|, qui propose des guillemets de
second rang, on écrit ainsi
\begindisplay
\parindent=0pt
\qu{Où il apparaît que, comme l'a dit un vieux poète grec, \qu{rien n'est
plus doux qu'Aphrodite d'or}.}
\enddisplay
\begincode
\qu{Où il apparaît que, comme l'a dit un vieux poète grec, \qu{rien
n'est plus doux qu'Aphrodite d'or}.}
\endcode
\sy{Remarque.} Certains préfèrent intercaler ainsi les points et les
fins de guillemets:~|.}}| au lieu de~|}.}| comme on l'a écrit
ci-dessus. Avec cette convention, les points et les virgules flottent
vers la gauche à travers les guillemets. Ceci n'est pas recommandé.
\smallbreak

Les guillemets de premier et de second rang sont définis par la
coutume, pour modifier ces paramètres on peut utiliser un définition
ressemblant~à:
\begincode
\def\quote@fr{\quote@S
  \pnguillemetleft\pnguillemetright
  \pndblquoteleft\pndblquoteright
}
\endcode


\formalpar Domaine du manuel

Les marques regroupées ici sont adaptées de celle proposées
dans~\xr{groff_mdoc(7)}, elles sont avant-tout destinées à la
préparation de manuels informatiques, comme par exemple la
documentation d'une bibliothèque~C.
\begincslist
\item\pa
pour les n½uds du système de fichiers, comme~\pa{/etc/rc.conf}
ou~\pa{$\br{HOME}/my_backup}, obtenus par les saisies~|\pa{/etc/rc.conf}|
et~|\pa{$\br{HOME}/my_backup}|, dans l'argument les caractères~|~|, |_|
et~|$| sont des lettres;%$

\item\xr
pour les références extérieures, comme notamment les pages de
manuel~\xr{groff_mdoc} et les URL~\xr{http://www.freebsd.org}, on a
saisi respectivement |\xr{groff_mdoc}|
et~|\xr{http://www.freebsd.org}|, les mêmes conventions que pour~|\pa|
sont vraies pour l'argument;

\item\bc
entourer l'argument d'accolades, comme~\pa{$\br{HOME}/.vacation}
produit par la séquence~|\pa{$\br{HOME}/.vacation}|;

\item\bk
entourer l'argument de crochets;

\item\ev
variables d'environnement, comme~\ev{PATH};

\item\dv
variables définies, par exemple avec un~\#define en~C,
soit~\dv{O_RDONLY}.
\endcslist
La commande~|\xr| peut-être considérée d'un intérêt légèrment plus
général que les autres.

\formalpar Désactivation du marquage
On utilise~|\disablemarkup| pour remplacer les commades de marquage
décrites dans cette section par des~NOP; il existe des exceptions,
pour~|\qu| qui produit des guillemets~ASCII, et pour~|~| qui produit
une~espace.


\bye

%%% End of file `text.tex'
