%%% math.tex -- Préparation du texte du mode mathématique

% Author: Michael Grünewald
% Date: Mon Oct  9 15:51:05 CEST 2006

% Copyright (C) 2006, 2013 Michael Grünewald
% All rights reserved.
%
% This file is part of Bhrìd TeX.
%
% Bhrìd TeX is free software: you can redistribute it and/or modify
% it under the terms of the GNU General Public License as published by
% the Free Software Foundation, either version 3 of the License, or
% (at your option) any later version.
%
% Bhrìd TeX is distributed in the hope that it will be useful,
% but WITHOUT ANY WARRANTY; without even the implied warranty of
% MERCHANTABILITY or FITNESS FOR A PARTICULAR PURPOSE.  See the
% GNU General Public License for more details.
%
% You should have received a copy of the GNU General Public License
% along with Bhrìd TeX.  If not, see <http://www.gnu.org/licenses/>.


\class bhridpaper
\enablehverbatim

\title Préparation du texte du mode mathématique
\author Michael \nm{Grünewald}

Nous décrivons les services que rend BHRID~\TeX\ dans la préparation du
texte du mode mathématique, soit \qu{les formules}.


\section Ensembles de nombres

Les ensembles de nombre usuels peuvent être obtenus à partir des
abréviations suivantes, en mode mathématique:
\begincslist
\item\N l'ensemble~$\N$ des entiers naturels;
\item\Z l'ensemble~$\Z$ des entiers relatifs;
\item\Q l'ensemble~$\Q$ des nombres reationnels;
\item\R l'ensemble~$\R$ des nombres réels;
\item\C l'ensemble~$\C$ des nombres complexes.
\endcslist
Les ensembles de nombres sont préparés avec la séléction de famille de
fontes~|\scalarsetfont|, qui peut être modifiée. Par exemple
\begincode
\let\scalarsetfont=\sf
L'ensemble~$\Z$ des entiers relatifs est le symmétrisé du demi-groupe
des entiers-naturels~$\N$.
\endcode
donne le résultat suivant
\begindisplay
\let\scalarsetfont=\sf
\noindent
L'ensemble~$\Z$ des entiers relatifs est le symmétrisé du demi-groupe
des entiers-naturels~$\N$.
\enddisplay
Les ensembles de nombres peuvent figurer dans dans les indices et dans
les exposants, on écrit par exemple~|$\Q^\N$| pour obtenir~$\Q^ \N$.


\formalpar Sur les nombres complexes

On écrit la partie réelle et la partie imaginaire des nombres complexes
au moyen de
\begincslist
\item\Re la partie réelle~$\Re z = x$ de~$z$;
\item\Im la partie imaginaire~$\Im z = x$ de~$z$;
\endcslist
et~$z = \Re z + i\Im z$ s'écrit~|$z = \Re z + i\Im z$|.

\formalpar Autres ensembles de scalaires

Il est très facile d'ajouter de nouveaux ensembles de scalaires. Par
exemple pour les nombres décimaux on écrit dans le manuscrit
\begincode
\def\D{{\mathscalarsetfont D}}
\endcode
et une définition ressemblant à la précédente permettent de créer des
abréviations pour les quaternions d'\nm{Hamilton}~$\scalarsetfont H$
et les octaves de~\nm{Cayley}~$\scalarsetfont O$.

Ces définitions ne font pas partie du jeu naturel des définitions de
BHRID~\TeX\ parceque~|\D| peut être utilisé en géométrie différentielle
pour práprer les dérivations, et~|\O| est réservé pour désigner le
groupe orthogonal d'un espace quadratique.


\formalpar Corps de nombres

La façon de décrire les corps de nombres dans le manuscrit est un cas
particulier de celle utilisée pour les sous-anneaux engendrés par une
partie finie. Par exemple si~$k$ est un sous-anneau de~$l$ et si~$x$
et~$y$ sont des éléments de~$l$, le sous-anneau~$k[x,y]$ de~$l$
engendré par~$k$,~$x$ et~$y$ est saisi~|$k[x,y]$|; si~$\cal X$ est une
partie de~$l$, le sous-anneau~$k[{\cal X}]$ de~$l$ engendré par~$k$
et~$\cal X$ est saisi~|$k[{\cal X}]$|. Ceci s'applique en particulier
aux anneaux de polynômes et aux corps de nombres, ainsi~$\Q[i,\sqrt2]$
est saisi~|$\Q[i,\sqrt2]$|.


\section Géométrie

Des abréviations exitent poue les groupes classiques


Les groupes classiques et les espaces géométriques usuels sont
préparés avec la famille de fontes sélectionnée par~|\usualsetfont|,
que l'on peut modifier selon le même procédé que~|\scalarsetfont|.


\formalpar Groupes classiques

Les groupes classiques peuvent être obtenus à partir des abréviations
suivantes:
\begincslist
\item\GL groupe linéaire général d'un espaces vectoriel;
\item\SL groupe spécial linéaire d'un espace vectoriel;
\item\O groupe orthogonal d'un espace quadratique;
\item\SO groupe spécial orthogonal d'un espace quadratique;
\item\GO groupe des similitudes d'un espace quadratique;
\item\Sp groupe symplectique d'un espace symplectique;
\item\GA groupe affine d'un espace affine;
\item\PGL groupe des homographies d'un espace projectif;
\item\PSL groupe des homographies d'un espace projectif;
\item\PSO groupe des homographies d'un espace projectif stabilisant
une quadrique de référence;
\endcslist
On écrit par exemple~|\GL_n|, |\GL_n\C|, |\GL(n,\C)| ou~|\GL(E)|, pour obtenir
respectivement~$\GL_n$, $\GL_n\C$, $\GL(n,\C)$ ou~$\GL(E)$.

\formalpar Espaces géométriques

Les espaces géométriques usuels peuvent quant à eux être obtenus à
partir des abréviations suivantes:
\begincslist
\item \A espace affine, on écrit par exemple~|\A^n|, ou~|\A^n_k| pour
      l'espace affine de dimension~$n$ sur le corps~$k$, ce qui
      donne~$\A^n$, ou~$\A^n_k$;
\item \P espace projectif, on saisit~|\P^n|, |\P^n_k| ou~|\P(k^{n+1})|
      pour l'espace projectif de dimension~$n$ sur le corps~$k$, ce
      qui donne~$\P^n$, $\P^n_k$ ou~$\P(k^{n+1})$, on peut aussi
      écrire~|\P E| pour obtenir l'espace projectif~$\P E$ associé à
      l'espace vectoriel~$E$;
\item \S sphère, la saisie~|\S^n| donne~$\S^n$ désignant la sphère de
      dimension~$n$;
\item \G grassmanniennes, on écrit~|\G_n E| pour obtenir la
      grasmannienne~$\G_n E$ des~$n$-plans contenus dans~$E$, en ce
      qui concerne les sous-espaces linéaires des espaces projectifs,
      on peut écrire~$\G_n{\P E} = \G_{n+1} E$;
\endcslist
Remarquons finalement que~|\S| et~|\P| ont un autre sens dans le mode
texte, où ces séquences produisent les symboles~`\S' et~`\P.'


\formalpar Autres espaces géométriques

On ajoute facilement de nouvelles abréviations pour représenter des
espaces géométriques ne figurant pas dans les listes précédentes, par
exemple si on souhaite saisir~|\T^n| pour le tore de dimension~$n$ on
peut ajouter la définition~|\def\T{\usualsetfont T}| dans le
manuscrit.

\formalpar Géométrie algébrique

\begincslist
\item \var la variété~$\var({\ef a})$ lieu des zéros de l'idéal~$\ef a$
      est saisie~|\var({\ef a})|;
\item \idl l'idéal~$\idl(X)$ des fonctions nulles sur~$X$ est
      saisi~|\idl(X)|;
\item \rad le radical~$\rad({\ef a})$ de l'idéal~$\ef a$ est
      saisi~|\rad({\ef a})|;
\item \Reg l'espace~$\Reg_X U$ des fonctions régulières sur
      l'ouvert~$U$ de la variété~$X$ est saisi~|\Reg_X U|, on peut
      obtenir les localisations~$\Reg_{x\,X}U$ et~$\Reg_x U$
      par~|\Reg_{x\,X}U| et~|\Reg_x U|.
\endcslist


\section Rubrique à brac

\formalpar Morphismes

Ces séquences utilisent la famille de fontes sélectionnée
par~|\categoricfont| pour préparer le matériel qu'elles ajoutent au
document, à l'exception de~|\circ|.
\begincslist
\item \circ pour la composition~$g\circ f$ de~$f$ et de~$g$ on
      écrit~|g \circ f|;
\item \homo pour les homomorphsimes de structure on
      obtient~$\homo(A,B)$ ou~$\homo_{k-{\rm alg}}(A,B)$
      avec~|\homo(A,B)| ou~|\homo_{k-{\rm alg}}(A,B)|;
\item \endo pour les endomorphsimes de structure on
      obtient~$\endo(A,B)$ avec~|\endo(A,B)|;
\item \isom pour les isomorphsimes de structure on
      obtient~$\isom(A,B)$ avec~|\isom(A,B)|;
\item \auto pour les automorphsimes de structure on
      obtient~$\auto(A,B)$ avec~|\auto(A,B)|.
\item \ker noyau d'un morphsime, on obtient~$\ker f$ avec~|\ker f|;
\item \im image d'un morphisme, on obtient~$\im f$ ave~|\im f|, à ne
      pas confondre avec~|\Im| pour la partie imaginiare des nombres
      complexes.
\endcslist


\formalpar Algèbre linéaire

Ces séquences utilisent la famille de fontes sélectionnée
par~|\tensorsetfont| pour préparer le matériel qu'elles ajoutent au
document.

\begincslist
\item \trace trace d'une application linéaire, on obtient~$\trace f$
      avec~|\trace f|;
\item \Mat espace de matrices, on écrit par exemple~|\Mat_n k|
      ou~|\Mat_{b_2\,b_1}f|, pour
      obtenir $\Mat_n k$ ou~$\Mat_{b_2\,b_1}f$ (notation supposée pour
      la matrice de l'application linéaire~$f$ de la base~$b_1$ vers
      la base~$b_2$);
\item \Ten espace de tenseurs ou algèbre tensorielle, on écrit par
      exemple~|\Ten E|, |\Ten^p E|, |\Ten_q E| ou~|\Ten^p_q E| pour
      obtenir respectivement l'algèbre tensorielle~$\Ten E$ engendrée
      par les vecteurs de~$E$, l'espace~$\Ten^p E$ des tenseurs~$p$ fois
      covariants --- ou espaces de formes $p$-linéaires --- sur~$E$,
      l'espace des~$\Ten_q E$ des des tenseurs~$q$ fois contravariants
      --- ou $q$-vecteurs --- sur~$E$, et l'espace~$\Ten^p_q E$ des
      tenseurs $p$~fois covariants et~$q$ fois contravariants;
\item \Sym l'espace~$\Sym^p E$ des $p$-covecteurs symétriques
      sur~$E$ est saisi~|\Sym^p E|, on l'utilise autrement de la même
      façon que~|\Ten|;
\item \Alt l'espace~$\Alt^p E$ des $p$-covecteurs antisymétriques
      sur~$E$ est saisi~|\Alt^p E|, on l'utilise autrement de la même
      façon que~|\Ten|.
\endcslist


\formalpar Ensembles

Les séquences~|\card| et~|\pr| utilisent la famille de fontes
sélectionnée par~|\categoricfont| pour préparer le matériel qu'elles
ajoutent au document.
\begincslist
\item \Pset pour~\em{power set}, l'ensemble des parties~$\Pset E$
      de~$E$ est saisi~|\Pset E|, on peut écrire~$\Pset_n E$ pour
      l'ensemble des parties de~$E$ dont le cardinal est~$n$;
\item \card on peut saisir~|\card E| pour obtenir~$\card E$ désignant
      le cardinal de l'ensemble~$E$;
\item \pr la projection~$\pr_J : \prod_{i\in I} E_i \to \prod_{j\in J}
      E_j$ pour~$J\subset I$ est
      saisie~|\pr_J : \prod_{i\in I} E_i \to \prod_{j\in J} E_j|.
\endcslist

\formalpar Espaces de fonctions

\begincslist
\item \Boun l'espace~$\Boun(X,\R)$ des fonctions bornées de~$X$
      dans~$\R$ est saisi~|\Boun(X,\R)|;
\item \Cont l'espace~$\Cont(X,\R)$ des fonctions continues de~$X$
      dans~$\R$ est saisi~|\Cont(X,\R)|;
\item \Func l'espace~$\Func(A,B)$ de toutes les applications de~$A$
      dans~$B$ est saisi~|\Func(A,B)|.
\endcslist


\formalpar Groupe symétrique

\begincslist
\item \Gsym le groupe symétrique~$\Gsym(E)$ ou~$\Gsym_n$ est
      saisi~|\Gsym(E)| ou~|\Gsym_n|;
\item \Galt le groupe alterné~$\Galt(E)$ ou~$\Galt_n$ est
      saisi~|\Galt(E)| ou~|\Galt_n|;
\item \sign le morphsime signature~$\sign:\Gsym(E) \to \C^*$ est
      saisi~|\sign:\Gsym(E) \to \C^*| (ne pas confondre avec la
      fonction signe, saisie~|\sgn|);
\endcslist


\formalpar Arithmétique

On sait déjà comment saisir~$\N$ et~$\Z$ les ensembles de nombres
entiers.
\begincslist
\item \Primes l'ensemble~$\Primes$ de tous les nombres premiers est
      saisi~|\Primes|;
\item \div on obtient $a \div b$, lu \qu{$a$ divise~$b$} en
      saisissant~|a \div b|, la négation est obtenue avec~|\ndiv|
      comme dans~$a \ndiv b$;
\item \times on obtient~$a\times b$ avec~|a \times b|;
\item \schooldiv on obtient~$a\schooldiv b$ avec~|a \schooldiv b|, le
      nom de cette séquence est assez long car ce symbole est très peu
      utilisé, on écrit normalement~$a/b$ tandis que~$a\schooldiv b$
      ne se rencontre guère que dans les livres d'école primaire, mais
      un manuscrit nécessitant de nombreux~|\schooldiv| peut fabriquer
      un nom court avec un~|\let|;
\item \pgcd on obtient~$\pgcd(a,b)$ ou~$\pgcd X$ avec~|\pgcd(a,b)|
      ou~|\pgcd X|;
\item \ppcm on obtient~$\ppcm(a,b)$ ou~$\ppcm X$ avec~|\ppcm(a,b)|
      ou~|\ppcm X|;
\endcslist
Rappelons que dans PLAIN~\TeX\ la séquence~|\div| produit le symbole
appelé ici~|\schooldiv|.


\formalpar Fonctions usuelles

Ces séquences utilisent la famille de fontes sélectionnée
par~|\functionfont| pour préparer le matériel qu'elles ajoutent au
document.
\begincslist
\item \sgn \em{Fonction signe}~$\sgn x$;
\item \sin \em{Sinus}~$\sin x$;
\item \cos \em{Cosinus}~$\cos x$;
\item \tg \em{Tangente}~$\tg x$;
\item \cotg \em{Cotangente}~$\cotg x$;
\item \Arctg \em{Arc tangente}~$\Arctg x$;
\item \Arccotg \em{Arc cotangente}~$\Arccotg x$;
\item \Arcsin \em{Arc sinus}~$\Arcsin x$;
\item \Arccos \em{Arc cosinus}~$\Arccos x$;
\item \Arg \em{Argument principal des nombres complexes}~$\Arg x$;
\item \ch \em{Cosinus hyperbolique}~$\ch x$;
\item \sh \em{Sinus hyperbolique}~$\sh x$;
\item \th \em{Tangente hyperbolique}~$\th x$;
\item \coth \em{Cotangente hyperbolique}~$\coth x$;
\item \Argch \em{Argument cosinus hyperbolique}~$\Argch x$;
\item \Argsh \em{Argument sinus hyperbolique}~$\Argsh x$;
\item \Argth \em{Argument tangente hyperbolique}~$\Argth x$;
\item \Argcoth \em{Argument cotangente hyperbolique}~$\Argcoth x$;
\endcslist


\formalpar Derniers regrets

\begincslist
\item \dim dimension;
\item \sup supremum;
\item \inf infimum;
\item \spm spectre des idéaux maximaux;
\endcslist


\section Macros ancillaires

\formalpar Texte ordinaire en mode mathématique

Pour insérer du texte ordinaire dans le mode mathématique on
utilise~|\mbox|, comme dans l'exemple
$$
a + b > 0
\qquad\mbox{et}\qquad
a < c
$$
qui s'écrit
\begincode
$$
a + b > 0
\qquad\mbox{et}\qquad
a < c
$$
\endcode
dans le manuscrit.

\formalpar Ensembles

Il est courant dans un teexte mathématique de décrire un ensemble par
une relation collectivisante, comme par exemple
$$
\Set{x\in\R \mid \sin x > 0 }
$$
que l'on peut saisir~|\Set{x\in\R \mid \sin x > 0}|. La
commande~|\Set| offre pluseurs avantages par rapport à la saisie
naïve~|\{x\in\R\mid\sin x > 0}|, les accolades ouvrantes et fermantes
ainsi que la barre de séparation ont leur taille calculée
automatiquement et les espacements sont meilleurs. On peut obtenir
$$
\Set{ {1\over p} \mid p \in \Primes}
$$
comme~$\Set{ {1\over p} \mid p \in \Primes}$ en
saisissant~|\Set{{1\over p} \mid p \in \Primes}|.
\bye


%%% End of file `math.tex'
