%% plainsample.tex -- Exemple de document rédigé avec Cadet TeX

% Author: Michael Grünewald
% Date: Mer 11 avr 2007 10:55:33 CEST

% Copyright (C) 2006-2014 Michael Grünewald
% All rights reserved.
%
% This file is part of Cadet TeX.
%
% Cadet TeX is free software: you can redistribute it and/or modify
% it under the terms of the GNU General Public License as published by
% the Free Software Foundation, either version 3 of the License, or
% (at your option) any later version.
%
% Cadet TeX is distributed in the hope that it will be useful,
% but WITHOUT ANY WARRANTY; without even the implied warranty of
% MERCHANTABILITY or FITNESS FOR A PARTICULAR PURPOSE.  See the
% GNU General Public License for more details.
%
% You should have received a copy of the GNU General Public License
% along with Cadet TeX.  If not, see <http://www.gnu.org/licenses/>.


% Ce fichier est préparé avec le format Cadet TeX, il montre de quelle
% façon saisir des éléments de texte ordinaire. Ce fichier est
% abondamment commenté.


%%% 1. Titre du document

% On commence par créer un titre au document. Ce titre est formé d'un
% espace vertical, d'un texte centré et d'un autre espace vertical. Il
% est important d'utiliser la commande `topglue', au lieu de `vglue',
% `vskip' ou ses dérivés `smallskip' etc., pour insérer un espace
% vertical au sommet d'une page, car TeX insère au sommet de la page
% un blanc spécial, contrôlé par `topskip'. En utilisant `topglue' on
% tient compte de ce saut spécial.
%
% Cf. TeXbook, p. 113--114.

\magnification\magstep1
\topglue 14mm
\centerline{\titlefont Exemple de document}
\vskip 14mm

%%% 2. Exemples d'abréviations simples

% Il est courante et utile de définir quelques abréviations pour
% faciliter la saisie de certaines expressions. Le TeXbook est
% plusieurs fois cité dans le texte, et une abréviation est définie
% pour faciliter la saisie de son titre.
%
% Pour forcer un espace après le résultat de `\TeXbook', on écrit
% `\TeXbook\ '.

\def\TeXbook{\ti{\TeX book}}

Ce document sert d'exemple pour la préparation d'un texte avec
\brTeX. Le fichier source traité par~\TeX\ pour le produire a le
nom~\pa{plainsample.tex}. Il est utile de consulter ce fichier en
même temps que ce document, car le code source est abondamment
commenté et présente de nombreux renvois précis au \TeXbook.


%%% 3. Un exemple de paragraphe

% Attention aux faux-amis: un `paragraph' anglais est un alinéa
% français; et un paragraphe français est une succession d'un ou
% plusieurs alinéas, éventuellement précédé d'un titre. Ce que les
% manuels de Cadet TeX appelent donc du terme anglais `section' est
% un `paragraphe' français.
%
% Le format Cadet TeX propose une procédure `section' simplette, qui
% peut être redéfinie dans une classe de document.

\section 1. \TeX nologie

Le premier alinéa d'un paragraphe n'est pas précédé d'un retrait, à la
différence des suivants. \TeX\ reconnaît la fin d'un alinéa lorsqu'il
trouve une ligne vide dans votre manuscrit. Cette ligne vide est
obtenue en insérant deux sauts de lignes consécutifs dans le fichier.

Les alinéas suivant le premier sont précédés d'un retrait, qu'on
appelle parfois \em{indentation}. (L'avez-vous vu?) La machine
détermine automatiquement les meilleurs endroits pour aller à la ligne
dans un alinéa, et coupe les mots; comme l'explique en détail
le~\TeXbook.

Des mesures spéciales ont été prises pour aider les rédacteurs à
respecter les conventions typographiques en vigueur dans leur culture.
À~condtion d'indiquer à l'ordinateur dans quelle langue est écrit le
texte, les espaces autour des signes de ponctuation sont placés
automatiquement et produisent le bons résultat, les règles de coupure
des mots sont respectées. Voici pour le montrer un extrait
de~\em{Wuthering heights}, de~Emily~Brontë:

%%% 4. Exemples d'affichage

% L'affichage est une partie de texte mise en exergue. Pour citer une
% portion importante de texte, on utilise généralement un affichage.
% Le code suivant montre quelques exemples de tels affichages.

\begindisplay
%
% On modifie les paramètres de l'algorithme de coupure des lignes, afin
% que celui-ci produise de nombreuses coupures de mots. Il s'agit de
% montrer que les coupures obtenues respectent les règles de
% l'orthographe anglaise. Pour reproduire ce genre de présentation, on
% peut omettre ce commentaire et les trois lignes qui le suivent.
%
\selectlocale{en}{GB}% Le texte qui suit est en anglais
\smallsize%             On veut une taille de caractère plus petite
\pretolerance=-1
\hyphenpenalty=-50
\noindent
\llap{``}%''
In vapid listlessness I leant my head against the window, and
continued spelling over Catherine Earnshaw, Heathcliff, Linton, till
my eyes closed; but they had not rested five minutes when a glare of
white letters started from the dark, as vivid as spectres---the air
swarmed with Catherines; and rousing myself to dispel the obtrusive
name, I discovered my candle-wick reclining on one of the antique
volumes, and perfuming the place with an odour of roasted calf-skin.
I snuffed it off, and, very ill at ease under the influence of cold
and lingering nausea, sat up and spread open the injured tome on my
knee.  It was a Testament, in lean type, and smelling dreadfully
musty: a fly-leaf bore the inscription ``Catherine Earnshaw, her
book,'' and a date some quarter of a century back.  I shut it, and
took up another and another, till I had examined all.  Catherine's
library was select, and its state of dilapidation proved it to have
been well used, though not altogether for a legitimate purpose:
scarcely one chapter had escaped, a pen-and-ink commentary---at least
the appearance of one---covering every morsel of blank that the
printer had left.  Some were detached sentences; other parts took the
form of a regular diary, scrawled in an unformed, childish hand.  At
the top of an extra page (quite a treasure, probably, when first
lighted on) I was greatly amused to behold an excellent caricature of
my friend Joseph,---rudely, yet powerfully sketched. An immediate
interest kindled within me for the unknown Catherine, and I began
forthwith to decipher her faded hieroglyphics.''
\enddisplay
Vous pouvez y voir que les espaces autour de certains signes de
ponctuation, comme le \em{deux-points} ou le \em{point-virgule}, sont
différents des espaces entourant les mêmes symboles dans un texte en
français.

Un affichage tel que le précédent ne termine pas nécessairement
l'alinéa, c'est une bonne chose car on utilise souvent ce type de
documentation pour citer un texte au cours d'une discussion. Pour
terminer l'alinéa après un affichage, il suffit de laisser une ligne
vide dans le manuscrit.

% Une partie importante de la suite est du texte en anglais. C'est un
% extrait de l'article de Alan D. Sokal.

\selectlocale{en}{GB}

\section 2. Manifold Theory: (W)holes and boundaries

Luce Irigaray, in her famous article ``Is the Subject of Science
Sexed?'', pointed out~that:
\begindisplay
\def\french{\selectlocale{fr}{FR}}% Abréviation commode
\smallsize
\noindent
the mathematical sciences, in the theorey of wholes
{\french[théorie des ensembles]}, concern themselves with
closed and open spaces\dots\ They concern themselves very little with
the question of partially open, with wholes that are not clearly
delineated {\french[ensembles flous]}, with any analysis of
the problem of borders~{\french[bords]}\dots
\enddisplay
In~1982, when Iragaray's essay first appeared, this was an incisive
criticism: differential topology has traditionnaly privileged the
study of what are known technically as ``manifolds without
boundary''. However, in the past decade, under the impetus of the
feminist critique, some mathematicians have given renewed attention to
the theory of ``manifolds with
boundary'' [Fr.~\em{\selectlocale{fr}{FR}variétés à bord}\/]. Perhaps not
coincidentally, it is precisely these manifolds that arise in the new
physics of conformal field theory, superstring theory and quantum
gravity.
\smallskip
\rightline{Alan D. Sokal, \emphasis Transgressing the Boundaries:}
\rightline{\emphasis Towards a Transformative Hermeneutics of Quantum Gravity}

% À la fin de notre texte en anglais, on sélectionne de nouveau des
% paramètres régionaux correspondant à la langue dans laquelle est
% écrit le document.

\selectlocale{fr}{FR}

%%% 5. Exemples de listes

\section Exemples de listes

\beginroster
\item ceci est le premier terme d'une liste;
\item ceci est le second terme d'une liste, ce second terme est long
  est permet d'observer comment est préparé un terme s'étendant sur
  plusieurs lignes;
\item ce troisième terme est de longueur intermédiaire.
\endroster

\beginenumeration
\item 1 ceci est le premier terme d'une liste;
\item 2 ceci est le second terme d'une liste, ce second terme est long
  est permet d'observer comment est préparé un terme s'étendant sur
  plusieurs lignes;
\beginroster
\item ceci est le premier terme d'une liste;
\item ceci est le second terme d'une liste, ce second terme est long
  est permet d'observer comment est préparé un terme s'étendant sur
  plusieurs lignes;
\endroster
\item 3 ce troisième terme est de longueur intermédiaire.
\endenumeration

Les termes de cet exemple de liste sont:
\begindescription
\item Le premier
  Ceci est le premier terme d'une liste;
\item Le deuxième
  Ceci est le second terme d'une liste, ce second terme est long est
  permet d'observer comment est préparé un terme s'étendant sur
  plusieurs lignes;
\item Le troisième
  Ce troisième terme est de longueur intermédiaire.
\enddescription

\begindescription
\item rm
  bascule vers la fonte définie par~\fn{setnormalfont} et sélectionne
  la famille~\va{rmfontfamily} (caractères romains), on attend que
  cette fonte soit celle utilisée pour le texte ordinaire du document;

\item tt
  bascule vers la fonte définie par~\fn{setnormalfont} et sélectionne
  la famille~\va{ttfontfamily} \em{(typewriter type)\/};

\item sf
  bascule vers la fonte définie par~\fn{setnormalfont} et sélectionne
  la famille~\va{sffontfamily} (caractères sans serifs).

\enddescription

\section 3. Encore de la \TeX nologie

%%% X. Affichages mathématiques

\section 4. Affichages mathématiques

Réalisons quelques uns des \em{challenges} que propose le \ti{\TeX
book}.
$$
{\bf S}^{-1}{\bf T}{\bf S}
 = {\bf dg}(\omega_1,\dots,\omega_n)
 = {\bf\Lambda}
$$

$$
{\rm Pr}(\, m = n \mid m + n = 3\,)
$$

$$
\sin 18^\circ = {1\over 4}(\sqrt 5 - 1)
$$

$$
k = 1,38 \times 10^{-16} {\rm erg}/{^\circ}K
$$

$$
\bar\Phi \subset NL_1^* /N = {\bar L}_1^* \subset
\cdots \subset NL_n^*/N = {\bar L}_n^*
$$

$$
I(\lambda) = \int\!\!\!\int_D g(x,y)e^{i\lambda h(x,y)} \,dx\,dy
$$

$$
\int_0^1 \cdots \int_0^1 f(x_1,\dots,x_n)\,dx_1\dots dx_n
$$

$$
x_{2m} \equiv \cases{
  Q(X_m^2 - P_2 W_m^2) -2S^2 & ($m$ odd)\cr
  P_2^2(X_m^2 - P_2 W_m^2) -2S^2 & ($m$ even)\cr
}\quad(\mod N)
$$

$$
(1+x_1z + x_1^2 z^2 + \cdots)\dots(1+x_nz + x_n^2z^2+\cdots)
 = {1\over (1-x_1z) \dots(1-x_nz)}
$$

%%% X. Un exemple

% On reproduit ici un court extrait du livre de Joe Harris, Algebraic
%  Geometry.
%
% Cet extrait figure p. 38.

\section X. Un exemple de texte mathématique

% Après une commande section, le premier élement de texte débute un
% paragraphe. Il et donc incorrect de saisir
%
%   \section X. Un exemple de texte mathématique
%
%   \selectlocale{en}{US}
%
%   Next, suppose we have ...
%
% Cette saisie crée en effet un paragraphe vide, puisqu'après
% `selectlocale' la double ligne blanche renvoie TeX au mode vertical.
\selectlocale{en}{US}%
Next, suppose we have a subvariety~$X$ of~$Y\times\P^2$ and would like
to make the same statement. We can do this by choosing a
point~$p\in\P^2$ and first mapping~$Y\times(\P^2\setminus\set{p})$
to~$Y\times\P^1$, then projecting from~$Y\times\P^1$ to~$Y$. This
works except where~$X$ meets the locus~$Y\times\set{p}$, i.e., if we
let~$V\subset Y$ be the closed subset~$\set{q\in Y\mid (q,p)\in X}$ it
shows that the image~$\bar X = \pi_1(X)\subset Y$ intersects the open
set~$U = Y \setminus V$ in a closed subset of~$U$. But since~$\bar X$
contains~$V$, it follows that~$\bar X$ is closed in~$Y$. In general,
this argument establishes the following theorem.

\proclaim Theorem {3.12}. Let~$Y$ be any variety
and~$\pi:Y\times\P^n\to Y$ the projection on the first factor. Then
the image~$\pi(X)$ of any closed subset~$X\subset Y \times\P^n$ is a
closed subset of~$Y$.

As an immediate consequence of this, we may combine it with
Exercise~2.24 to deduce the following fundamental theorem.

\proclaim Theorem {3.13}. If~$X\subset\P^n$ is any projective variety
and~$\phi:X\to\P^m$ any regular map then the image of~$\phi$ is a
projective subvariety of~$\P^m$.

\selectlocale{fr}{FR}

\vfill
\bye

%%% End of file `plainsample.tex'
