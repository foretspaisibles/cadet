%%% pfssintro.tex -- Introduction au PFSS

% Cadet TeX (https://github.com/foretspaisibles/cadet)
% This file is part of Cadet TeX.
%
% Copyright © 2001–2021 Michaël Le Barbier
% All rights reserved.

% This software is governed by the CeCILL-B license under French law and
% abiding by the rules of distribution of free software.  You can  use,
% modify and/ or redistribute the software under the terms of the CeCILL-B
% license as circulated by CEA, CNRS and INRIA at the following URL
% "https://cecill.info/licences/Licence_CeCILL-B_V1-en.txt"


\class cadetpaper
\macro zapfd
\title Introduction à la sélection des fontes

% Pour les exemples, on a besoin d'une macro produisant des caractères
% dans une forte taille. Pour compenser cela, on utilise une teinte
% éclaircie, bien que l'utilisation de la couleur n'ait pas encore
% été ajouté à Cadet TeX.

\def\fontsample#1#2{%
   \hbox{\disableconventionpunc#1\selectfontsize{38}%
   \testdriver{dvips}\ifdriver\special{color push CadetBlue}\fi
   {#2}\relax
   \testdriver{dvips}\ifdriver\special{color pop}\fi
   }%
}

\def\sampleline#1{\nobreak\bigskip\centerline{#1}\bigbreak}

% On charge quelques fontes Adobe.

\begingroup
\gdef\setTimes{%
  \setfontfamily{tx}%
  \setfontshape{r}%
  \setfontweight{m}%
  \setfontwidth{m}%
  \setfontpage{CT1}%
  \setfontsize{10}%
}
\gdef\Times{\setTimes\selectfont}
\setTimes
\fontbind{t1xr}
\gdef\setHelvetica{%
  \setfontfamily{txss}%
  \setfontshape{r}%
  \setfontweight{m}%
  \setfontwidth{m}%
  \setfontpage{CT1}%
  \setfontsize{10}%
}
\gdef\Helvetica{\setHelvetica\selectfont}
\setHelvetica
\fontbind{t1xss}
\gdef\setAltHelvetica{%
  \setfontfamily{txssalt}%
  \setfontshape{r}%
  \setfontweight{m}%
  \setfontwidth{m}%
  \setfontpage{CT1}%
  \setfontsize{10}%
}
\gdef\AltHelvetica{\setAltHelvetica\selectfont}
\setAltHelvetica
\fontbind{t1xss scaled 868}
\endgroup


La sélection des fontes avec \brTeX\ se fait au moyen d'un module
sophistiqué que ce document décrit. Utiliser une fonte avec
\brTeX\ demande apparemment un peu plus de travail que pour l'utiliser
avec~\TeX; il faut en effet prendre le temps d'inscrire la fonte
auprès d'une administration. La patience se trouve cependant
récompensée car le système s'avère plus flexible. Le module que l'on
utilise pour la sélection des fontes ressemble beaucoup à la solution
que propose \LaTeX\ pour fournir le même type de service.

Ces informations sont utiles au programmeur écrivant des macros pour
le système \brTeX, mais peu vraisemblablement à l'utilisateur
ordinaire.

\begindescription

\item 1. Définition des termes techniques
  On définit un certain nombre de termes techniques, afin d'écarter
  toute ambigüité dans la discussion suivante. En particulier, on
  qualifie diversement le terme \em{fonte} car on l'emploie pour
  désigner plusieurs choses différentes.

\item 2. Paramètres de description des fontes
  Les fontes sont enregistrées auprès d'une administration, dans un
  registre faisant se correspondre le nom externe d'une fonte ---
  celui du fichier TFM --- et ses caractéristiques visuelles.

\item 3. Couche inférieure du sytème
  On définit les commandes modifiant les registres du système de
  sélection des fontes, réalisant l'inscription d'une fonte externe
  auprès de l'administration, puis les fonctions effectuant des
  recherches dans ces inscriptions.

\item 4. Ajustement des paramètres
  On ajoute des facilités permettant de faire face à la situtation qui
  consiste à devoir répondre à une requête dont les paramètres ne
  décrivent aucune fonte.

\item 5. Couche supérieure du système
  Lorsqu'on change la taille du corps d'une fonte, il peut être
  nécessaire d'ajuster certains paramètres, par
  exemple~\va{\selectlocale{en}{gb}normalbaselineskip}. Les commandes
  de haut niveau destinées à l'utilisateur définissent des \em{hooks}
  que les dans lesquels les bibliothèques de macros peuvent
  enregistrer une procédure appelée à chaque changement de taille.

\enddescription


\section 1. Définition des termes techniques

Une \em{fonte} est un assortiment de caractères. Cet assortiment
contient généralement les lettres de l'alphabet latin et des signes de
ponctuation mais certaines fontes présentent de très riches
assortiments, d'autres encore peuvent n'offrir qu'un jeu de capitales.
On dit aussi parfois \em{police de caractères} pour \em{fonte de
caractères}, car il semblerait qu'on ne fonde plus toujours les
caractères. Nous n'emploierons cependant jamais ce mot et lui
préfèrerons toujours \em{fonte}, à cause de l'expression très
plaisante~\em{fonderie éléctronique}. Nous examinerons bientôt les
différentes qualifications de ce terme que nous employons, mais pour
l'heure, examinons les sortes de caractères que peut présenter une
fonte.

Un alphabet de vingt-six lettres ne permet pas d'écrire les français,
car il faut en effet considérer comme distinctes les lettres munies de
signes diacritiques --- par exemple l'accent grave et la cédille ---
et les lettres naturelles. Avec cette façon de compter, \em{u, ü
\em{et}~ù} sont donc trois lettres distinctes entre elles. Les signes
\em{æ} et \em{œ} doivent aussi être comptés, la ligature entre
l'\em{o} et l'\em{e} n'est pas un simple effet esthétique mais produit
bien un signe différent: on écrit~\em{œuvre} mais~\em{coercitif}, et
les orthographes~\em{oeuvre} et~\em{cœrcitif} sont fausses. Une fonte
destinée à composer un texte, comme un courrier ou un roman, doit donc
posséder un grand nombre de lettres, en voici quelques unes:

\sampleline{%
\fontsample\rm{f}\hfil
\fontsample\rm{¼}\hfil
\fontsample\rm{ÿ}\hfil
\fontsample\rm{g}%
}

À coté des lettres, il faut pour composer un texte un grand nombre de
signes de poncutation, dont voici quelques uns:

\sampleline{%
\fontsample\rm{«}\hfil
\fontsample\rm{—}\hfil
\fontsample\rm{;}\hfil
\fontsample\rm{?}%
}

Dans un texte ordinaire se trouvent enfin des chiffres, et une grande
variété d'autres signes:

\sampleline{%
\fontsample\rm{7}\hfil
\fontsample\rm{*}\hfil
\fontsample\rm{†}\hfil
\fontsample\rm{¶}%
}

Des compositions spécialisées nécessitent l'utilisation de caractères
supplémentaires, notamment les compositions de textes mathématiques
pour lesquelles on utilise un grand nombre de formes de caractères
différentes aussi bien que des symboles spécifiques:

\sampleline{%
\fontsample\rm{$\int$}\hfil
\fontsample\rm{$\star$}\hfil
\fontsample\rm{$\otimes$}\hfil%
\fontsample\rm{$\coprod$}
}

Terminons ces exemples sur les caractères décoratifs:

\sampleline{%
\fontsample\rm{\zdchar'41}\hfil
\fontsample\rm{\zdchar'46}\hfil
\fontsample\rm{\zdchar'55}\hfil
\fontsample\rm{\zdchar'141}%
}

Une fonte de caractères peut consister en un assortiment très complet
de caractères, une autre en un ensemble de belles capitales, une autre
en une famille de symboles décoratifs, etc. Nous avons besoin de
classer les caractères présents dans une fonte en sous assortiments,
que nous appelons~\em{domaines}. Voici la liste de ces domaines:

\begindescription

\item Domaine des lettres
  Les caractères figurant dans ce domaine sont toutes les lettres,
  comme~\em{f}, \em{¼}, \em{ÿ} et~\em{g}, ainsi que les chiffres arabes
  et les signes de ponctuation tels que~\em{«}, \em{—}, \em{;}
  et~\em{?}

\item Domaine compagnon
  Le domaine compagnon est le domaine des symboles accompagant le
  texte, comme~\em\dag\ et~\em\P.

\item Domaine mathématique
  Il comprend tous les caractères spécifiques à la composition des
  mathématiques.

\enddescription
Ceci dit, tournons nous vers les différentes qualifications du
terme~\em{fonte}. Pour le programme~\TeX, une fonte est un assortiment
de caractères étiquetté par les nombres entre~0 et~255; et un
caractère n'est que la donnée des dimensions d'une boîte pouvant le
contenir. Cette donnée est consignée dans un fichier de
type~TFM~\em{(pour \TeX\ font metrics),} que le programme~\TeX\ lit
lorsqu'il rencontre une instruction~\fn{font} l'y obligeant. Nous
appelons \em{fonte métrique} ce type de fonte. Le nombre limité de
caractères présents dans une fonte métrique rend l'utilisation des
fontes pourvues d'un grand nombre de caractères quelque peu
acrobatique puisqu'elle est représentée du point de vue de \TeX\ par
plusieurs fontes métriques. On dit qu'une telle fonte est \em{paginée}
et les fontes métriques qui la représentent dans~\TeX\ en sont
les~\em{pages}. nous utilisons le terme~\em{fonte} sans qualification
pour parler d'un assortiment de caractères.



\section 2. Paramètres de description des fontes

L'utilisation simple des fontes métriques présente des limites
inconfortables. Commençons par discuter sur le choix des formes de
fontes. Pour souligner une partie du texte, il est ordinaire de la
composer avec une fonte d'une forme différente de celle employée pour
le texte où apparaît cette partie. On peut par exemple utiliser des
caractères italiques, dont l'aspect tranche sur celui des caractères
romains. Avec \TeX\ on écrit par exemple
\beginverbatim
à cause de l'expression très plaisante {\it fonderie éléctronique}.
\endverbatim
Mais si notre texte doit être composé en italiques, on utilise les
caractères romains pour la mise en relief, on écrit alors:
\beginverbatim
à cause de l'expression très plaisante {\rm fonderie éléctronique}.
\endverbatim
Puisque~\TeX\ est un système programmable, on peut exiger de lui qu'il
se débrouille pour savoir quelle forme de caractères utiliser pour la
mise en relief. Le programmeur doit donc créer une commande \fn{em}
qui s'utilise comme suit:
\beginverbatim
à cause de l'expression très plaisante \em{fonderie éléctronique}.
\endverbatim
et produit des italiques ou des romains, selon ce qui est qpproprié à
la mise en relief d'une portion de texte. Une manière de s'y prendre
est de définir une variable \va{emphasisfont} qu'utilise~\fn{em} pour
savoir quel forme utiliser pour la mise en relief, et les commandes de
sélection de fontes modifient la valeur de~\va{emphasisfont} de façon
convenable chaque fois qu'elles mettent en vigueur une fonte métrique.
Cette méthode simplette fonctionne mais est difficile à maintenir et
s'adapte mal à la multiplication des situations analogues.

Une bien meilleure manière de résoudre ce problème est de sélectionner
la fonte métrique devant être mise en vigueur à tel instant au moyen
de paramètres décrivant cette fonte plutôt qu'au moyen d'un nom de
fichier. Pour mettre en œuvre cette solution, on établit une table de
correspondance entre les noms externes des fontes métriques et des
paramètres décrivant ces fontes métriques, et on définit une procédure
capable de mettre en vigueur l'utilisation d'une fonte mérique à
partir d'un ensemble de paramètres. Les moyens techniques sont
discutés dans le paragraphe~3, nous discutons ici des paramètres.

Un premier paramètre est la \em{famille}. Une famille de fontes
regroupe plusieurs fontes pouvant être utilisées simultanément dans un
même document. Par exemple, voici quelques mots composés en caractères
\em{Latin Modern,} romains et italiques:
\begindisplay
à cause de l'expression très plaisante \em{fonderie éléctronique}.
\enddisplay
\noindent
Au sein d'une même famille peuvent coexister plusieurs formes de
caractères très différentes, comme le romain et l'italique, mais ces
différentes formes ont été conçues pour s'utiliser ensemble et
s'harmonisent. Voici la lettre~\em{m} majuscule et la ligature~{fl}
dans les caractères romains des familles \em{Latin Modern}, \em{Latin
  Modern Typewriter Type} et~\em{Latin Modern Sans Serif}:

\sampleline{%
\fontsample\rm{M}\hfil
\fontsample\tt{M}\hfil
\fontsample\sf{M}%
}

\sampleline{%
\fontsample\rm{fl}\hfil
\fontsample\tt{fl}\hfil
\fontsample\sf{fl}%
}

Un second paramètre est la forme des caractères. Montrons pour la
famille \em{Latin Modern}, la lettre~{g} et la ligature~{fl} pour
chacune des formes, romaine, oblique, italique, petites capitales et
petites capitales obliques:
\sampleline{%
\fontsample\rm{g\thinspace fl}\hfil
\fontsample\sl{g\thinspace  fl}\hfil
\fontsample\it{g\thinspace  fl}%
}

\sampleline{%
\fontsample\sc{g\thinspace  fl}\hfil
\fontsample{\sl\sc}{g\thinspace  fl}%
}

Un autre paramètre est la graisse des caractères.

Un autre paramètre est l'extension des caractères.

Un autre paramètre est la taille optique des cractères.

On termine sur un paramètre propre aux fontes métriques, la page de
codes. On appelle ainsi la correspondance particulière entre les~256
positions de la fonte et les caractères qu'ils représentent. On peut
accéder avec \TeX\ à une fonte composée d'un très large assortiment de
caractères en découpant la fonte en plusieurs fontes métrique, chacune
paramétrée par une page de codes différente. Pour insérer un caractère
particulier, il faut donc connaître sa page de codes et sa position
dans cette page.

\section 3. Couche inférieure du système

La couche inférieure du système de sélection des fontes établit une
table mettant en correspondance les attributs des fontes et les noms
des fontes métriques auxquels ils correspondent. Il fournit également
des procédures permettant de manipuler cette table.

Pour commencer, on énumère les \em{paramètres de sélection des
fontes}. Tous les paramètres sont contenus dans des macros, sauf l'un
d'entre eux, la taille optique, qui est contenu dans un registre
numérique. Les paramètres sont:

\begindescription\variables

\item fontfamily
  Décrit la famille d'une fonte. Ces noms sont généralement abrégés,
  comme \li{lm} pour \em{Latin Modern}, \li{px} pour \em{Palatino
    Extra}, etc.

\item fontshape
  Décrit la forme d'une fonte. Les valeurs sont les suivantes: \li{r}
  pour les romains (les caractères ordinaires), \li{o} pour les
  caractères romains obliques, \li{i} pour les caractères
  italiques. Les variantes en petites majuscules utilisent la même
  lettre que la forme ordinaire, mais en majuscule: \li{R} pour les
  romains en petites majsucules, etc. Pour certaines fontes, la notion
  de forme n'est pas très adaptée, notamment pour les fontes de
  symboles qui ne proposent pas de variantes oblique ou italique. Dans
  ce cas on peut utiliser comme valeur pour \va{fontshape} la
  lettre~\li{n} pour~\em{naturel}.

\item fontweight
  Décrit la graisse d'une fonte. Les valeurs sont~\li{m} pour la
  graisse moyenne, celle utilisée pour le texte ordinaire, \li{l} pour
  une graisse plus faible que la graisse ordinaire, \li{d} pour des
  caractères demi-gras et~\li{b} pour les caractères gras. On trouve
  parfois d'autres appelations pour la graisse comme~\em{book},
  ou~\em{semibold}.

\item fontwidth
  Décrit l'extension d'une fonte. Les valeurs sont~\li{m} pour
  l'extension moyenne, celle des caractères utilisés pour le texte
  ordinaire, \li{c}~pour les caractères condensés et~\li{x} pour les
  caractères étendus.

\item fontpage
  Décrit la page de codes présentée par une fonte métrique. Les
  appelations des pages de code sont décrites ci-dessous.

\item fontsize
  La taille optique de la fonte. Cette taille optique est exprimée en
  nombre entier de points. Cette restriction ne signifie pas que la
  taille optique de la fonte doit toujours être un nombre entier de
  points. Lorsqu'on compose un texte avec des caractères de dix
  points, les caractères d'autres fontes peuvent avoir des tailles
  optiques voisines de dix points, mais légèrement différente de cette
  valeur. Cf.~Alan~Hoenig, \em{\TeX\ unbound},~p.~151.

\enddescription
\noindent
Pour affecter une valeur à ces paramètres on utilise les
commandes~\fn{setfontfamily}, etc. Le paramètre~\va{fontsize} est un
registre numérique, on obtient donc sa valeur avec~|\the\fontsize|,
les autres sont des macros et on écrit simplement |\fontpage|,~etc.

\formalpar
L'inscription d'une fonte dans la table de l'administration et la
consultation de cette table se fait dans un premier temps en
positionnant les valeurs des registres précédents, puis on appelle la
procédure effectuant l'action souhaitée. On discute de ces
procédures dans la suite de ce paragraphe.

\formalpar Procédure fontbind.
La procédure~\fn{fontbind} inscrit une fonte métrique dans les tables
de l'administration, et charge cette fonte métrique au moyen de la
primitive~\fn{font} de~\TeX. Par exemple, enregistrons auprès de
l'administraton les fontes Adobe Times Roman et Adobe Helvetica en dix
points. Pour Adobe Times Roman, on procède ainsi:
\beginverbatim
\setfontfamily{tx}
\setfontshape{r}
\setfontweight{m}
\setfontwidth{m}
\setfontpage{CT1}
\setfontsize{10}
\fontbind{t1xr}
\endverbatim
Le fichier~\pa{t1xr.tfm} décrit en effet les propriétés métriques des
caractères de la page de code~\li{CT1} de la fonte Adobe Times Roman.
Pour procéder à l'enregistrement de la fonte Adobe Helvetica, on
procède de façon analogue:
\beginverbatim
\setfontfamily{txss}
\setfontshape{r}
\setfontweight{m}
\setfontwidth{m}
\setfontpage{CT1}
\setfontsize{10}
\fontbind{t1xss}
\endverbatim
Il apparaît que ces deux fontes présentent dans leur taille optique de
dix points une différence de taille considérable:
\begindisplay
\Times Adobe Times\quad
\Helvetica Adobe Helvetica
\enddisplay
Pour utiliser ces fontes dans le même document, il faut procéder à un
réajustement des dimensions d'une des deux fontes, ou bien des deux. À
titre d'exemple, choisissons de réduire la taille de Adobe
Helvetica. Après quelques essais, il semble qu'une réduction de la
fonte Adobe Helvetica à~$868/1000$ de sa taille fournit un résultat
satisfaisant. On peut donc choisir d'enregistrer la fonte Adobe
Helvetica comme précedemment avec la déclaration
\beginverbatim
\fontbind{t1xss scaled 868}
\endverbatim
Lorsque les deux fontes apparaissent côte à côte, la différence n'est
plus si dérangeante qu'auparavant:
\begindisplay
\Times Adobe Times\quad
\AltHelvetica Adobe Helvetica
\enddisplay
Il peut arriver que certaines fontes ne puissent pas être utilisées
dans un même document, même après un réajustement de leur taille:
certaines formes ne vont pas ensemble, et on ne peut rien faire contre
cela.

\formalpar Procédure toksloadfontname et apparentées.
La procédure \fn{fontbind} charge la fonte métrique indiquée au moyen
de la primitive~\fn{font}. Le noms sous laquelle cette fonte est
chargée est produit par la procédure \fn{toksloadfontname}. L'appel
\beginverbatim
\toksloadfontname\to\rtA
\endverbatim
charge le registre~\va{rtA} avec le nom d'une macro utilisée pour lier
la fonte. Par exemple, l'appel |\fontbind{t1xss scaled 868}| est une
version améliorée de
\beginverbatim
\toksloadfontname\to\rtA
\expandafter\font\the\rtA=t1xss scaled 868
\endverbatim
Cette procédure lie la fonte \pa{t1xss} à la séquence
\li{%
  \setAltHelvetica
  \toksloadfontname\to\rtA
  \expandafter\string\the\rtA
}, produite grâce la primitive~\fn{csname}.

Pour produire l'identifiant~\li{\setAltHelvetica\fontname}, la
procédure~\fn{toksloadfontname} utilise la macro~\fn{fontname}. La
variante~\fn{fontshortname} produit cet identifiant sans indication de
taille, par exemple~\li{\setAltHelvetica\fontshortname} et la
macro~\fn{toksloadcsfontshortname} charge cet identifiant dans un
registre de~\em{tokens}.

\formalpar Sauvegarder les paramètres de sélection.
La procédure~\fn{toksloadfontstate} sauvegarde l'ensemble des
paramètres de sélection des fontes dans un registre. On restaure ces
paramètres sauvegardés en évaluant les~\em{tokens} contenus dans le
registre.  La variante~\fn{toksloadfontstatealt} fait la même chose
mais omet de sauvegarder la taille.

\formalpar Procédure fontswitch.  Cette procédure consulte la table
administrative pour trouver une fonte correspondant aux paramètres de
description des fontes en vigueur au moment de l'appel. Lorsqu'une
fonte correspondante est trouvée, cette-dernière devient la fonte
courante pour le mode texte (cet appel ne modifie pas la fonte
courante en mode mathématique).  Lorsqu'aucune fonte inscrite dans les
tables ne concorde avec les paramètres, une procédure d'ajustement des
paramètres est utilisée. Cette procédure est définie par l'utilisateur
de \fn{fontswitch}, d'une façon décrite ci-dessous, sous le
titre~\em{Ajustement des paramètres}. Tout se passe comme si aucun
paramètre de sélection des fontes n'est modifié entre le début et la
fin de la procédure, même lorsque les paramètres ont été ajutés
pendant la recherche.

La procédure \fn{fontswitch} a deux pendants, \fn{toksloadfontswitch}
et \fn{fontloadfontswitch}. Au lieu de faire du résultat de la requête
la fonte courante, le premier range une séquence de contrôle activant
cette fonte dans un registre de tokens, l'autre lie cette séquence de
contrôle à une autre séquence de contrôle ou a un registre de fontes
du mode mathématique. Voici un exemple d'utilisation de
\fn{toksloadfontswitch}:
\beginverbatim
\toksloadfontswitch\setsectionfont\to\rtA
\endverbatim
\noindent
Dans ce exemple on suppose que la procédure~\fn{setsectionfont} a pour
rôle de charger des paramètres de sélection des fontes. Une recherche
a lieu et le résultat de cette recherche est une séquence de contrôle
placée dans~\va{rtA}. Par exemple, après une requête, le
registre~\va{rtA} peut contenir~|\txss/r/m/m/CT1/10|~(un~seul
\em{token}).
Voici un exemple d'utilisation de \fn{fontloadfontswitch}:
\beginverbatim
\fontloadfontswitch\setsectionfont\to{\textfont0}
\fontloadfontswitch\setsectionfont\to\shorthand
\endverbatim
Avec le deuxième exemple,~\fn{shorthand} est une séquence de contrôle
ayant le même effet que |\txss/r/m/m/CT1/10|, si les paramètres sont
les mêmes que dans l'exemple précédent.

\formalpar Ajustement des paramètres.
Comme il a été dit plus haut, lorsque les paramètres de description
des fontes ne correspondent à aucune fonte enregistrée
par~\fn{fontbind}, un mécanisme d'ajustement des paramètres est
utilisé. Ce mécanisme n'est pas défini par le schéma de sélection des
fontes mais par l'utilisateur de cette bibliothèque. On trouve un
exemple de mécanisme d'ajustment dans la bibliothèque~\pa{pfssadj}.

Le mécanisme d'ajustement des paramètres est défini par deux
procédures, \fn{fontswitchadjust} et \fn{fontswitchadjustinit}. La
première est le corps d'une boucle et la seconde est un séquence
d'initialisation pour cette boucle. Le mécanisme peut être représenté
par le pseudo-code suivant:
\beginlines
\fn{fontswitchadjustinit}
\sy{while} \sy(\fn{font parameters have no match}\sy) \sy{do}
\quad\fn{fontswitchadjust}
\sy{done}
\endlines
Le mécanisme d'ajustement des paramètres peut prendre plusieurs
décisions, parmi lesquelles:

\beginroster

\item lorsqu'une fonte dont les paramètres d'enregistrement
diffèrent seulement des paramètres actuels par la valeur de
\va{fontsize}, on peut charger cette dernière fonte métrique avec un
facteur d'échelle;

\item lorsque des petites capitales ne sont pas disponibles, on peut
prendre des lettres ordinaires;

\item  lorsque les italiques ne sont pas disponibles mais que les
obliques le sont, on peut utiliser ces derniers.

\endroster
Les décisions particulières dépendent généralement de la famille de
fontes utilisée. Les procédures \fn{testfontfamily}, etc. et
\fn{testfontglob} peuvent être utilisées pour programmer une procédure
\fn{fontswitchadjust}.


\formalpar Prédicats examinant les paramètres.
Plusieurs prédicats permettent d'examiner les paramètres de sélection
des fontes. Des prédicats permettent d'examiner un paramètre à la
fois, il s'agit de \fn{testfontfamily}, \fn{testfontshape},
\fn{testfontweight} et \fn{testfontwidth}. Le prédicat
\fn{testfontglob} permet de tester plusieurs paramètres simultanément.
La valeur \va{fontsize} peut être examinée avec le test
primitif~\fn{ifnum}.  On dit par exemple
\beginverbatim
\testfontwidth m%
\iffontwidth
  \ifx\relax\undefined A\else B\fi
\else
  \ifx\relax\undefined A\else B\fi
\fi
\endverbatim

La commande~\fn{testfontglob} permet de tester simultanément plusieurs
paramètres: la forme, la graisse et l'extension. Par exemple:
\beginverbatim
\testfontglob R*c%
\iffontglob
   texte vrai
\else
   texte faux
\fi
\endverbatim
Cet appel exécute le texte vrai lorsque la forme est~\li{R}, la
graisse quelconque (i.e.~il n'y a pas de condition sur la~graisse) et
l'extension~\li{c}. Les positions où l'argument est une étoile ne sont
pas examinées, et le test~|\testfontglob ***| évalue toujours le texte
vrai.

Remarquons pour terminer que les paramètres de forme, de graisse et
d'extension sont représentés par une seule lettre, ce qui permet
d'utiliser le test primitif~\fn{if} au lieu de~\fn{testfontshape},
de~\fn{testfontweight} et de~\fn{testfontwidth}. Cette pratique est
découragée.


\section 4. Ajustement des paramètres

Ce paragraphe présente la procédure d'ajustement des paramètres
définies par la bibliothèque~\pa{pfssadj}. Rappelons que la procédure
d'ajustement est mise en œuvre par~\fn{fontswitch},
\fn{fontloadfontswitch} et~\fn{toksloadfontswitch} lorsque les
paramètres de description de fonte ne correspondent pas aux paramètres
d'enregistrement d'une fonte avec~\fn{fontbind}. La bibliothèque
définit la séquence d'initialisation~\fn{fontswitchadjustinit} de
cette procédure et le corps de la boucle~\fn{fontswitchadjust} de
cette procédure.


\section 5. Couche supérieure du système

Ce paragraphe décrit les commandes du schéma de sélection des fontes
destinées à être utilisées par le rédacteur d'un document ou le
concepteur d'une classe de document. La commande~\fn{selectfontsize}
modifie la taille des caractères utilisés pour composer le texte. La
commande~\fn{selectfont} modifie la fonte utilisée pour composer le
texte. Des commandes~\fn{rm}, \fn{tt}, \fn{bf}, etc. ressemblant à
celles de plain~\TeX\ permettent d'activer certains ensembles de
paramètres de sélection des fontes.

\formalpar Procédure selectfont.
Un point important à relever est que la commande~\fn{fontswitch} est
une commande de bas niveau, qui opère une instruction élémentaire dans
la sélection des fontes.  Cependant, le changement de fonte pour la
préparation du texte peut appeler d'autres modifications, surtout
lorsqu'une taille différente de la taille utilisée précedemment est
demandée. Dans ce cas, il est nécessaire de changer les paramètres
définissant les espaces interlignes et de rafraîchir les contenus des
registres de fontes du mode mathématique. Dans certaines applications,
il peut être nécessaire nécessaire d'effectuer des ajustements lorsque
la taille du texte que l'on compose est changée. Pour cela, la
commande~\fn{selectfont} définit un~\em{hook},
nommé~\va{fontsizehook}, auquel on peut ajouter des instructions
devant être traitées chaque fois qu'un nouveau jeu de caractères est
activé. Les concepteurs de classe de document doivent utiliser
\fn{selectfont} pour changer de caractères pour la préparation du
texte, et jamais la commande~\fn{fontswitch}.

Il est commode de définir des commandes pour charger une fonte
particulière, par exemple la fonte utilisée pour préparer les titres
peut être activée par~\fn{titlefont}, la fonte pour composer les haut
de page par~\fn{runningheaderfont}, etc. Il est commode d'écrire ces
macros avec une commande auxiliaire dont le rôle est de modifier les
valeurs des registres; dans nos exemples on
définirait~\fn{settitlefont} et~\fn{setrunningheaderfont}, puis
\beginverbatim
\def\titlefont{\settitlefont\selectfont}
\def\runningheaderfont{\setrunningheaderfont\selectfont}
\endverbatim
Cette façon de procéder permet de réaliser l'héritage des
attributs. Si par exemple on souhaite utiliser pour les titres de
paragraphes la même fonte que pour le titre du document, on peut
définir~\fn{setsectiontitlefont} de la façon suivante:
\beginverbatim
\def\setsectiontitlefont{\settitlefont\setfontsize{12}}
\def\sectiontitlefont{\setsectiontitlefont\selectfont}
\endverbatim
Les modifications ultérieures apportées au style des titres réalisé
par~\fn{settitlefont} se répercute alors automatiquement sur les
titres des sections. On aurait aussi pu~écrire
\beginverbatim
\def\sectiontitlefont{\titlefont\setfontsize{12}\selectfont}
\endverbatim
Il n'est cependant pas très satisfaisant de savoir que dans l'appel
à~\fn{sectiontitlefont} ainsi définie, la procédure~\fn{selectfont}
est évaluée deux fois. Le choix du préfixe~\fn{set} pour nommer ces
procédures est en accord avec les recommandations générales sur le
nommage des commandes.

\formalpar Procédure selectfontsize.
Le format plain~\TeX\ définit des commande~\fn{tenpoint}, etc. Le
format~\brTeX\ définit les commandes~\fn{twelvepoint}, \fn{elfpoint},
et ainsi de suite jusqu'à~\fn{fivepoint} pour changer de taille de
texte. On définit aussi des commandes de changement de taille dont les
identifiant ne font pas référence à la mesure de la taille de
caractères utilisée mais à une appréciation de sa grosseur relativement
à la taille des caractères pour le texte normal. Ces commandes
sont~\fn{normalsize}, \fn{smallsize}, \fn{verysmallsize}, \fn{bigsize}
et~\fn{verybigsize}. À chacune de ces commandes correspond une
commande modifiant la valeur du paramètre~\va{fontsize} sans effectuer
l'appel~\fn{selectfontsize}, ces commandes sont
nommées~\fn{setnormalsize}, etc.; dans~\brTeX\ elles correspondent aux
tailles~10, 9, 7, 12 et~17, mais elles peuvent être redéfinies par une
classe de document ou une bibliothèque de macros. À cet ensemble de
commandes s'ajoutent les procédures~\fn{setlargersize},
\fn{largersize}, \fn{setsmallersize} et~\fn{smallersize}, qui font ce
qu'indique leur nom.

Une classe de document devrait définir des noms symboliques pour les
tailles de différents éléments apparaissant dans le document, par
exemple~\fn{settitlesize} ou~\fn{setdocumenttitlesize} pour une
commande chargeant la taille utilisée pour préparer le titre du
document,~\fn{setfootnotesize} pour charger la taille des caractères
utilisée pour préparer les notes de bas page, etc.


\formalpar Sélection de certaines fontes.
Comme dans plain~\TeX, on définit un certain nombre de commandes
permettant d'activer directement une fonte particulière. Ces commandes
sont les suivantes:
\begindescription\functions
\item rm
  bascule vers la fonte définie par~\fn{setnormalfont} et sélectionne
  la famille~\va{rmfontfamily} (caractères romains), on attend que
  cette fonte soit celle utilisée pour le texte ordinaire du document;

\item tt
  bascule vers la fonte définie par~\fn{setnormalfont} et sélectionne
  la famille~\va{ttfontfamily} \em{(typewriter type)\/};

\item sf
  bascule vers la fonte définie par~\fn{setnormalfont} et sélectionne
  la famille~\va{sffontfamily} (caractères sans serifs).

\enddescription
Chacune de ces commandes applique un~\em{hook}, la première
applique~\va{rmhook} et ainsi de suite. On peut ajouter à
ces~\em{hook} des commandes modifiant les paramètres du mode
mathématique pour utiliser des~\em{typewriter type} dans le mode
mathématique,~etc. La documentation sur les conventions de saisie dans
le mode mathématique explique en détails quels commandes sont ajoutées
à ces~\em{hooks}. Les commandes suivantes servent à modifier la forme
ou la graisse des caractères employés dans le texte:
\begindescription%\functions

\item it
  pour les italiques;

\item sl
  pour les obliques;

\item bf
  pour les caractères gras;

\item sc
  pour les petites capitales.

\enddescription
Ces différentes procédures peuvent être combinées, par exemple pour
obtenir des caractères gras italiques si ceux-cis sont
disponibles. Chacune de ces commandes applique un~\em{hook}, comme
précedemment.

Remarque: 1/La procédure~\va{setnormalfont} et les
variables~\fn{rmfontfamily}, \fn{ttfontfamily} et~\fn{sffontfamily} ne
sont pas définies par le module de sélection des fontes. Elles doivent
être définies par le module chargeant les fontes du format. 2/Le
module \pa{basedef} définit des commandes de marquage du texte, comme
par exemple~\fn{em} et~\fn{sy} pour le soulignement, \fn{pa} pour un
chemin ou une~URL, etc.

\bye

%%% End of file `pfssintro.tex'
