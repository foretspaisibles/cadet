%%% fontes.tex -- Utiliser FSS pour BHRID TeX

% Author: Michael Grünewald
% Date: Thu Sep 14 21:40:38 CEST 2006

% Copyright (C) 2006, 2013 Michael Grünewald
% All rights reserved.
%
% This file is part of Bhrìd TeX.
%
% Bhrìd TeX is free software: you can redistribute it and/or modify
% it under the terms of the GNU General Public License as published by
% the Free Software Foundation, either version 3 of the License, or
% (at your option) any later version.
%
% Bhrìd TeX is distributed in the hope that it will be useful,
% but WITHOUT ANY WARRANTY; without even the implied warranty of
% MERCHANTABILITY or FITNESS FOR A PARTICULAR PURPOSE.  See the
% GNU General Public License for more details.
%
% You should have received a copy of the GNU General Public License
% along with Bhrìd TeX.  If not, see <http://www.gnu.org/licenses/>.


\class bhridpaper
\enablehverbatim

\title Les fontes de caractères et BHRID \TeX
\author Michael Grünewald

Nous présentons l'utilisation des fontes de caractères avec Sugar
Cane. Ces explications sont destinées aux préparateurs de classes de
documents, et sont probablement inutiles à l'utilisateur ordinaire.

Pour utiliser une fonte avec BHRID \TeX, il faut lier son nom externe,
le radical du fichier TFM utilisé par \TeX, avec une description
symbolique de cette fonte. Cette opération est la déclaration de la
fonte, il s'agit d'une inscription administrative. Une fois cette
déclaration accomplie, on peut utiliser la fonte dans un document en
faisant de cette fonte la fonte courante. Cette opération est la
sélection de la fonte.


\section Déclaration et sélection des fontes

La commande |\fontmatch| est utilisée pour réaliser l'inscription
administrative d'une fonte. Pour cette déclaration, il est nécessaire
de connaître le nom du ou des fichier TFM décrivant cette fonte pour
\TeX, et les attributs de cette fonte. Les attributs sont
\beginlist
\item la disposition des caractères dans la fonte
      \em{(font~encoding)};
\item un nom court pour la famille des caractères \em{(font~family)};
\item un nom court pour un paramètre combinant la graisse et l'étendue
      des caractères \em{(font~series)};
\item un nom court pour la forme des caractères.
\endlist
Nous discutons ces élements ci-dessous, puis donnons plusieurs
exemples d'utilisation de la commande |\fontmatch|.


\formalpar Disposition des caractères

Les fichiers décrivant les fontes électroniques utilisés par \TeX\ ont
256 positions pouvant chacune contenir la description d'un
caractère. Dans des fontes éléctroniques différentes, les dispositions
de caractères sont parfois différentes. Dans la fonte~\em{Computer
Modern Roman}~\li{cmr10}, seules les~$128$ premières positions sont
utilisées, elles recèlent des caractères utilisés ordinairement dans
un texte anglais, et des accents sans lettres pour composer
d'éventuels caractères manquants. Dans la fonte~\em{Latin Modern
Roman}~\li{cork-lmr10}, les~$256$ positions sont utilisées par des
caractères apparaissant dans les textes composés dans de nombreuses
langues d'\nm{Europe}, utilisant un alphabet suffisamment proche de
l'alphabet latin. Les caractères situés dans les~$128$ premières
positions ne sont pas les mêmes pour~\li{cmr10} et
pour~\li{cork-lmr10}, de plus ces fontes ne décrivent pas le même
ensemble de caractères; pour ces deux raisons ces deux fontes ont des
dispositions de caractèrs différentes. L'exemple des fontes recelant
uniquement des symboles spéciaux comme les~\em{Zapf Dingbats} dessinés
par~\nm{H.~Zapf} ou les fontes destinées à la préparation de formules
mathématique présente des fontes dont la disposition de caractères
s'éloigne singulièrement de celle des fontes destinées au texte.

Pour produire le résultat escompté, certaines commandes doivent
sélectionner une fonte à la disposition des caractères appropriée, par
exemple les commandes produisant un signe de section, etc. Les valeurs
utilisées par BHRID \TeX\ pour la disposition des caractères sont les
suivantes.
\beginlist\tag
\item{T1}	text 1
\item{S1}	symbol 1
\item{ML}	math letter
\item{MS}	math symbol
\item{MX}	math extension
\smallskip
\item{OT1}	original \TeX\ text 1 encoding
\item{OML}	original \TeX\ math letter
\item{OMS}	original \TeX\ math symbol
\item{OMX}	original \TeX\ math extension
\smallskip
\item{U}	unknown
\item{Xxx}	expert encoding
\item{XZD}	Zapf's dingbats
\item{Lxx}	local encoding
\endlist
Nous utilisons des identificateurs formés de un à trois caractères,
tous les identificateurs sont réservés par BHRID \TeX, hormis les
identificateurs à trois lettres dont la première est un~\li{L} qui
sont réservés pour l'utilisateur.


\formalpar Nom court pour la famille de caractères

Quelques exemples
\beginlist\tag
\item{lmr}	Latin Modern Roman
\item{lmss}	Latin Modern Sans Serifs
\item{lmtt}	Latin Modern Typewriter Type
\item{cmr}	Computer Modern Roman
\item{cmmi}	Computer Modern Math Italics
\item{zd}	Zapf's Dingbats
\endlist
Plusieurs dessins de caractères coexistent au sein d'une même famille
de caractères. Par exemple, les familles de caractères destinées à la
préparation du texte comportent généralement des caractères romains,
des caractères italiques, des caractères gras, des caractères gras
italiques, et parfois de petites capitales; bien que les dessins des
romain et des italiques soient très différents, ils sont visuellement
compatibles et sont à ce titre regroupés dans une famille.

\formalpar Graisse et étendue des caractères

Les deux données de l'étendue et de la graisse des caractères sont
rassemblées dans un seul paramètre.
\beginlist\tag
\item{m}	medium, parfois appelé \em{book} ou livre, la graisse
		des caractères~usuels
\item{b}	bold
\item{s}	semi bold
\smallskip
\item{c}	condensed, caractères resserés, par exemple pour les~titres
\item{x}	extended, caractères écartés
\smallskip
\item{bx}	bold extended
\endlist
Lorsque, comme dans le dernier exemple, une valeur du paramètre
mentionne à la fois la graisse et l'extension des caractères, la
description de la graisse figure toujours en premier.

\formalpar Forme des caractères

Nous utilisons les noms courts suivants pour les formes de caractères
\beginlist\tag
\item{n}	\em{naturel}, caractères ordinaires
\item{i}	\em{italique}, caractères italiques
\item{o}	\em{oblique}, caractères obliques,
		à ne pas confondre avec les italiques
\item{N}	\em{Naturel}, petites capitales naturelles
\item{I}	\em{Italique}, petites capitales italiques
\item{O}	\em{Oblique}, petites capitales obliques
\smallskip
\item{u}	\em{unslanted italics}, italiques droites
\endlist
Les italiques droites ne sont pas normalement utilisées, mais elles
sont présentes avec \TeX\ afin d'insister sur la différence de dessin
entre les caractères italiques et les caractères romains. La forme de
caractères~\li{u} n'est pas utilisée dans BHRID~\TeX, mais elle cet
identifiant lui est réservée.


\formalpar Taille nominale d'une fonte

Les dessins des caractères d'une fonte ont une taille nominale, par
exemple le fichier \li{cork-lmr10} décrit pour \TeX\ la fonte Latin
Modern Roman à la taille nominale~$10$pt. En principe la taille
nominale de la fonte correspond à la hauteur de la majuscule~\li{X}
et à la taille du tiret long, obtenue en \TeX\ par~\li{-{}-{}-}, mais
il arrive souvent que la taille nominale de la fonte soit différente
de la taille de ces deux derniers caractères; en outre des fontes de
caractères d'aspect visuel très voisins peuvent avoir des tailles
nominales relativement très différentes, et ce, qu'elles proviennent
ou non de fondeurs différents. La taille nominale des fontes de
caractères indique quels caractères ont des dimensions visuellement
compatibles; ainsi, si l'on souhaite utiliser conjointement des
romaines et des italiques dans un même paragraphe de texte, on peut
par exemple choisir les romaines Latin Modern Roman $10$pt et les
italiques Latin Modern Roman Italic $10$pt, ces deux fontes de
caractères affichent la même taille nominale de caractères,
proviennent du même fondeur et font partie de la même famille Latin
Modern, elles ont donc des dimensions visuellement compatibles. Il est
important de souligner que des fontes de caractères de taille nominale
identiques mais appartenant à des familles différentes peuvent ne pas
avoir des dimensions visuellement compatibles, même si elles
proviennent de la même fonderie.


\formalpar Sélection d'une fonte

La commande~|\fontwish T1/lmr/m/n/10pt \as\tenrm| demande à BHRID \TeX
de trouver une fonte resemblant le plus possible à la fonte dont la
description est donnée, et fait de |\tenrm| une séquence de contrôle
déclenchant l'utilisation de la fonte obtenue. Si, par exemple, la
fonte obtenue dans les données administratives est décrite par le
fichier \li{cork-lrm10}, un résultat équivalent peut être obtenu par
|\font\tenrm=cork-lmr10|.

On peut demander une fonte sans préciser explicitement tous les
paramètres la décrivant. Pour cela, BHRID \TeX\ prévoit cinq commandes
\begincode
\fontpage
\fontfamily
\fontseries
\fontshape
\fontsize
\endcode
qui chargent une valeur dans une mémoire appropriée, on écrit par
exemple |\fontpage{S1}| ou |\fontseries{b}|. Ces commandes sont les
\em{modificateurs des paramètres de sélection}, les mémoires qu'ils
altèrent sont les \em{paramètres de sélection}. On demande alors une
fonte particulière au moyen de la forme spéciale
\begincode
\fontwish\selection\as\tenrm
\endcode
la séquence |\selection| n'est pas réservée par BHRID \TeX, elle peut
être utilisée dans des jeux de macros, mais elle revêt ici une
signification spéciale. Au début d'un \em{job} traité avec BHRID \TeX,
tout se passe comme si les commandes
\begincode
\fontpage{T1}
\fontfamily{lmr}
\fontseries{m}
\fontshape{n}
\fontsize{10pt}
\endcode
avaient été évaluées; dans ce contexte, l'appel
\begincode
\fontwish\selection\as\tenrm
\endcode
équivaut au suivant
\begincode
\fontwish T1/lmr/m/n/10pt \as\tenrm
\endcode
Certaines macros peuvent modifier les fontes utilisées pour préparer
le texte avec BHRID \TeX\ et la famille de caractères romains en
vigueur peut porter un nom court différent de~\li{lmr}. Une macro
arrangeant les choses pour que des fontes différentes soient utilisées
doit redéfinir les séquences
\begincode
\fontrmfamily
\fontsffamily
\fontttfamily
\endcode
pour qu'elles représentent les noms courts appropriés.

\formalpar Paramètres de sélection des fontes et groupes

Les paramètres de sélection des fontes sont restaurés à la fin du
groupe en cours. Il peut néanmoins être utile de sauvegarder
l'ensemble de ces paramètres, BHRID \TeX\ prévoit donc une commande de
sauvegarde et une commande de restauration, on dit par exemple
\begincode
\fontselectionsave\myselection
...
\fontselectionload\myselection
\endcode
La mémoire utilisée dans l'exemple est relative au niveau de
regroupement courant, les mémoires sont détruites ou restaurées à la
fin du niveau de groupement courant.


\formalpar Noms pour les commandes modifiant les paramètres %
 de sélection des fontes

BHRID \TeX\ réserve la séquence |\fontcurrent|, elle est utilisée par
la commande |\fontselect|, dont la définition est
\begincode
\def\fontselect{\fontwish\selection\as\fontcurrent\fontcurrent}
\endcode
Les noms de séquences |\...fontselect| doivent toujours être utilisés
pour des commandes qui modifient et rendent active |\fontcurrent|,
comme par exemple |\tlfontselect| qui sélectionne la fonte pour le
texte ordinaire, là où la commande |\tlfont| ne fait que modifier les
paramètres de chargement.



\formalpar Inscription administrative des fontes

Nous discutons plusieurs exemples d'utilisation de la
commande~|\fontmatch|. La déclaration
\begincode
\fontmatch T1/lmr/m/n \with
|  5pt cork-lmr5
|  6pt cork-lmr6
|  7pt cork-lmr7
|  8pt cork-lmr8
|  9pt cork-lmr9
| 10pt cork-lmr10
| 12pt cork-lmr12
| 17pt cork-lmr17
\stop
\endcode
enregistre administrativement~$8$ fichiers de fontes \TeX\ dans le
système BHRID \TeX, la commande |\fontmatch| est ainsi faite qu'elle
permet de déclarer plusieurs fichiers en même temps lorsque dans leurs
descriptions, seule varie la taille nominale.

Il est possible d'indiquer un facteur d'aggrandissement, en millièmes,
de la façon suivante:
\begincode
\fontmatch T1/pcr/m/n \with
| 10pt 988 pcrr8t
\stop
\endcode
Dans cet exemple, la taille demandée à \TeX\ est la taille nominale
multipliée par~$988/1000$, soit $10 \cdot 988/1000 = 9,88$ en points
d'imprimeur. Avec cette déclaration, la commande
\begincode
\fontwish T1/pcr/n/n/10pt\as\courierten
\endcode
équivaut à
\begincode
\font\courierten=pcrr8t at 9.88pt
\endcode
Lorsqu'on sélectionne une fonte dans une taille nominale qui n'a pas
été déclarée, une fonte dans autre taille est sélectionnée
automatiquement, et sa taille est ajustée pour obtenir la taille
nominale souhaitée.

Lorsque la commande |\fontmatch T1/lmr/m/n \with...| est exécutée, une
séquence de contrôle |\fontmatch T1/lmr/m/n| est créee (l'espace est
dans la séquence, comme on peut l'obtenir grâce à |\csname|); cette
séquence est une lsite dont chaque élément est un triplet refletant
chaque ligne de la déclaration |\fontmatch|. Voici une correspondance
ligne/triplet.
\begindisplay
\halign{#\hfil\quad&#\hfil\cr
\li{10pt 988 cork-lmr10}&\li{(10pt,988,cork-lmr10)}\cr
\li{10pt cork-lmr10}&\li{(10pt,0,cork-lmr10)}\cr
\li{* 988 cork-lmr10}&\li{(0pt,988,cork-lmr10)}\cr
\li{* cork-lmr10}&\li{(0pt,0,cork-lmr10)}\cr
}
\enddisplay
Les triplets sont de la
forme~\li{(taille\_nominale,ajustement,tfm)}. Le champ
\li{taille\_nominale} a \li{0pt} comme valeur à sémantique spéciale,
elle signifie que ce fichier \li{tfm} doit être utilisé avec un
ajustement pour toutes les tailles nominales n'apparaissant pas
explicitement dans la liste; le champ \li{ajustement} a \li{0} comme
valeur à sémantique spéciale, elle révèle simplement une absence
de demande d'ajustement.

\formalpar Procédure de sélection d'une fonte
Il se peut que l'utilisateur réclame à l'administration une fonte dont
la description ne correspond à aucune des fontes enregistrées auprès
d'elle. Dans ce cas les paramètres de la requète sont \qu{dégradés}
peu à peu jusqu'à obtenir des valeurs désignant une fonte connue de
l'adminsitration. Le mécanisme de dégradation est mis en ½uvre par la
fonction |\fontwish@Y|, celle-ci à pour rôle de rappeler~|\fontwish@A|
avec un nouveau jeu de paramètres. Les appels successifs à
|\fontwish@Y| font converger l'argument de |\fontwish@A| vers
|\fontwish@A{T1}{lmtt}{m}{n}| sans modifier la taille nominale.

La première dégradation appliquée consiste à remplacer la forme \li{I}
par la forme \li{O}, ainsi lorsque des petites capitales obliques sont
déclarées, elles peuvent aussi servir de petites capitales pour les
italiques. La seconde dégradation appliquée consiste à remplacer la
graisse \li{bx} en~\li{b}, la troisième transforme \li{bc} en \li{b}.


\formalpar Liaison de fontes

Lorsqu'une fonte a fait l'objet d'une inscription administrative, on
peut utiliser cette information avec un nouveau nom, par exemple si on
décide que les fontes dont la description est \li{T1/lmr/bx/n} doit
utiliser les mêmes dessins que les fontes dont la description est
\li{T1/lmr/b/n}, vous pouvez obtenir ce résultat en déclarant
\begincode
\fontbind{T1/lmr/bx/n}\with{T1/lmr/b/n}
\endcode
Remarquez que la liaison dans cet exemple est en fait réalisée
implicitement par le système de dégradations des paramètres pour
|\fontwish|. L'exemple illustre cependant l'utilisation pour laquelle
la commande~|\fontbind| a été conçue.



\section Fontes de caractères et mode horizontal

Le mode horizontal de \TeX\ est celui dans lequel ce dernier prépare
les lignes d'un paragraphe. On discute ici des commandes spécifiques à
l'utilisation des fontes dans pour le texte du paragraphe.


\formalpar Afficher un caractère arbitraire d'une fonte

Si la séquence |\tenrm| sélectionne une certaine fonte, on peut
afficher n'importe lequel des~$256$ caractères de celle-ci au moyen de
la commande~|\fontchar| qui prolonge la commande |\char| de~\TeX. On
dit par exemple
\begincode
\fontchar\tenrm"41
\fontchar\tenrm'101
\fontchar\tenrm 65
\endcode
et ces commandes sont équivalentes à~|{\tenrm\char"41}|. On peut
utiliser~|\fontchar| avec |\tenrm| après une déclaration ressemblant
aux suivantes:
\begincode
\fontwish T1/lmr/m/n/10pt \as\tenrm
\def\tenrm{...\fontselect}
\endcode
mais les autres usages sont à éviter.


\formalpar Nommer un caractère arbitraire

La séquence~|\fontchardef| est à |\fontchar| ce que |\chardef| est à
|\char|; elle permet de nommer un certain caractère d'une certaine
fonte. On dit par exemple
\begincode
\def\syfontselect{\fontpage{S1}\fontselect}
\fontchardef\syfontselect\pilcrow="B6
\endcode
et |\pilcrow| est une séquence qui bascule temporairement vers la
fonte appropriée et affiche le caractère situé à la position~|"B6|.


\formalpar Sélectionner la taille du texte

Nous avons vu comment le paramètre de sélection des fontes |\fontsize|
régissait la taille de la fonte employée, mais composer un texte avec
une fonte dans une certaine taille nécessite que des dispositions
autres que le simple changement de la taille du corps des
caractères soient prises. Par exemple, \TeX\ tient compte de plusieurs
paramètres pour ajuster la distance entre les lignes du paragraphe, et
il est bien sûr nécessaire des les ajuster lorsqu'on souhaite composer
un paragraphe dans un corps d'une taille particulière.

Le texte du paragraphe est composé dans un corps dont la taille est
l'une des suivantes: 12pt, 11pt, 10pt, 9pt, 8pt, 7pt, 6pt, 5pt.

Les macros |\twelvepoint|, etc. sélectionnent la taille correspondante
pour la composition du corps de paragraphe. Ces macros ajustent la
valeur du registre |\textsize| (nombre), puis délèguent leur
travail aux macros |\textsizesetup| pour modifier les paramètres du
paragraphe, et |\mathsizesetup| pour modifier les paramètres du mode
mathématique. Les ensembles de macros peuvent enregistrer des actions
à effectuer lorsqu'une commande |\twelvepoint| etc. est utilisée, via
|\textsizehook| qui est appelé par toutes les commandes modifiant la
taille du texte. Au moment de l'appel la valeur de |\textsize|
vaut 12 dans |\twelvepoint|, etc. On peut aussi enregistrer des
commandes dans |\twelvepointhook|, etc. ces commandes sont déclenchées
après le |\textsizehook|.

Dans un document donné, il vaut mieux n'utiliser qu'un petit nombre
de tailles différentes. Pour favoriser une politique cohérente dans
l'utilisation des tailles de caractères, on peut utiliser des alias
comme
\begincode
\let\mainsize=\tenpoint
\let\normalsize=\tenpoint
\let\footnotesize=\sevenpoint
\endcode
N'oubliez pas que |\textsize|, |\fontsize| et |\mainsize| sont
réservés. La définitions de |\mainsize| est celle de l'exemple, ainsi
l'auteur qui souhaite revenir à la taille rodinaire des caractères
dans son document peut utiliser |\mainsize|. Les auteurs de classe de
document doivent penser à ajuster |\mainsize| pour que cette séquence
ait toujours le bon comportement.

Comme expliqué plus haut, la définition de la macro |\tenpoint| est
équivalente à
\begincode
\def\tenpoint{%
  \textsize=10
  \expandafter\fontsize\expandafter{\the\textsize pt}%
  \runhook\textsizeadjustments
  \runhook\mathsizeadjustments
  \rm\runhook\tenpointhook
}
\endcode
En réalité, une partie de ce code est factorisée pour les commandes
|\tenpoint|, |\ninepoint|, etc.


\formalpar Paramètres du paragraphe liés à la taille du texte

La préparation du paragraphe utilise notamment trois valeurs,
|\normalbaselineskip|, |\normallineskip|, et |\lineskiplimit|, de la
manière expliquée dans le {\sl\TeX book}, p.~78. Nous avons une table
|\lineskiptable|, définie comme suit
\begincode
\def\lineskiptable{%
  \\{{12}{{13pt}{1pt}{0pt}}}%
  \\{{11}{{13pt}{1pt}{0pt}}}%
  \\{{10}{{12pt}{1pt}{0pt}}}%
  \\{{9}{{11pt}{1pt}{0pt}}}%
  \\{{8}{{9pt}{1pt}{0pt}}}%
  \\{{7}{{9pt}{1pt}{0pt}}}%
  \\{{6}{{8pt}{1pt}{0pt}}}%
  \\{{5}{{7pt}{1pt}{0pt}}}%
}
\endcode
La macro |\lineskipadjust| inspecte cette table et si elle
trouve une entrée dont le premier terme vaut |\textsize|, elle
charge le triplet de valeurs en regard
dans~|(\normalbaselineskip, \normallineskip, \lineskiplimit|), ce qui
influe sur la préparation du paragraphe comme indiqué dans le {\sl\TeX
book}. Il est facile d'augmenter cette table avec~|\leftappenditem|,
comme dans
\begincode
\leftapppenditem{17}{{20pt}{2pt}{0pt}}\to\lineskiptable
\endcode
La macro |\rightappenditem| aurait ausi bien pu être utilisée, les
macros inspectant cette table ne doivent pas faire d'hypotèse sur un
ordre éventuel des entrées.

Bien entendu, on enregistre |\lineskipadjust| dans |\textsizeadjust|
avec
\begincode
\addhook\lineskipadjust\to\textsizeadjust
\endcode
Une classe de document peut redéfinir complètement les entrées de
|\lineskiptable| ou la macro |\lineskipadjust|, par exemple pour avoir
un très gros interligne dans certaines circonstances.


\formalpar Sélection rapide des fontes

Quelques abréviations commodes sont disponibles pour permettre aux
auteurs de changer rapidement de fonte. Quelques unes de ces
abréviations sont utilisées dans PLAIN~\TeX, il s'agit de~|\rm|,
|\it|, |\sl|, |\sc|, d'autres sont nouvelles, comme |\mainfont| et
|\mainfontselect| que nous décrivons à présent.

La commande~|\mainfont| positionne tous les paramètres de sélection
des fontes, hormis la taille pour que la commande~|\fontselect|
bascule vers la fonte principale du document, tandis que
|\mainfontselect| signifie simplement |\mainfont\fontselect|. Il est
conseillé aux auteurs de macro d'utiliser des noms qui ressemblent aux
précédents dans les nouvelles défintions qu'ils proposent, par exemple
|\footnotefont| et |\footnotefontselect|, etc.

Si vous shouhaitez revenir à la fonte principale et à la taille
principale du document la séqunece recommandée est
|\mainfont\mainsize|, remarquez que |\mainfont| ne sélectionne pas
effectivement une fonte, mais |\mainsize|, qui est un alias pour
|\tenpoint| au une autre de ces commandes, le fait.

Les commandes |\rm|, |\it|, |\sl| et |\sc| ne modifient que le
paramètre |\fontshape| et sélectionnent la fonte ainsi paramétrée. La
commande |\bf| sélectionne une graisse différente en ne modifiant que
le paramètre |\fontseries| et sélectionne la fonte ainsi
paramétrée. Les commandes |\rm|, etc. ainsi que la commande |\bf|
peuvent modifier certains paramètres du mode mathématique (voir plus
bas).

Remarquons qu'il n'y a pas de séquence courte pour passer d'une
graisse élevée à une graisse moyenne. Ces commandes ne sont pas utiles
aux auteurs, qui délimitent la portée des caractères fortement
graissés au moyen d'un groupe.

Enfin, les commandes |\tt| et |\sf| modifient le paramètre
|\fontfamily| pour obtenir des Typewriter Type ou Sans Serifs; la
commande |\sf| modifie également les paramètres du mode mathématique.


\formalpar Mise en relief d'un élément du texte

La mise en relief d'un élément du texte est réalisée par un changement
temporaire de fonte; par exemple, pour mettre en valeur une portion de
texte au milieu de caractères romains, on utilise souvent des
italiques, mais à l'inverse, pour mettre en relief un portion de texte
au milieu de caractères italiques ou penchés, on utilise des
caractères romains. La réalisation d'une mise en relief dépend donc de
la fonte utilisée lors de la composition du texte dont un élément
donné doit se détacher.

\em{Remarque: les écoliers n'utilisent pas des formes de caractères,
de plomb ou éléctroniques, et ils utilisent un tout autre procédé pour
mettre en valeur un élément de texte: ils le soulignent ou
l'encadrent, pafois avec une couleur vive. Cette manière est adaptée
pour les textes mansucrits, mais pas pour les textes typographiés.}

La commande |\emphasis| peut être utilisée pour basculer vers une
fonte appropriée à la mise en relief d'un élément de texte. L'auteur
délimite cette mise en valeur au moyen d'un groupe, ou utilise la
forme courte de la commande
\begincode
\def\em#1{{\emphasis #1}}
\endcode
pour marquer une courte locution. La version longue peut-être
appliquée à plusieurs paragraphes.

La commande |\emphasis| utilise pour son fonctionnement une table de
correspondance entre la forme actuelle de caractères et la forme à
utiliser pour la mise en relief. Le seul paramètre modifié est donc
|\fontshape|. Les auteurs de classes de document souhaitant proposer
un autre comportement doivent réecrire une macro |\emphasis|. Nous
utilisons la table suivante
\begincode
\def\emphasistable{%
  \\{{n}{i}}%
  \\{{i}{n}}%
  \\{{o}{n}}%
  \\{{N}{I}}%
  \\{{I}{N}}%
  \\{{O}{N}}%
}
\endcode
Cette table se lit de la façon suivante: la forme de caractère |n| est
mise en relief au moyen de |i| (italiques), etc. On peut augmenter
cette table au moyen des commandes |\leftappenditem| et
|\rightappenditem|, les entrées les plus à gauche dominent celles de
droite.

On peut utiliser la commande |\emphasisfont| pour ajuster les
paramètres de sélection des fontes en sorte que l'appel |\fontselect|
sélectionne la fonte qu'utiliserait à ce moment la commande~|\em|.


\section Domaines de caractères

La notion de domaine de caractères est liée à celle de disposition des
caractères dans une fonte. Un domaine de caractère est défini par un
ensemble de commandes, qui ressemblent à
\begincode
%
% Symboles dans le Texte
%  du mode paragraphe
%
\def\tsfont{}% Doit être défini par une macro \selectts{whatever}
\def\tsfontselect{\tsfont\fontselect}
\def\tschar{\fontchar\tsfontselect}
\def\tschardef#1{\fontchardef#1\tsfontselect}
\def\selecttspage#1{\csname tspage@#1\endcsname}
\def\selectts#1{\csname selectts@#1\endcsname}
\newhook\tssizeadjustments
\addhook\tssizeadjustments\to\textsizeadjustments
\endcode
Ces commandes décrivent le domaine de caractères |ts| \em{(text
symbols)}. Un domaine de caractères est utilisé pour garantir l'accès
à certains caractères au travers de séquences de contrôle. Par
exemple, si une séquence de contrôle a besoin de produire un
caractère~§, elle a besoin de deviner quel cqrqctère de quelle fonte
doit être utilisé; avec les domaines de caractères, un macro
appropriée permet d'accéder symboliquement à ce caractère.

Voici comment fonctionnent les choses. L'utilisateur
appelle~|\selectts|, comme par exemple~|\selectts{Latin Modern}|,
cette commande définit~|\tsfont| et appelle une ou plusieurs
commandes~|\selecttspage{whatever}|. Une commande~|\selecttspage{S1}|
rend accessible tous les symboles situés sur la page de caractères
\li{S1} au moyen de noms symboliques. L'ensemble des commandes
définies par un appel |\selectts{whatever}| est constant, la commande
|\missingchar| ou des assemblages plus ou moins heureux peuvent être
utilisés lorsqu'un caractère demandé n'est pas disponible dans
certaines fontes.


\formalpar Domaine des lettres pour le texte Latin 9

Voici la liste des commandes en regard du glyphe attendu. Il s'agit
des caractères Latin~9 (aussi ISO~8859~15) dits spéciaux, c'est à dire
les caractères accentués et les ligatures. Le domaine défini
est~\li{tl}, où le~\li{t} est pour \em{texte} et le~\li{l} est pour
\em{latin}.

\begindomaintable\relax
%
% Quote marks
%
"10 \tldblquoteleft
"11 \tldblquoteright
"12 \tldblquotebase
"13 \tlguillemetleft
"14 \tlguillemetright
"22 \tldblquote
"27 \tlquoteright*
"60 \tlquoteleft*
\enddomaintable
\begindomaintable\relax
%
% Punctuation marks
%
"21 \tlexclam*
"2C \tlcomma*
"2E \tlperiod*
"3A \tlcolon*
"3B \tlsemicolon*
"3F \tlquestion*
"BD \tlexclamdown
"BE \tlquestiondown
\enddomaintable
\begindomaintable\relax
%
% Delimiters
%
"28 \tlparenleft*
"29 \tlparenright*
"5B \tlbracketleft*
"5D \tlbracketright*
"7B \tlbraceleft*
"7D \tlbraceright*
\enddomaintable
\begindomaintable\relax
%
% Dashes
%
"15 \tlendash
"16 \tlemdash
"2D \tlhyphen*
"7F \tlhyphenchar*
\enddomaintable
\begindomaintable\relax
%
% Various Symbols
%
"20 \tlvisiblespace
"23 \tlnumbersign*
"24 \tldollar*
"25 \tlpercent*
"26 \tlampersand*
"2A \tlastersik*
"2B \tlplus*
"2F \tlslash*
"3C \tlless*
"3D \tlequal*
"3E \tlgreater*
"40 \tlat*
"5C \tlbackslash*
"5E \tlasciicircumflex*
"5F \tlunderscore*
"7C \tlbar*
"7E \tlasciitilde*
"9F \tlsection
"BF \tlsterling
\enddomaintable
\begindomaintable\relax
%
% Ligatures
%
"9C \tlIJ
"BC \tlij
"C6 \tlAE
"E6 \tlae
"D7 \tlOE
"F7 \tloe
"DF \tlEsZet
"FF \tleszet
\enddomaintable
\begindomaintable\relax
%
% Letters with diacritics
%
"80 \tlAbreve
"81 \tlAogonek
"82 \tlCacute
"83 \tlCcaron
"84 \tlDcaron
"85 \tlEcaron
"86 \tlEogonek
"87 \tlGbreve
"88 \tlLacute
"89 \tlLcaron
"8A \tlLslash
"8B \tlNacute
"8C \tlNcaron
"8D \tlEng
"8E \tlOhungarumlaut
"8F \tlRacute
"90 \tlRcaron
"91 \tlSacute
"92 \tlScaron
"93 \tlScommaaccent
"94 \tlTcaron
"95 \tlTcommaaccent
"96 \tlUhungarumlaut
"97 \tlUring
"98 \tlYdieresis
"99 \tlZacute
"9A \tlZcaron
"9B \tlZdotaccent
"9D \tlIdotaccent
"9E \tldcroat
"A0 \tlabreve
"A1 \tlaogonek
"A2 \tlcacute
"A3 \tlccaron
"A4 \tldcaron
"A5 \tlecaron
"A6 \tleogonek
"A7 \tlgbreve
"A8 \tllacute
"A9 \tllcaron
"AA \tllslash
"AB \tlnacute
"AC \tlncaron
"AD \tleng
"AE \tlohungarumlaut
"AF \tlracute
"B0 \tlrcaron
"B1 \tlsacute
"B2 \tlscaron
"B3 \tlscommaaccent
"B4 \tltcaron
"B5 \tltcommaaccent
"B6 \tluhungarumlaut
"B7 \tluring
"B8 \tlydieresis
"B9 \tlzacute
"BA \tlzcaron
"BB \tlzdotaccent
"C0 \tlAgrave
"C1 \tlAacute
"C2 \tlAcircumflex
"C3 \tlAtilde
"C4 \tlAdieresis
"C5 \tlAring
"C7 \tlCcedilla
"C8 \tlEgrave
"C9 \tlEacute
"CA \tlEcircumflex
"CB \tlEdieresis
"CC \tlIgrave
"CD \tlIacute
"CE \tlIcircumflex
"CF \tlIdieresis
"D0 \tlEth
"D1 \tlNtilde
"D2 \tlOgrave
"D3 \tlOacute
"D4 \tlOcircumflex
"D5 \tlOtilde
"D6 \tlOdieresis
"D8 \tlOslash
"D9 \tlUgrave
"DA \tlUacute
"DB \tlUcircumflex
"DC \tlUdieresis
"DD \tlYacute
"DE \tlThorn
"E0 \tlagrave
"E1 \tlaacute
"E2 \tlacircumflex
"E3 \tlatilde
"E4 \tladieresis
"E5 \tlaring
"E7 \tlccedilla
"E8 \tlegrave
"E9 \tleacute
"EA \tlecircumflex
"EB \tledieresis
"EC \tligrave
"ED \tliacute
"EE \tlicircumflex
"EF \tlidieresis
"F0 \tleth
"F1 \tlntilde
"F2 \tlograve
"F3 \tloacute
"F4 \tlocircumflex
"F5 \tlotilde
"F6 \tlodieresis
"F8 \tloslash
"F9 \tlugrave
"FA \tluacute
"FB \tlucircumflex
"FC \tludieresis
"FD \tlyacute
"FE \tlthorn
\enddomaintable


\formalpar Caractères absents

Le contrat entre les commandes précentes |\tlwhatever| et
l'utilisateur de la commande est que la commande fait de son mieux
pour produire le glyphe en regard, il ne s'agit de rien de plus que
d'une promesse de bonne volonté.

On peut utiliser la commande |\missingchardef\tlthorn|, par exemple, pour
annoncer que le glyphe de |\tlthorn| n'est pas disponible avec les
fontes chargées. Les commandes
\begincode
\missingcharhush
\missingcharverbose
\missingcharstop
\endcode
sélectionnent le comportement des caractères dont le glyphe est
indisponible. La première passe tout sous silence mais reporte un
message à la fin du traitement du fichier principal, la seconde
affiche un message à chaque caractère manquant rencontré et la
troisième arrête le traitement.


\formalpar Domaine des symboles du texte

\def\tsfontselect{\fontpage{S1}\fontselect}
\begindomaintable\tsfontselect
"00 \tsgrave
"01 \tsacute
"02 \tscircumflex
"03 \tstilde
"04 \tsdieresis
"05 \tshungarumlaut
"06 \tsring
"07 \tscaron
"08 \tsbreve
"09 \tsmacron
"0A \tsdotaccent
"0B \tscedilla
"0C \tsogonek
"0D \tsquotebase
"12 \tsdblquotebase
"15 \tstwelveudash
"16 \tsthreequartersemdash
"17 \tscwmcapital
"18 \tsarrowleft
"19 \tsarrowright
"1A \tstieaccentlowercase
"1B \tstieaccentcapital
"1C \tstieaccentlowercasenew
"1D \tstieaccentcapitalnew
"1F \tscwmascender
"20 \tsblanksymbol
"24 \tsdollar
"27 \tsquotesingle
"2A \tsasteriskmath
"2C \tscomma
"2D \tshyphendbl
"2E \tsperiod
"2F \tsfraction
"30 \tszerooldstyle
"31 \tsoneoldstyle
"32 \tstwooldstyle
"33 \tsthreeoldstyle
"34 \tsfouroldstyle
"35 \tsfiveoldstyle
"36 \tssixoldstyle
"37 \tssevenoldstyle
"38 \tseightoldstyle
"39 \tsnineoldstyle
"3C \tsangleleft
"3D \tsminus
"3E \tsangleright
"4D \tsmho
"4F \tsbigcircle
"57 \tsohm
"5B \tsdblbracketleft
"5D \tsdblbracketright
"5E \tsarrowup
"5F \tsarrowdown
"60 \tsgrave
"62 \tsborn
"63 \tsdivorced
"64 \tsdied
"6C \tsleaf
"6D \tsmarried
"6E \tsmusicalnote
"7E \tstildelow
"7F \tsvarhyphendbl
"80 \tsbreve
"81 \tscaron
"82 \tshungarumlaut
"83 \tsdblgrave
"84 \tsdagger
"85 \tsdaggerdbl
"86 \tsdblverticalbar
"87 \tsperthousand
"88 \tsbullet
"89 \tscentigrade
"8A \tsdollaroldstyle
"8B \tscentoldstyle
"8C \tsflorin
"8D \tscolonmonetary
"8E \tswon
"8F \tsnaira
"90 \tsguarani
"91 \tspeso
"92 \tslira
"93 \tsrecipe
"94 \tsinterrobang
"95 \tsgnaborretni
"96 \tsdong
"97 \tstrademark
"98 \tspermyriad
"99 \tsvarparagraph
"9A \tsbaht
"9B \tsnumero
"9C \tsdiscount
"9D \tsestimated
"9E \tsopenbullet
"9F \tsservicemark
"A0 \tsquillbracketleft
"A1 \tsquillbracketright
"A2 \tscent
"A3 \tssterling
"A4 \tscurrency
"A5 \tsyen
"A6 \tsbrokenbar
"A7 \tssection
"A8 \tsdieresis
"A9 \tscopyright
"AA \tsordfeminine
"AB \tscopyleft
"AC \tslogicalnot
"AD \tspublished
"AE \tsregistered
"AF \tsmacron
"B0 \tsdegree
"B1 \tsplusminus
"B2 \tstwosuperior
"B3 \tsthreesuperior
"B4 \tsacute
"B5 \tsmu
"B6 \tsparagraph
"B7 \tsperiodcentered
"B8 \tsreferencemark
"B9 \tsonesuperior
"BA \tsordmasculine
"BB \tsradical
"BC \tsonequarter
"BD \tsonehalf
"BE \tsthreequarters
"BF \tseuro
"D6 \tsmultiply
"F6 \tsdivide
\enddomaintable


\section Fontes de caractères et mode mathématique

Le mode mathématique de \TeX\ est celui dans lequel celui-ci prépare
les formules mathématiques. On discute ici des commandes spécifiques à
l'utilisation des fontes pour le mode mathématique.

\formalpar Assigner des fontes aux registres d'une famille

L'appel |\mathfontwish\mit{12pt}{10pt}{7pt}\as\yoyo| a deux effets.
Premièrement, tout se passe comme si les séquences
\begincode
\fontsize{12pt}\fontwish\selection\as\yoyotext
\fontsize{10pt}\fontwish\selection\as\yoyoscript
\fontsize{7pt}\fontwish\selection\as\yoyoscriptscript
\endcode
étaient éavluées, mais le paramètre |\fontsize| est
restauré. Deuxièmement les registres décrivant la famille |\mit| sont
remplis avec les valeurs obtenues, ce qui équivaut à évaluer
\begincode
\textfont\mit=\yoyotext
\scriptfont\mit=\yoyoscript
\scriptscriptfont\mit=\yoyoscriptscript
\endcode
La commande~|\mathfontwish| ressemble à la forme~%
|\fontwish\selection...|; avec des paramètres spécialement adaptés à
son rôle. Rappelons qu'on peut sauvegarder et restaurer les paramètres
de sélection des fontes en travaillant dans un groupe ou en utilisant
les commandes~|\fontselectionsave| et~|\fontselectionload|.


\formalpar Familles pour le mode mathématique

Les familles suivantes sont proposées par BHRID \TeX\ pour composer les
variables dans le mode mathématique:
\beginlist\tag\literal
\item{\string\rm} Romaines, famille |z@|;
\item{\string\it} Math Italics, famille |\@ne|;
\item{\string\bf} Romaines Grasses, famille |\bffam|;
\item{\string\sf} Sans Serifs, famille |\sffam|;
\item{\string\mit} Math Italics, famille |\@ne|;
\item{\string\cal} pour les belles majuscules, famille |\mathscriptfam|;
\item{\string\ef} Euler Fraktur, famille |\effam|;
\endlist
Les séquences |\rm|, |\it|, |\bf|, et |\sf| modifient également des
paramètres pour le mode horizontal.

Par exemple,
$$
{\sf\Gamma}\otimes\mathop{\hbox{\sf Hom}}({\ef A}, {\ef B})
\simeq
{\cal C}({\bf\Gamma})
$$
est obtenu avec
\begincode
$$
{\sf\Gamma}\otimes\mathop{\hbox{\sf Hom}}({\ef A}, {\ef B})
\simeq
{\cal C}({\bf\Gamma})
$$
\endcode
Rappelons que les commandes modifiant les familles dans le mode
mathématique n'affectent que les variables, il faut mettre en ½uvre
d'autres moyens pour obtenir des symboles gras, par exemple; ou
modifier le comportement de toute autre élément du mode mathématique.

\formalpar Paramètres du mode mathématique liés %
 à la taille du texte

Lorsqu'on modifie la taille du texte au moyen d'une commande
|\tenpoint|, \em{etc.}, on souhaite que la taille des caractères du
mode mathématique suivent ces changements. La commande
|\familysizeadjust| opère les changements nécessaires pour une famille
donnée du mode mathématique;
l'appel~|\familysizeadjust\mitfont\@ne\mit| affecte aux regsitres
|\textfont\@ne|, |\scriptfont\@ne|, |\scriptscriptfont\@ne| la fonte
courante après l'appel de |\mitfont|, avec les tailles fournies par la
table~|\familysizetable|~(cf. infra). De plus, chacune de ces fontes
est chargée dans les registres de fontes de \TeX\ nommés |\mittext|,
|\mitscript|, et~|\mitscriptscript|.

Comme annoncé ci-dessus, la table |\familysizetable| décrit les
tailels à utiliser pour charger les trois fontes d'une famille du mode
mathématique, en fonction de la taille du texte, telle qu'elle est
donnée par le registre~|\textsize|. Chaque triplet indexé par le
paramètre |\textsize| a comme membres: |\fontsize| pour |\textfont|,
pour |\scriptfont|, et pour~|\scriptscriptfont|.
\begincode
\def\familysizetable{%
  \\{{12}{{12pt}{8pt}{6pt}}}%
  \\{{11}{{11pt}{8pt}{6pt}}}%
  \\{{10}{{10pt}{7pt}{5pt}}}%
  \\{{9}{{9pt}{6pt}{5pt}}}%
  \\{{8}{{8pt}{6pt}{5pt}}}%
  \\{{7}{{7pt}{6pt}{5pt}}}%
  \\{{6}{{6pt}{5pt}{5pt}}}%
  \\{{5}{{5pt}{5pt}{5pt}}}%
}
\endcode
Cette table peut être étendue et augmentée avec les commandes de
manipulation des listes. Une classe de document peut redéfinir
complètement cette table ou préparer un mécanisme différent pour
ajuster les tailles des fontes.

Pour appeler automatiquement |\familysizeadjust| quand nécessaire, on
dit par exemple
\begincode
\addhook{\familysizeadjust\mitfont\@ne\mit}\to\mathsizeadjust
\endcode
et les tailles des fontes de la famille |\@ne|~\em{(math italics)} ont
leurs tailles ajustées automatiquement par les
commandes~|\tenpoint|,~\em{etc.}


\formalpar Belles majuscules

\def\belmaj{%
\halign{\qquad\hfil##\hfil&\quad##\hfil\cr
${\cal F}(A,B)$&toutes les fonctions de~$A$ dans~$B$\cr
${\cal P}(A)$&ensemble des parties de~$A$\cr
${\cal T}_X$&topologie de l'espace~$X$\cr
${\cal C}(X,\R)$&ensemble des fonctions numériques de~$X$
       continues\cr
${\cal V}_X(x)$&ensemble des voisisanges de~$x$ dans~$X$\cr
}\noindent\ignorewhitespace}%
On utilise en mathématiques de belles majuscules dans de nombreuses
situations, par exemple
\belmaj
On obtient ces majusucules dans le mode mathématique grâce à la
commande~|\cal|, comme par exemple dans~|{\cal F}|. Les fontes
préchargées dans~BHRID~\TeX\ proposent plusieurs jeux de belles
majuscules, on choisit ce jeu avec la commande~|\selectmathcal|, les
appels possibles sont
\begincode
\selectmathcal{formal}
\selectmathcal{informal}
\selectmathcal{eulerfraktur}
\selectmathcal{eulerscript}
\endcode
Qui donnent respectivement
\selectmathcal{formal}\mathsizeadjust\belmaj\medbreak
\selectmathcal{informal}\mathsizeadjust\belmaj\medbreak
\selectmathcal{eulerfraktur}\mathsizeadjust\belmaj\medbreak
\selectmathcal{eulerscript}\mathsizeadjust\belmaj\medbreak
Les caractères du jeu \li{formal} se rapprochent de ce que l'on peut
trouver dans beaucoup de livres parus aux alentours des années
soixante; le second jeu~\li{informal} est notamment utilisé dans {\sl
the Art of Computer Programming} de~\nm{D.~E.~Knuth}, ainsi que dans
de nombreux polys; le quatrième jeu~\li{eulerscript} apparaît dans le
livre de {\sl Géométrie} de~\nm{M.~Audin}; tandis que le troisième
jeu~\li{euler fraktur} apparaît dans des traités plus anciens, comme
l'{\sl Algebra} de~\nm{Van der Waerden}. Les majuscules \li{euler
fraktur} peuvent par ailleurs être utilisées à travers la
commande~|\ef|.

En plus de ces noms courts, on peut utiliser des noms longs,
\begincode
\selectmathcal{Ralph Smith Formal Script}
\selectmathcal{Computer Modern Informal Script}
\selectmathcal{Euler Fraktur}
\selectmathcal{Euler Script}
\endcode
Il s'agit d'\em{alias} qui permettent seulement d'avoir de la
cohérence avec les commandes comme~|\selecttl{Latin Modern}| en
donnant comme argument un nom de fonte de caractères ou un nom de
famille de fonte caractères. Si vous ajoutez des des nouvelles fontes
pouvant être utilisées comme~|\mathcalfont|, privilégiez les noms
longs.


\bye

%%% End of file `fontes.tex'
