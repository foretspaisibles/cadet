%%% agraf.tex -- Actions des groupes affines

% Copyright (C) 2006-2014 Michael Grünewald
% All rights reserved.

\section Actions des groupes affines

Une représentation linéaire d'un groupe affine~$G$ est un morphisme
de~$G$ dans le groupe linéaire d'un espace vectoriel~$E$ de dimension
finie. Si on veut laisser dans l'anonymat le morphisme lui-même et
qu'on préfère parler de~$E$, on dit qu'il est un espace de
représentation linéaire de~$G$, ou pour être bref, qu'il est une
représentation de~$G$.

Dans cet exposé, on utilise ces représentations pour étudier les
groupes affines. Elles permettent d'étendre la décomposition de~Jordan
aux groupes non linéaires, de construire les espaces homogènes, et
c'est dans l'étude de la représentation adjointe qu'apparaissent les
fameuses racines.

\paragraph Représentation adjointe
Lorsque~$G$ est un groupe affine, l'automorphisme intérieur~$\eta(x)$
associé à~$x\in G$ stabilise l'élément neutre~$I$ de~$G$, et au niveau
infinitésimal, l'application tangente~$T_I \eta(x)$ est un
automorphisme du groupe linéaire de~$\lie g = T_I G$. De plus
l'application
$$
\application
\Ad:G\to \GL(\lie g)\cr
x\mapsto T_I \eta(x)\cr
$$
est d'après la \em{chain-rule} un morphisme de groupes, on l'appelle
la représentation adjointe de~$G$.

On montre au \ref[pr:pousse] comment la structure de la représentation
adjointe influence celle des autres.

\medskip
Pour l'étude des groupes affines, on a besoin d'autres espaces de
représentations, comme ceux que produit~le

\proposition Théorème de Chevalley
Soit~$G$ un groupe affine et~$H$ un sous-groupe fermé de~$G$. Il
existe une représentation~$E$ de~$G$ et une droite~$L$ de~$E$, tels
que~$H$ est le groupe d'isotropie de~$L$.

\paragraph [pa:ligraf] Linéarisation de groupes affines
En prenant~$H=\set{I}$ on montre que~$G$ est isomorphe à un
sous-groupe fermé d'un groupe linéaire, ce qui permet parfois de
travailler concrètement sur des matrices.

\paragraph
En passant à l'action de~$G$ sur~$\P(E)$, l'orbite de~$\P(L)$ est
une variété quasi-projective, à travers laquelle se factorisent les
morphismes constants modulo~$H$.

Il ne s'agit pas d'un quotient ensembliste, par exemple
si~$G=\SL(2,k)$ et~$H=\left[\matrix{1&*\cr0&1\cr}\right]$, la variété
quotient confond tous les points fixes de~$H$.
% XXX vérifer l'exemple

\proposition Décomposition de Jordan
Soit~$x\in\GL(E)$. Il existe un, et un seul, couple~$(x_s,x_u)$
d'éléments de~$\GL(E)$ vérifiant les conditions: $x=x_s x_u$; $x_s$
est semi-simple, $x_u$ est unipotent; et~$x_s x_u = x_u x_s$. Les
endomorphsimes~$x_s$ et~$x_u$ commutent à tous les endomorphismes qui
commutent avec~$x$. Lorsqu'un sous-espace de~$E$ est stabilisé par~$x$
il est également stabilisé par~$x_s$ et par~$x_u$. Enfin, lorsque~$x$
et~$y$ commutent dans~$\GL(E)$, on a~$(xy)_s = x_s y_s$ et~$(xy)_u =
x_u y_u$.

% Lorsqu'on représente grâce à~\re[pa:ligraf] un groupe affine dans un
% groupe linéaire, on peut transporter

\proposition [pr:cj] Décomposition de Chevalley-Jordan
Soit~$G$ un groupe affine et soit~$x\in G$. il existe un, et un seul,
couple~$(x_s,x_u)$ d'éléments de~$G$ tels que: $x=x_s x_u$; dans toute
représentation linéaire de~$G$, $x_s$ est semi-simple, $x_u$ est
unipotent; et~$x_s x_u = x_u x_s$. Lorsque~$\phi$ est un morphisme de
groupes~$\phi(x)_s = \phi(x_s)$ et~$\phi(x_u) = \phi(x)_u$.

Lorsque~$G$ est un groupe linéaire, la décompostion de
Chevalley-Jordan est la décomposition de Jordan.

Pour prouver que~$(x_s,x_u)$ est une décompositon de Chevalley-Jordan,
il suffit de trouver \em{une}~représentation fidèle de~$G$ dans
laquelle~$x_s$ est semi-simple et~$x_u$ unipotent.

L'ensemble~$G_u$ est fermé dans~$G$, mais ça n'est que rarement le cas
pour~$G_s$. Par exemple~$\GL(E)_s$ est une partie ouverte dense
de~$\GL(E)$, distincte de~$\GL(E)$ lorsque~$E$ est de dimension plus
grande que~un.

\paragraph [pa:redfd]
Comme la décomposition de Chevalley-Jordan est compatible avec les
homomorphismes, et en particulier avec les linéarisations des groupes
affines, elle hérite certaines des propriétés de la décomposition de
Jordan. Par exemple, un groupe abélien dont tous les éléments sont
semi-simples peut-être réduit à la forme diagonale dans tous ses
espaces de représentation.

%%% End of file `agraf.tex'
