%%% torcn.tex -- Groupes résolubles

% Copyright (C) 2006-2014 Michael Grünewald
% All rights reserved.

\section Groupes résolubles

On est motivé par la symétrie de~$\Phi$ remarquée pour~$\SL(2,k)$, et
pour décrire comment la décomposition~\ref[eq:decomp] de~$E$
selon~$\Psi(G,T)$ est transformée par les automorphismes de~$T$
intérieurs à~$G$.

\paragraph Suite dérivée et suite centrale
Si~$A$ et~$B$ sont des parties d'un groupe~$G$, on note~$(A,B)$ le
groupe engendré par les commutateurs d'éléments pris une fois dans~$A$
et une fois dans~$B$. Pour un groupe~$G$ on définit la suite
dérivée~$\Derive G$ et la suite Centrale~$\Central G$ par les
relations
$$
\Derive^0 G = G
,\qquad\Derive^{i+1}G = (\Derive^iG, \Derive^iG)
;\qquad\mbox{et}\qquad
\Central^0 G = G
,\qquad\Central^{i+1}G = (\Central^iG,G)
.
$$
Il s'agit de suites décroissantes de sous-groupes de~$G$, pour chaque
rang on a~$\Derive^i G \subset \Central^i G$. Lorsque la suite dérivée
converge vers~1, on dit que~$G$ est résoluble; lorsque même la suite
centrale converge vers~1, on dit que~$G$ est nilpotent. On
note~$G^\infty = \lim \Central G$.

Comme conséquence du théorème de rigidité des tores~\ref[pr:rigid] on
prouve que

\proposition
Si~$H$ est un sous-groupe d'un groupe
résoluble~$G$, peut-être pas fermé  et formé d'éléments tous
semi-simples, alors
\beginenumeration
\item 1 $H$ est inclus dans un tore;
\item 2 le centralisateur et le normalisateur de~$H$ dans~$G$ sont égaux
      entre eux, et connexes.
\endenumeration

Cela signifie que pour un groupe résoluble, on n'a pas
``d'automorphismes'' de la décomposition~\ref[eq:decomp] associés à
des automorphismes de~$T$ intérieurs à~$G$.


\proposition [pr:grres] Théorème de structure pour les groupes résolubles
Soit~$G$ un groupe, que l'on suppose résoluble et
connexe. L'ensemble~$G_u$ est alors un sous-groupe fermé, normal
(résoluble) et connexe de~$G$, contenant le sous-groupe
dérivé~$(G,G)$. Le groupe~$G_u$ a une chaîne décroissante de
sous-groupes de~$G$ connexes et fermés, tous normaux dans~$G$, et de
codimensions relatives toutes égales à~1.
\endgraf
Les tores maximaux de~$G$ sont conjugués sous l'action de~$G^\infty$,
et on~a $G=T\ltimes G_u$ lorsque~$T$ est l'un d'entre eux.

Lorsque~$G$ est nilpotent, $G^\infty = 1$ et~$G$ n'a qu'un tore
maximal~$T$, il est alors produit direct de~$T$ et de~$G_u$. Si
réciproquement~$G_s$ est un sous-groupe de~$G$, il faut que~$G$ soit
nilpotent.

\paragraph Vocabulaire des groupes semi-simples
On considère d'après \ref[pr:grres] que les groupes résolubles sont
relativement bien compris et on aborde l'étude des groupes affines
généraux de la façon suivante:

A. On étudie le radical~$R(G)$ du groupe,~$G$: parmi les sous-groupes
   connexes et résolubles de~$G$, il en existe un plus grand que tous
   les autres,\footnote*{lorsque~$A$ et~$B$ sont des sous-groupes
   de~$G$ qui sont résolubles, connexes, et normaux dans~$G$,
   l'adhérence de $AB$ dans~$G$ a ces mêmes qualités.} on l'appelle
   radical de~$G$. Par exemple
$$
R(\GL(n,k)) = k^* I
,\qquad
R(\SL(n,k)) = \set{I}
,\qquad\mbox{et}\qquad
R(\Sp(n,k)) = \set{I}.
$$

B. On étudie le groupe quotient~$G/R(G)$, dont le radical
   est~$\set{I}$. Ces groupes dont le radical est réduit au
   groupe nul sont appelés semi-simples.

C. On étudie l'extension de~$G/R(G)$ par~$R(G)$.

\proposition
Si~$f:G_1\to G_2$ est un épimorphisme de groupes affines,
alors~$f(R(G_1))=R(G_2)$.


\paragraph Vocabulaire des groupes réductifs
On est en fait amené à considérer un classe de groupes un peu plus
large que celle des seuls groupes semi-simples. C'est celle des
groupes unipotents, dont la propriété caractéristique est que la
partie unipotente de leur radical, ou leur radical unipotent, soit
nul, soit~$R(G)_u = \set{I}$.

Pour ces groupes~$R(G)$ est la composante neutre du centre de~$G$,
c'est un tore qui rencontre~$\Derive^1G$ selon un ensemble fini.

D'un point de vue technique, l'intérêt que l'on porte aux groupes
réductifs est justifié par

\proposition
Soit~$G$ un groupe affine et~$N$ le normalisateur d'un tore
de~$G$. Si~$G$ est réductif, alors~$N$ est lui-aussi réductif.

Il n'y a pas d'anlogue pour les groupes semi-simples, puisque par
exemple le normalisateur dans~$\SL(3,k)$ du tore dont l'élément
générique est
$$
\Matrix{s&0&0\cr0&s&0\cr0&0&1/s^2\cr}
$$
est isomorphe à~$\GL(2,k)$ et n'est pas réductif.

%%% End of file `torcn.tex'
