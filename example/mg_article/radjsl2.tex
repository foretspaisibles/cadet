%%% radjsl2.tex -- Représentation Adjointe de SL2k

% Copyright (C) 2006, 2013 Michael Grünewald
% All rights reserved.

\begingroup
\def\G{\SL(2,k)}%
\def\g{\lie{sl}(2,k)}%
On étudie \ref[eq:decomp] lorsque~$T$ est un tore maximal de~$\G$
et~$E$ la représentation adjointe~$\g$.

Comme un tore de~$\G$ est conjugué dans~$\GL(2,k)$ à un tore diagonal
\ref[pa:redfd] et que~$\G$ est normal dans~$\GL(2,k)$ le tore
$$
T =\set{\Matrix{s&0\cr0&1/s\cr}\mid s\in G_m}
\eqlabel[eq:sl2t]
$$
est maximal dans~$\G$, et les tores maximaux de~$\G$ lui sont
conjugués: de bien des façons, l'étude de~$\g$ sous l'action de~$T$
ne dépend pas du choix particulier du tore maximal, mais juste de
cette dernière qualité. Pour étudier la représentation adjointe
de~$\G$ on introduit les sous-groupes à un paramètre
$$
h(s) = \Matrix{s&0\cr 0&1/s\cr}
\quad s\in G_m
\qquad\mbox{et}\qquad
\bar h(s) = \Matrix{1/t&0\cr 0&t\cr}
\quad t\in G_m
$$
et les morphismes
$$
x(u) = \Matrix{1&u\cr0&1}
\quad u\in G_a
\qquad\mbox{et}\qquad
y(v) = \Matrix{1&0\cr v&1}
\quad v\in G_a
.
\eqlabel[radjsl2xy]
$$
Les caractères de~$T$ forment un groupe~$X(T)$ monogène admettant
deux générateurs
$$
\xi = h^*
\qquad\mbox{et}\qquad
\bar\xi = \bar h^*
$$
où l'étoile signifie la base duale pour~\ref[pa:tdual].

Déterminons l'ensemble~$\Phi$ des racines de~$\G$ relatives  à~$T$:
les vecteurs tangents
$$
H = h'(1) = \Matrix{1&\hphantom{-}0\cr 0 & -1\cr}
,\qquad
X = x'(0) = \Matrix{0&1\cr 0 & 0\cr}
,\qquad\mbox{et}\qquad
Y = y'(0) = \Matrix{0&0\cr 1 & 0\cr}
$$
forment un système-base de~$\g$ et on calcule~$\Ad h(s) H = H$, puis
$$
\Ad h(s) X
= \left.{d\over du}\;h(s)x(u)\inv{h(s)} \right|_{u = 0}
=\; \left.{d\over du}\;x\left(\xi(s)^2u\right)\right|_{u = 0}
=\; \xi^2(s) X
$$
ainsi que~$\Ad h(s) Y = \bar\xi^2 (s) Y$. On a donc
$$
\Phi = \set{\xi^2, \bar\xi^2} = \set{\alpha,\bar\alpha}
\eqlabel[radjsl2root]
$$
et
$$
\g
= {\lie t}\oplus \g_{\alpha} \oplus \g_{\bar\alpha}
= kH \oplus kX \oplus kY
.
$$
On peut remarque que dans le calcul de la représentation adjointe, le
chemin~$x$ choisi pour représenter l'espace tangent~$\g_\alpha$
vérifie~$h(s)x(u)\inv{h(s)} = x(\alpha(s)u)$. Tous les chemins
représentant cet espace tangent ne partagent pas cette propriété, par
exemple~$x(u^2)$ ou~$x(u^3)$, on a fait un choix approprié de~$x$.
\endgroup

%%% End of file `radjsl2.tex'
