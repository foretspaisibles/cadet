%%% torrep.tex -- Les tores et leurs représentations

% Copyright (C) 2006, 2013 Michael Grünewald
% All rights reserved.

\section Représentations des tores

\paragraph
Un tore est un groupe affine isomorphe à un groupe
diagonal~$D(n,k)$. Pour qu'un groupe affine~$T$ soit un tore, il faut
et il suffit qu'il soit connexe et que le groupe de ses
caractères~$X(T) = \homo(T,G_m)$ soit une base de l'algèbre~$k[T]$ des
fonctions régulières sur~$T$.

\paragraph [pa:tdual]
Pour un tore~$T$, le groupe abélien~$X(T)$ est libre, de type fini. Le
groupe~$Y(T) = \homo(G_m,T)$ des sous-groupes à un paramètre de~$T$ est
le dual de~$X(T)$ grâce à la forme bilinéaire
$$
\langle\chi,\lambda\rangle = n
\qquad\mbox{pour l'unique entier~$n$ tel que~$\chi\circ\lambda(s) = s^n$.}
$$


\paragraph
Un tore~$T$ d'un groupe affine~$G$ agit diagonalement dans toutes les
représentations linéaires~$E$ de~$G$~\ref[pa:redfd], c'est à dire que
$$
E =
\bigoplus_{\chi\in X(T)} E_\chi =
\bigoplus_{\chi\in X(T)}
\set{x\in E\mid \forall t\in T\quad tx = \chi(t) x}
.
\eqlabel[eq:decomp]
$$
Les éléments du support~$\Psi$ de cette somme sont les poids de~$T$
dans la représentation~$E$.

\paragraph [pa:radjsl2]%
  Représentation adjointe du deuxième groupe spécial linéaire
\input radjsl2

\paragraph [pa:rstdsl2]%
  Représentation standard du deuxième groupe spécial linéaire
\input rstdsl2

\paragraph Deuxième groupe spécial linéaire et formes quadratiques
\input rsm2sl2

\paragraph
On peut observer que dans toutes les représentations de~$\SL(2,k)$,
l'ensemble~$\Psi$ des poids du tore~$T$ \ref[eq:sl2t] est symétrique par
rapport à l'automorphisme~$\theta(x)=\inv x$ du groupe abélien~$T$,
c'est-à-dire que
$$
\chi\in\Psi
\qquad\mbox{si, et seulement si,}\qquad
\theta^*\chi\in\Psi
.
$$
Cette symétrie s'explique par le fait que~$\theta$ est réalisé par un
automorphisme intérieur à~$\SL(2,k)$, celui associé par exemple à
l'élément~$\left[\matrix{\hphantom{-}0&1\cr-1&0\cr}\right]$ du
normalisateur de~$T$ dans~$\SL(2,k)$.

Dans ce dernier cas, il ne peut il y avoir qu'une seule symétrie
parceque~$T\simeq G_m$ n'a qu'un automorphisme. Les tores de dimension
plus grande ont plus d'automorphismes, mais très peu
peuvent être intérieurs au groupe ambiant:

\proposition [pr:rigid] Rigidité des tores
Si~$T$ est un tore dans un groupe affine~$G$, la composante neutre du
centralisateur de~$T$ dans~$G$ égale celle du normalisateur de~$T$
dans~$G$. En particulier l'image du morphisme
$$
\app\eta:N_G(T)\arrow\to\Aut(T)\cr
$$
est finie puisque~$\ker\eta=C_G(T)$.

On peut pressentir que l'image de~$\eta$ est finie en associant à
chaque élément~$x$ de~$N_G(T)$ l'automorphisme de~$X(T)$ associé
à~$\eta(x)$. Les automorphismes de~$X(T)$ sont un groupe isomorphe à
un~$\GL(r,\Z)$, et les groupes algébriques comme~$N_G(T)$ n'ont qu'un
nombre fini de composantes connexes. Cet argument n'est pas correct
car la flèche décrite n'est pas un morphisme de variétés algébriques.

\paragraph Racines
Lorsqu'on décompose la représentation adjointe d'un groupe affine~$G$
relativement à un tore~$T$ comme dans~\ref[eq:decomp],
l'ensemble~$\Phi(G,T) = \Psi-\set{1}$ est l'ensemble des racines
de~$G$ relatives à~$T$. Voici un exemple de la façon dont les racines
structurent les représentations de~$G$:

\proposition [pr:pousse]
Soit~$\rho:G\to E$ une représentation linéaire de~$G$, $T$ un tore
de~$G$, et~$\Psi$ les poids de~$T$ dans~$E$. Considérons d'une part un
poids~$\chi\in\Psi$ et l'espace propre~$E_\chi$ qui lui est associé,
d'autre part une racine~$\alpha\in\Phi$.
Supposons qu'il existe un morphsime~$x:G_a\to G$ tel que~$x'(0)\in
{\lie g}_\alpha$ et que
$$
t x(u)\inv t = x\left(\alpha(t) u\right)
.
$$
Dans ces conditions l'action de~$x$ sur~$E_\chi$ vérifie:
$$
x(u) E_\chi \subset \sum_{r\ge0} E_{\alpha^r\chi}
.
\eqlabel[eq:exrac]
$$

\proof
On réduit simultanément l'action de~$T$ et celle de~$U_\alpha = \Im x$
aux formes diagonale et triangulaire supérieure, dans cette base~$e\in
E^I$ bien choisie, le vecteur~$e_i$ est propre pour le poids~$\chi_i
\in \Psi(E,T)$. On étudie ensuite les entrées~$x(u)_{ij}$ de l'action
de~$x(u)$ dans cette base bien choisie pour établir~\ref[eq:exrac].

\proposition Lemme. Un petit ``théorème de Lie-Kolchin.''
Il existe une base~$e$ de~$E$ dans laquelle~$x(u)$ agit par matrices
triangulaires supérieures et~$T$ par matrices diagonales.

\proof Preuve du lemme.
Pour réduire l'action de~$G'= TU_\alpha$ à la forme souhaitée, il
suffit de trouver une droite stable sous~$G'$ dans toutes ses
représentations. On applique ce résultat à~$E$ pour obtenir une
droite~$L$, puis à~$E' = E/L$ pour obtenir une droite~$L'$, puis à
~$E'' = E'/L'$, etc. L'action de~$G'$ diagonale dans~$L \times
L'\times\cdots$ est triangulaire dans~$E$.
\par
Dans une représentation~$E$ de~$G'$ le sous-groupe
unipotent~$U_\alpha$ est trigonalisable et admet donc un point fixe
non-nul. Comme~$T$ normalise~$U_\alpha$, il stabilise l'espace
des points-fixes de~$U_\alpha$, comme celui-là n'est pas réduit à zéro
et que les actions du tore~$T$ se réduisent à la forme diagonale, il
contient une droite fixée par~$G'$, ce qu'il fallait démontrer.

\proof Revenons à la preuve de~\ref[pr:pousse].
Étudions les coefficients matriciels de~$x(u)$ pour cette base, soit
$x(u)_{ij} = \langle e_j^*, x(u) e_i \rangle$. On a d'une part
$$
\langle e_j^*, t x(u) e_i \rangle
= \langle e_j^*, x(\alpha(t)u) t e_i \rangle
= \chi_i(t)\langle e_j^*, x(\alpha(t)u) e_i \rangle
= \chi_i(t) x(\alpha(t)u)_{ij}
$$
et d'autre part
$$
\langle e_j^*, t x(u) e_i \rangle
= \langle e_j^*, t x(u) e_i \rangle
= \langle e_j^*, \chi_j(t) x(u)_{ij} e_j \rangle
= \chi_j(t) x(u)_{ij}
$$
ce qui entraine
$$
x(u)_{ij}{\chi_j \inv\chi_i}(t) - x(\alpha(t)u)_{ij} = 0
.
$$
En fixant~$t$, cette relation devient une identité polynomiale en~$u$,
si le coefficient de degré~$r$ de~$x(u)_{ij}$ n'est pas nul, il
vient
$$
\chi_j \inv\chi_i = \alpha^r
$$
et comme~$X(T)$ est libre, cela n'arrive que pour une valeur
de~$r$. La matrice de~$x(u)$ est triangulaire supérieure, ses
entrées pour~$j < i$ sont nulles et pour que l'image du
vecteur~$e_i$ ait une coordonnée non nulle selon~$e_j^*$ il faut
donc que~$i\ge j$ et que~$\chi_j = \alpha^r\chi_i$, ce qu'il fallait
démontrer.

On a aussi démontré que l'entrée~$x(u)_{ij}$ est nulle ou de la
forme~$au^r$, et comme~$x(u) = \exp(u x'(0))$ avec~$x'(0)$ nilpotent
dans~$\lie g$, on peut voir que~$r = |i -j|$. On exploite cette
information dans le cas de~$\SL(2,k)$: si~$m$ est la première
coordonnée nulle de~$T_I\rho\;x'(0)$ (application tangente de~$\rho$),
l'espace
$$
E' = E_{\chi_1}\oplus E_{\chi_1}\oplus \cdots\oplus E_{\chi_m}
$$
est stable sous l'action de~$x(u)$. Comme~$y(v)$ pousse les espaces
propres de~$T$ dans la direction~$-\alpha$~\ref[pa:radjsl2],
\ref[pr:pousse], il stabilise également~$E'$, et puisque ce dernier est
également stabilisé par~$T$, on peut conclure que

\proposition
Dans une représentation irréductible de~$\SL(2,k)$ les poids de~$T$
forment un ensemble
$$
\set{\bar\chi,\alpha\bar\chi,\dots,\bar\alpha\chi,\chi}
,
$$
où~$\chi$ et~$\bar\chi = \inv\chi$ sont congrus modulo~$\alpha$
dans~$X(T)$.  [\ref[pa:radjsl2] pour les notations]

%% Un des buts de cet exposé est de prouver que lorsque~$T$ est un tore
%% maximal d'un groupe semi-simple~$G$~\ref[XXX], $\Phi(G,T)$ est un
%% système de racines.

% XXX Systèmes de racines

%%% End of file `torrep.tex'
