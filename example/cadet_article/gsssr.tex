%%% gsssr.tex -- Groupes semi-simples et systèmes de racines

% Copyright © 2001–2015 Michael Grünewald
% All rights reserved.

\title Groupes semi-simples et systèmes de racines
\author\nm{M.~Le~Barbier}

Dans cet exposé, j'explique comment la structure des groupes
algébriques affines semi-simples peut-être en partie élucidée grâce à
l'analyse de certains de leurs sous-groupes: les tores et les
sous-groupes résolubles et connexes. L'exposé doit permettre de
comprendre comment on associe à chaque groupe affine semi-simple un
système de racines, et comment la structure du groupe est reflétée par
ce système, puis comment la classification des systèmes de racines
aboutit à celle des groupes semi-simples.

Dans l'exposé de la théorie des groupes semi-simples que donne
J.~Humphreys~[LAG], les actions de groupes interviennent comme outil
fondamental. Dans la première partie de l'exposé, j'introduis la
représentation adjointe et le théorème de Chevalley qui produisent les
représentation-outils qu'on utilise dans la suite. En montrant sur un
exemple comment la structure de la représentation adjointe influence
sur celle de toutes les autres, je veux faire apparaître les
structures qu'on étudie pour passer du particulier au général. Dans la
deuxième partie, je présente les résultats fondamentaux concernant les
sous-groupes de Borel d'un groupe affine et je montre sur quelques
exemples comment ils peuvent être utilisés dans l'analyse des groupes
affines. Dans la troisième partie, je décris ``la vue d'ensemble'' de
la structure des groupes semi-simples qu'on obtient grâce aux outils
de la seconde partie, puis je montre comment on associe à chaque
groupe semi-simple un système de racines.

%%% End of file `gsssr.tex'
