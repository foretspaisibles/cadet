% Copyright © 2001–2015 Michael Grünewald
% All rights reserved.

\class cadet_examination
\input um2_examen

\module         FLMA 303
\year           2007

\title          À propos du contrôle continu


\section Ce qui doit disparaître des copies

1. Les calculs de limites du genre
$$
\lim_{x\to+\infty} x^5
 = (+\infty)(+\infty)(+\infty)(+\infty)(+\infty)
 = (+\infty)
$$
n'ont pas de sens. Une des raisons pour lesquelles on ne donne pas de
sens à ce genre d'expressions est qu'à cause des \em{formes
indéterminées} apparaissant dans certaines expressions, on n'obtient
pas de règles de calcul dans~$[-\infty,+\infty]$ adaptées au calcul
des limites.

2. Les symboles~$\equiv$ et~$\approx$ apparaissent parfois dans des
calculs du type
$$
\lim_{x\to+\infty} x^5(1 + 1/x) \approx \lim_{x\to+\infty} x^5
 = +\infty
.
$$
Pourquoi ne pas utiliser une égalité là où elle est légitime au lieu
d'employer des symboles sans prendre la peine d'en préciser le sens?

3. On ne peut pas citer le théorème des valeurs intermédiaires dans
une question et prétendre qu'il atteste l'existence d'un extremum dans
une autre, c'est de l'escroquerie.

4. Quand on cite un un théorème, une proposition, les hypothèses font
partie de l'énoncé. Un énoncé privé d'hypothèses ne doit plus jamais
apparaître dans vos copies.


\section Ce qu'on préférerait ne pas voir

On doit justifier l'existence d'une limite avant de procéder à son
calcul (du moins lorsqu'on présente ses résultats).  La présentation
suivante:

{\sl Pour~$x$ voisin de~$+\infty$, on a $-x + x^5 = x^5(1-1/x^4)$
et~$1/x^4$ tend vers~$0$, on conclut donc que
$$
\lim_{x\to+\infty} -x + x^5 = +\infty
.
$$}
est à préférer à la suivante:
{\sl $\lim_{x\to+\infty} -x + x^5 = \lim_{x\to+\infty} x^5(1-1/x^4) =
+\infty$}.


\section Quelques bêtises en vrac

1. Les fonctions polynomiales ne sont pas continues et
dérivables~\em{par définition}. La continuité et la dérivabilité des
fonctions polynomiales font partie de leurs propriétés remarquables,
mais ne figurent pas dans leur définition!

2. Un terme infiniment proche de~$0$ est négligeable dans une
somme, mais pas dans un produit!

3. Beaucoup confondent ensemble d'arrivée d'une application et
ensemble des valeurs d'une fonction. Il faut réviser ces notions,
ainsi que la restriction et la corestriction (restriction à une partie
de l'ensemble d'arrivée) des fonctions.

4. La stricte croissance d'une fonction ne suffit pas à prouver
qu'elle tend vers~$+\infty$ en~$+\infty$. Cherchez de nombreux contre
exemples!

5. Pour une fonction~$f$ définie au voisinage de~$+\infty$,
l'existence de deux suites~$x_n$ et~$y_n$ convergeant vers~$+\infty$
et telles que~$f(x_n)$ converge vers~$X$ et~$f(y_n)$ converge vers~$Y$
permet lorsque~$X\not=Y$ de conclure que~$f$ n'a pas de limite
en~$+\infty$. Dans le cas contraire où~$X=Y$ on ne peut rien dire de
plus sur~$f$, et en particulier on ne peut pas conclure que~$f$
converge vers~$X$ en~$+\infty$.

6. Dans la question~2.2, la fonction~$1/x$ n'est pas un exemple car
elle est définie (et soit dit en passant, continue) de~$\R^*$
dans~$\R$, tandis qu'on demande une fonction définie de~$\R$
dans~$\R$. En revanche la fonction définie par~$g(x) = 1/x$
si~$x\not=0$ et~$g(0) = 0$ accompagnée de l'intervalle~$J=]-1, 1[$
donnent un exemple valide; avec de nombreuses variations possibles.

7. Pour une fonction réelle de la variable réelle~$f$ dérivable en un
point~$x_0$, la condition~$f'(x_0)$ est une condition nécessaire
d'extremum \em{local}. Ça n'est pas une condition suffisante
d'extremum (exemple~$f(x) = x^3$ et~$x_0 = 0$) et encore moins une
condition suffisante de maximum (trouvez de nombreux exemples).
C'est en général l'étude des variations d'une fonction qui permet de
trouver les extrema globaux parmi les extréma locaux; mais attention
une fonction peut très bien avoir des minima locaux sans pour autant
avoir un minimum global (trouvez ici aussi de nombreux exemples).

\bye
