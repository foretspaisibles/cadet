%%% ccont1.tex -- Un exemple d'examen

% Copyright © 2001–2015 Michael Grünewald
% All rights reserved.

\class cadet_examination
\input um2_examen

\module         FLMA 303
\year           2007

\begindirectives
Chacune de vos réponses doit être soigneuesement justifiée.
L'utilisation des calculatrices et la consultation de tout document
sont interdits. Durée de la composition: 1h30.
\enddirectives

\exercise Un théorème d'existence de racine %
             pour les fonctions polynomiales de degré impair

\question
Énoncez le théorème des valeurs intermédiaires.


\question Un cas particulier

\item{a)}
Rappelez le comportement de la fonction~$m$ de~$\R$ dans~$\R$ définie
par~$m(x) = x^5$, lorsque~$x\to-\infty$ et
lorsque~$x\to+\infty$. \em{(Rappelez l'existence, et le cas échéant la
valeur, d'une limite finie ou infinie.)}

\item{b)}
Étudiez le comportement de la fonction~$p$ de~$\R$ dans~$\R$ définie
par
$$
p(x) = 1 - 3x^2 + x^4 + x^5
,
$$
lorsque~$x\to-\infty$ et lorsque~$x\to+\infty$.

\item{c)}
Quelle est l'image de~$\R$ par l'application~$p$? Cette image
contient-elle~$0$?


\question Le cas général

On considère un nombre entier positif~$k$, des nombres réels~$a_0$,
$a_1$, \dots,~$a_{2k}$ et un nombre réel strictement
positif~$a_{2k+1}$. On forme alors la fonction
$$
f(x) = a_0 + a_1 x + \cdots + a_{2k+1} x^{2k+1}
$$
définie sur~$\R$, à valeurs dans~$\R$.

\item{a)}
Étudiez le comportement de la fonction~$f$ lorsque~$x\to-\infty$ et
lorsque~$x\to+\infty$.

\item{b)}
Quelle est l'image de~$\R$ par l'application~$f$? Cette image
contient-elle~$0$?


\exercise Exemples

\question
Donnez un exemple de fonction continue~$f\colon\left]0,1\right[\to\R$ bornée mais
n'atteignant pas ses bornes.

\question
Donnez un exemple de fonction non continue~$g\colon\R\to\R$ et
d'intervalle~$I\subset\R$ dont l'image par~$g$ n'est pas un
intervalle de~$\R$.

\question
Donnez un exemple de fonction continue~$h\colon\R\to\R$ et
d'intervalle~$J\subset\R$ dont l'image par~$h$ n'est pas
un~\em{segment}.


\exercise Rectangle d'aire maximale inscrit dans une ellipse

On considère deux nombre réels strictement positifs~$a > b > 0$, et
l'ellipse~$\cal E$ de grand rayon~$a$ et de petit rayon~$b$. À~tout
nombre~$\theta$ dans~$[0,\pi/2]$, on associe les points de~$\cal E$
suivants:
$A = (a\cos\theta, b\sin\theta)$,
$B = (-a\cos\theta, b\sin\theta)$,
$C = (-a\cos\theta, -b\sin\theta)$
et~$D = (a\cos\theta, -b\sin\theta)$.
Cette situation est représentée sur la figure ci-dessous:
\imagedisplay{figure.1}

\question
Exprimez la surface~$s(\theta)$ du rectangle~$ABCD$ en fonction des
constantes~$a$, $b$ et de la variable~$\theta$.

\question
Montrez que la fonction~$s$ a un maximum sur le segment~$[0,\pi/2]$.

\question
Dressez le tableau des variations de~$s$ et trouvez une
valeur~$\theta_m\in[0,\pi/2]$ en laquelle~$s$ atteint son maximum.

\bye

%%% End of file `ccont1.tex'
