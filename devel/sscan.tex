\class devel
\title Scanning macro arguments

\section Playing with getoptag

\def\makeattribute#1{$\left\{\mbox{#1}\right\}$}

\def\taga{\makeattribute{A}}

\def\tagb#1{\makeattribute{B = #1}}

\def\tagc#1{%
  \teststrempty{#1}%
  \ifstrempty
    \makeattribute{C}%
  \else
    \makeattribute{C = #1}%
  \fi
}

\def\getopttrial{%
  \def\getopttagunit{\\{\taga}}
  \def\getopttagmandatory{\\{\tagb}}
  \def\getopttagoptional{\\{\tagc}}
  \getopttag\getopttrial@M
}

\def\getopttrial@M#1{%
  \noindent
  \hbox to .6\hsize{\##1\#\hfill\#}%
  \ignorespaces
}

\begingroup
\obeylines
\getopttrial\taga 		Off with her head!
\getopttrial\taga\taga		Off with her head!
\getopttrial\tagb 1		Off with her head!
\getopttrial\tagb{2}\taga	Off with her head!
\getopttrial\tagc[3]\tagc	Off with her head!
\getopttrial\tagb{4}\tagc\taga	Off with her head!
\endgroup



\section Playing once again with getopttag

\def\alpha{ALPHA}
\def\beta{BETA}
\def\gamma{GAMMA}
\def\delta{DELTA}

\def\trial{%
  \begingroup
  \def\alpha{(alpha)}
  \def\beta{(beta)}
  \def\gamma##1{(gamma ##1)}
  \def\delta##1{%
    \ifoptbracket(delta "##1")\else(delta)\fi
  }%
  \def\getopttagunit{\\{\alpha}\\{\beta}}%
  \def\getopttagmandatory{\\{\gamma}}%
  \def\getopttagoptional{\\{\delta}}%
  \getopttag\trial@A
}
\def\trial@A#1{\tt\par(tags\teststrempty{#1}\ifstrempty\else\space #1\fi)\par\endgroup}

\trial\alpha\beta
\trial\alpha\gamma{A}
\trial\alpha\gamma1
\trial\alpha
\trial\delta[B]
\trial\delta\alpha
\beta\alpha\beta
\trial``Oh oh!'' cried baby Sally.




\section Read blanks then trigger a callback

\def\discardblanks@L#1{%
  \beginnext
    \def\rmA{#1}%
    \def\rmB{\stop}%
    \ifx\rmA\rmB
      \let\next=\relax
    \else
      \def\next{#1\discardblanks}%
    \fi
  \endnext
}

\def\discardblanks{%
  \readblanks\then\discardblanks@L\done
}

\medskip
\noindent
This is how we are discarding blanks:
\begindisplay
\discardblanks AB C D EFG HI JK L\stop
\enddisplay
Note that macros in the argument are not expanded, so if you put the
macro `space' between D and EFG, you will get
\begindisplay
\discardblanks AB C D\space EFG HI JK L\stop
\enddisplay


\section Read specifications

The specifications we want to read look like those appearing in {\TeX}
boxes specifications.

\def\getoptspecreset{%
  \let\getoptspecvoid\empty
  \let\getoptspecmandatory\empty
  \let\getoptspecbracket\empty
}

\def\getoptspec@P#1{%
  \beginnext
  \let\next\@false
  \ifcat A\noexpand#1\let\next\@true\fi
  \endnext
}

\def\getoptspec{%
  \beginnext
  \readtokens\getoptspec@P\to\rtA\then\getoptspec@L\done
}

\def\getoptspec@A#1#2{%
  \ifflag\else
    \beginnext
    \rtB={\toksloadalistvalue#1}%
    \rtC={\to\rtB}%
    \edef\next{\the\rtB{\the\rtA}\the\rtC}%
    \endnext
    \ifexception\else
      \flagtrue
      \rtD{#2}%
    \fi
  \fi
}

\def\getoptspec@L{%
  % We look for the identifier stored in rtA
  % in the association lists getoptspecvoid, getoptspecmandatory,
  % getoptspecbracket.
  \flagfalse
  \rtD{\getoptspec@E}%
  \getoptspec@A\getoptspecvoid\getoptspec@V
  \getoptspec@A\getoptspecmandatory\getoptspec@M
  \getoptspec@A\getoptspecbracket\getoptspec@B
  \ifflag
    \edef\next{%
      \noexpand\readblanks
      \noexpand\then
      \the\rtD
      \noexpand\done
    }%
  \else
    % We did not find a match
    %  and rtD = {\getoptspec@E}
    \def\next{\the\rtD}%
  \fi
  \next
}

\def\getoptspec@E{%
  % We reached the end of the specification and
  % put back the tokens stored in rtA on the input
  \edef\next{\the\rtA}%
  \endnext
}

\def\getoptspec@V{%
  % Specifications without values
  %  we give control to the callback
  %  and start the whole thing again
  \edef\next{\the\rtB}%
  \endnext
  \getoptspec
}

\def\getoptspec@M{%
  % Specifications with a value
  %  we expect the value linked with the key rtA
  %  to call again readspec on its own
  \edef\next{\the\rtB}%
  \endnext
}

\def\getoptspec@B{%
  % Specifications with an optional bracket value
  \getoptbracket\getoptspec@C
}

\def\getoptspec@C#1{%
  \rtA={#1}%
  \edef\next{\the\rtB{\the\rtA}\noexpand\getoptspec@S}%
  \endnext
}

\def\getoptspec@S{%
  \readblanks\then\getoptspec\done
}

\def\getoptspec@R{%
  \afterassignment\getoptspec@S
}

%
% Un exemple
%

\def\testgetoptspec#1#{\testgetoptspec@M{#1}}

\def\testgetoptspec@M#1#2{%
  \rdA=0pt
  \rdB=0pt
  \rdC=0pt
  \getoptspecreset
  \def\getoptspecmandatory{%
    \\{{width}{\getoptspec@R\rdA}}%
    \\{{height}{\getoptspec@R\rdB}}%
    \\{{depth}{\getoptspec@R\rdC}}%
  }%
  \def\getoptspecbracket{%
    \\{{star}{\makestar}}%
  }%
  \def\makestar##1{%
    \def\makestar@x{##1}%
  }%
  \let\makestar@x\undefined
  %
  % Here we go
  %
  We process the sequence
  \begindisplay
  \tt #1$\{$#2$\}$%
  \enddisplay
  through `readspec' now.
  \par
  \tt
  START\par
  \getoptspec#1{#2}\par
  STOP\par
  \normalfont
  \begindisplay\tt
  \indent
  \ifx\makestar@x\undefined\else
     \ifx\makestar@x\empty
       \llap{* }%
     \else
       \llap{\makestar@x\space}%
     \fi
  \fi
  rdA = \the\rdA\endgraf
  rdB = \the\rdB\endgraf
  rdC = \the\rdC\endgraf
  \enddisplay
}

\testgetoptspec star width 12pt height 16pt depth 13pt{text}
\testgetoptspec star[\P]width 12pt height 16pt depth 13pt{text}
\testgetoptspec star [\P] width 12pt height 16pt depth 13pt{text}


\meaning\ignoreblanks@l
\meaning\spacetoken
\meaning\tabtoken
\meaning\newlinetoken

\section Reading specifications (reprise)

\newtoks\getoptspecargument

%
% Managing getoptspec parameters
%

\def\getoptspecreset{%
  \let\getoptspecvoid\empty
  \let\getoptspecmandatory\empty
  \let\getoptspecbracket\empty
}

\def\getoptspecregister#1{%
  \beginnext
  \def\getoptspecregister@N{%
    \getoptspecregister@M{#1}%
  }%
  \afterassignment\getoptspecregister@N#1%
}

\def\getoptspecregister@M#1{%
  \edef\next{%
    \noexpand\getoptspecargument={\the#1}%
    \noexpand\getoptspec@M
  }%
  \endnext
}


\def\getoptspecdimen{%
  \getoptspecregister\rdA
}

\def\getoptspeccount{%
  \getoptspecregister\rcA
}

\def\getoptspecskip{%
  \getoptspecregister{\skip0}%
}

\def\getoptspectoks{%
  \getoptspecregister\rtA%
}

\let\selectgetoptspec@reset\getoptspecreset
\defselect{getoptspec}{reset}

\def\defgetoptspec#1{%
  \cslet{getoptspec@#1@unit}\empty
  \cslet{getoptspec@#1@mandatory}\empty
  \cslet{getoptspec@#1@optional}\empty
  \beginnext
  \toksloadcsname{selectgetoptspec@#1}\to\rtA
  \toksloadcsname{getoptspec@#1@unit}\to\rtB
  \toksloadcsname{getoptspec@#1@mandatory}\to\rtC
  \toksloadcsname{getoptspec@#1@optional}\to\rtD
  \let\\\noexpand
  \edef\next{%
    \\\def\the\rtA{%
      \\\let\\\getoptspecunit\the\rtB
      \\\let\\\getoptspecmandatory\the\rtC
      \\\let\\\getoptspecoptional\the\rtD
    }%
  }%
  \endnext
}

\def\getoptspecunitadd#1#2#3{%
  \beginnext
  \toksloadcsname{getoptspec@#1@unit}\to\rtA
  \rtB={#2}%
  \rtC={#3}%
  \let\\\noexpand
  \edef\next{%
    \\\listrappend{\the\rtB}{\the\rtC}\\\to\the\rtA
  }%
  \endnext
}

\def\getoptspecmandatoryadd#1#2#3#4{%
  \beginnext
  \toksloadcsname{getoptspec@#1@mandatory}\to\rtA
  \rtB={#2}%
  \rtC={{#3}{#4}}%
  \let\\\noexpand
  \edef\next{%
    \\\listrappend{\the\rtB}{\the\rtC}\\\to\the\rtA
  }%
  \endnext
}

\def\getoptspecoptionaladd#1#2#3{%
  \beginnext
  \toksloadcsname{getoptspec@#1@optional}\to\rtA
  \rtB={#2}%
  \rtC={#3}%
  \let\\\noexpand
  \edef\next{%
    \\\listrappend{\the\rtB}{\the\rtC}\\\to\the\rtA
  }%
  \endnext
}

\def\getoptspec#1{%
  \beginnext
  \rtA={}%
  \rtD={#1}%
  \getoptspecargument={}%
  \getoptspec@M
}

\def\getoptspec@M{%
  \testtoksempty\getoptspecargument
  \iftoksempty\else
  \beginnext
  \edef\next{%
    \noexpand\toksrappend{{\the\getoptspecargument}}%
    \noexpand\to\noexpand\rtA
  }%
  \endnext
  \getoptspecargument={}%
  \fi
  \readblanks\then\getoptspec@Z\done
}

\def\getoptspec@Z{%
  \readtokens\getoptspec@P\to\rtB\then\getoptspec@L\done
}


\def\getoptspec@P#1{%
  \beginnext
  \let\next\@false
  \ifcat A\noexpand#1\let\next\@true\fi
  \endnext
}

% getoptspec@A ALIST CALLBACK
%
%  ifflag
%   true iff we already found a match in a previous list
%
%  next
%   expands to an action to perform after that each list has been looked up
%
%  rtA
%   contains the tokens gathered so far
%
%  rtB
%   contains the key to look up in ALIST
%
%  rtD
%   contains the callback from the invocation of getoptspec
\def\getoptspec@A#1#2{%
  \ifflag\else
    \beginnext
    % We construct the call to `toksloadalistvalue'
    %  where the key rtB is expanded
    \rtA={\toksloadalistvalue#1}%
    \rtC={\to\rtC}%
    \edef\rmN{\the\rtA{\the\rtB}\the\rtC}%
    \rmN
    \ifexception
      \let\next\empty
    \else
      % rtC contains the value matching the key rtB
      \rtA={#2}%
      \edef\next{%
        \noexpand\def\noexpand\rmN{%
          \noexpand\flagtrue
          \the\rtA{\the\rtC}%
        }%
      }%
    \fi
    \endnext
  \fi
}

\def\getoptspec@L{%
  % We look for the identifier stored in rtB
  %  in the association lists getoptspecunit, getoptspecmandatory,
  %  and getoptspecoptional.
  \flagfalse
  \def\rmN{\getoptspec@E}%
  \getoptspec@A\getoptspecunit\getoptspec@U
  \getoptspec@A\getoptspecmandatory\getoptspec@V
  \getoptspec@A\getoptspecoptional\getoptspec@O
  \rmN
}

\def\getoptspec@E{%
  % We reached the end of the specification
  %  and put back the tokens stored in rtB on the input
  \edef\next{\the\rtD{\the\rtA}\the\rtB}%
  \endnext
}

\def\getoptspec@U#1{%
  \toksrappend#1\to\rtA
  \getoptspec@M
}

\def\getoptspec@V#1{\getoptspec@W#1}

\def\getoptspec@W#1#2{%
  \toksrappend#1\to\rtA
  #2%
}


%%% TEST

\def\getoptspecunit{%
  \\{{compact}{\test@compact}}%
}

\def\getoptspecmandatory{%
  \\{{width}{{\test@width}{\getoptspecdimen}}}%
  \\{{height}{{\test@height}{\getoptspecdimen}}}%
  \\{{depth}{{\test@depth}{\getoptspecdimen}}}%
  \\{{glue}{{\test@glue}{\getoptspecskip}}}%
  \\{{text}{{\test@text}{\getoptspectoks}}}%
  \\{{iteration}{{\test@iteration}{\getoptspeccount}}}%
}%

\def\getoptspecoptional{%
  \\{{star}{\test@star}}%
}%

\getoptspec\showtoks compact pomme
\getoptspec\showtoks
 width 12pt
 compact
 height 12pt
 glue 12 pt plus 1fil minus 13.1mm
 iteration 13
 text {Oh la la ...}%
\getoptspec\showtoks nomatch ``Oh Oh!''


\section A framed box

The macro `fbox' creates a `vbox' surrounded by a frame.  The
parameters describing this frame are passed as specifications.

The following specifications are recognized:
\beginroster
\item`margin<dimen>'
set the four margin parameters to `<dimen>', these parameters can also
be set individually.
\item`thickness<dimen>'
set the four thickness parameters to `<dimen>', these parameters can
also be set individually.
\item`to<dimen>'
set the desired height of the resulting `vbox'.
\endroster

% Quoting from the TeX book, page 290:
%
% \afterassignment token . The token is saved in a special place; it
% will be inserted back into the input just after the next assignment
% command has been per- formed. An assignment need not follow
% immediately; if another \afterassignment is performed before the
% next assignment, the second one overrides the first. If the next
% assignment is a \setbox, and if the assigned box is \hbox or \vbox
% or \vtop, the token will be inserted just after the { in the box
% construction, not after the }; it will also come just before any
% tokens inserted by \everyhbox or \everyvbox.


\def\fbox{%
  \vbox\bgroup
  \selectgetoptspec{fbox}%
  \fboxoverlapfalse
  \getoptspec\fbox@T
}

\def\fbox@T#1{%
  \rtA{#1}%
  \afterassignment\fbox@M\rtB
}

\def\fbox@M{%
  \beginnext
  \let\\\noexpand
  \edef\next{%
    \\\fbox@A{\the\rtA}{\the\rtB}%
  }%
  \endnext
}

\def\fbox@A#1#2{%
  \setbox\rbA\vbox{#2}%
  \relax#1\relax
  \boxloadframe\rbA\to\rbB
  \hsize=\wd\rbB
  \iffboxoverlap
  \hsize=\wd\rbA
  \kern-\framethicknesstop
  \kern-\framemargintop
  \hbox\bgroup
    \kern-\framethicknessleft
    \kern-\framemarginleft
  \fi
  \box\rbB
  \iffboxoverlap
    \kern-\framethicknessleft
    \kern-\framemarginleft
    \egroup
  \fi
  \egroup
}%


\defgetoptspec{fbox}

\newif\iffboxoverlap
\getoptspecunitadd{fbox}{overlap}{\fboxoverlaptrue}

\getoptspecmandatoryadd{fbox}{margin}{\fbox@margin}{\getoptspecdimen}
\getoptspecmandatoryadd{fbox}{margintop}{\fbox@margintop}{\getoptspecdimen}
\getoptspecmandatoryadd{fbox}{marginbottom}{\fbox@marginbottom}{\getoptspecdimen}
\getoptspecmandatoryadd{fbox}{marginright}{\fbox@marginright}{\getoptspecdimen}
\getoptspecmandatoryadd{fbox}{marginleft}{\fbox@marginleft}{\getoptspecdimen}

\def\fbox@margin#1{%
  \framemargintop=#1\relax
  \framemarginbottom=#1\relax
  \framemarginleft=#1\relax
  \framemarginright=#1\relax
}

\def\fbox@margintop#1{%
  \framemargintop=#1\relax
}

\def\fbox@marginbottom#1{%
  \framemarginbottom=#1\relax
}

\def\fbox@marginleft#1{%
  \framemarginleft=#1\relax
}

\def\fbox@marginright#1{%
  \framemarginright=#1\relax
}

\getoptspecmandatoryadd{fbox}{thickness}{\fbox@thickness}{\getoptspecdimen}
\getoptspecmandatoryadd{fbox}{thicknesstop}{\fbox@thicknesstop}{\getoptspecdimen}
\getoptspecmandatoryadd{fbox}{thicknessbottom}{\fbox@thicknessbottom}{\getoptspecdimen}
\getoptspecmandatoryadd{fbox}{thicknessright}{\fbox@thicknessright}{\getoptspecdimen}
\getoptspecmandatoryadd{fbox}{thicknessleft}{\fbox@thicknessleft}{\getoptspecdimen}

\def\fbox@thickness#1{%
  \framethicknesstop=#1\relax
  \framethicknessbottom=#1\relax
  \framethicknessleft=#1\relax
  \framethicknessright=#1\relax
}

\def\fbox@thicknesstop#1{%
  \framethicknesstop=#1\relax
}

\def\fbox@thicknessbottom#1{%
  \framethicknessbottom=#1\relax
}

\def\fbox@thicknessleft#1{%
  \framethicknessleft=#1\relax
}

\def\fbox@thicknessright#1{%
  \framethicknessright=#1\relax
}


\def\content{%
  \hbox{Hello!}%
}

\hbox{%
\rdA=1ex
\advance\rdA-1pt
\fbox margin \rdA overlap thickness 1pt{%
  \content
}%
\vbox{\content}%
}

\bye
