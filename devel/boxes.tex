% boxes.tex -- Playing with boxes

\input setup
\title

% First, we draw an horizontal line, to find the basepoint of the
% containing line.
%
% If we want to have the baseline of the first box on the page to be
% located at `layoutmarginleft,layoutmargintop), we have to unskip the
% `topskip' glue.
\null
\vskip-\topskip
\hrule width \hsize height 0pt depth 10pt
\vskip 10mm
\hrule
\medskip
\noindent
Off with her head!
\newpage

% Production of a box with its baseline, this is from the TeXbook,
% experiments, that only shall go on-screenlines 3624--.

% Macros for drawing figures (not in Appendix E)
\def\hidehrule#1#2{\kern-#1\hrule height#1 depth#2 \kern-#2 }
\def\hidevrule#1#2{\kern-#1{\dimen0=#1
    \advance\dimen0 by#2\vrule width\dimen0}\kern-#2 }
% \makeblankbox puts rules at the edges of a blank box
% whose dimensions are those of \box0 (assuming nonnegative wd,ht,dp)
% #1 is rule thickness outside, #2 is rule thickness inside
\def\makeblankbox#1#2{\hbox{\lower\dp0\vbox{\hidehrule{#1}{#2}%
    \kern-#1% overlap the rules at the corners
    \hbox to\wd0{\hidevrule{#1}{#2}%
      \raise\ht0\vbox to #1{}% set the vrule height
      \lower\dp0\vtop to #1{}% set the vrule depth
      \hfil\hidevrule{#2}{#1}}%
    \kern-#1\hidehrule{#2}{#1}}}}
\def\maketypebox{\makeblankbox{0pt}{1pt}}
\def\makelightbox{\makeblankbox{.2pt}{.2pt}}

% \box\bigdot is a null box with a bullet at its reference point
\newbox\bigdot \newbox\smalldot
\setbox0=\hbox{$\vcenter{}$} % \ht0 is the axis height
\setbox1=\hbox to\z@{$\hss\bullet\hss$} % bullet is centered on the axis
\setbox\bigdot=\vbox to\z@{\kern-\ht1 \kern\ht0 \box1 \vss}
\setbox1=\hbox to\z@{$\hss\cdot\hss$} % cdot is centered on the axis
\setbox\smalldot=\vbox to\z@{\kern-\ht1 \kern\ht0 \box1 \vss}

% \samplebox makes the outline of a box, with big dot at reference point
\def\samplebox#1#2#3#4{% #1=ht, #2=dp, #3=wd, #4=text
  {\setbox0=\vtop{\vbox to #1{\hbox to #3{}\vss}
      \nointerlineskip
      \vbox to #2{}}% now \box0 has the desired ht, dp, and wd
    \hbox{\copy\bigdot
      \vrule height.2pt depth.2pt width#3%
      \kern-#3%
      \makelightbox
      \kern-#3%
      \raise#1\vbox{\hbox to #3{\hss#4\hss}
        \kern 3pt}}}}

% \sampleglue makes glue between sample boxes
\newdimen\varunit
\varunit=\hsize \advance\varunit by-2\parindent
\divide\varunit by 58 % illustrations in Chapter 12
\def\sampleglue#1#2{% #1=width, #2=text
  \vtop{\hbox to #1{\xleaders\hbox to .5\varunit{\hss\copy\smalldot\hss}\hfil}
    \kern3pt
    \tabskip \z@ plus 1fil
    \halign to #1{\hfil##\cr#2\cr}}}

{\eightpoint
\setbox0=\hbox{$\uparrow$}
\setbox1=\hbox to \wd0{$\hss\mid\hss$} % with luck, they'll line up
\setbox2=\vbox{\copy0
  \nointerlineskip \kern-.5pt \copy1
  \nointerlineskip \kern-.5pt \copy1
  \moveleft 1em\hbox{height}
  \copy1 \nointerlineskip \kern-.5pt
  \copy1 \nointerlineskip \kern-.5pt
  \hbox{$\downarrow$}
  \kern.2pt}
\setbox3=\vbox{\kern.2pt\copy0
  \moveleft 1em\hbox{depth}
  \hbox{$\downarrow$}
  \kern0pt}
\setbox4=\vtop{\kern-3pt % this cancels the null text above the samplebox
  \hbox{\samplebox{\ht2}{\ht3}{6em}{}%
    \kern-6em
    \raise3pt\hbox to 6em{\hss Baseline\hss}}
  \kern3pt
  %\arrows{6em}{width}
}
\medskip\indent
\setbox0=\hbox{$\vcenter{}$}% \ht0 is the axis height
\lower\ht0\hbox{Reference point$-$\kern-.2em$\rightarrow$\kern2pt}%
\raise\ht2\box4
\kern1.5em
\raise\ht2\vtop{\kern0pt\box2\nointerlineskip\box3}}

\newpage


% We study the placement of points using there coordinates.



% Now we demonstrate the technique allowing us to place a box anywhere
% on the page.

\newdimen\gridhoffset
\newdimen\gridvoffset
\newdimen\gridhorigin
\newdimen\gridvorigin

\gridhorigin=0pt
\gridvorigin=0pt

\def\setgridorigin#1#2{%
  \gridhorigin=#1\relax
  \gridvorigin=#2\relax
}

\def\setgridoriginzero{%
  \setgridorigin{0pt}{0pt}%
}

\def\setgridoriginpaper{%
  \setgridorigin{-\layoutmarginleft}{-\layoutmargintop}%
}

\def\begingrid{%
  \setgridoriginzero
  \null
  \nointerlineskip
}

\def\begintopgrid{%
  \begingrid
  \setgridoriginpaper
  \vskip-\topskip
}

\def\endgrid{%
  \relax
}

\def\gridbox#1{%
  \vbox to 0pt{%
    \kern\gridvorigin
    \kern\gridvoffset
    \hbox{%
      \kern\gridhorigin
      \kern\gridhoffset
      \box#1
    }%
    \vss
  }%
  \nointerlineskip
}

\def\gridboxbaseline#1{%
  \begingroup
  \advance\gridvoffset-\ht#1%
  \gridbox#1%
  \endgroup
}


\begintopgrid
\setbox\rbA\copy\bigdot
\gridbox\rbA
\gridvoffset=10mm
\gridhoffset=20mm
\setbox\rbA\copy\bigdot
\gridbox\rbA
\setbox\rbA=\vbox{\nointerlineskip\samplebox{10pt}{20pt}{40pt}{}}
\gridboxbaseline\rbA
\setbox\rbB=\hbox{X}
\boxloadframe\rbB\to\rbA
\gridbox\rbA
%\boxloadtframe\rbB\to\rbA{X}{X}
%\gridbox\rbA
%\boxloadcframe\rbB 0pt\to\rbA
%\gridbox\rbA
\setbox\rbA=\hbox{X}
\gridboxbaseline\rbA
\endgrid

%% \vskip\gridvoffset
%% \vbox to 0pt{%
%%   \noindent
%%   \hbox to 0pt{%
%%     \hskip\gridhoffset
%%     % Here goes your own content. The baseline of this box will be
%%     % the baseline of all your construction.
%%     \vtop to\gridvsize{%
%%       \hsize=\gridhsize
%%       \parindent=0pt
%%       \strut X\par
%%       \vrule width \gridhsize depth 5pt
%%       \vfil
%%     }%
%%     % Now we are back to the general code.
%%     % First we negate the width of the previous hbox together with its
%%     % leading hskip, this makes the containing box look like it has
%%     % width 0.
%%     \hss
%%   }%
%%   % We also need to negate the height of our box.
%%   \vss
%% }


\newpage



%\def\begintopgrid#1#2{%
%  \null

\newpage



\bye
