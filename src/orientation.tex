%%% orientation.tex -- Orientation dans le jeu de modules

% Cadet TeX (https://github.com/foretspaisibles/cadet)
% This file is part of Cadet TeX.
%
% Copyright © 2001–2021 Michaël Le Barbier
% All rights reserved.

% This software is governed by the CeCILL-B license under French law and
% abiding by the rules of distribution of free software.  You can  use,
% modify and/ or redistribute the software under the terms of the CeCILL-B
% license as circulated by CEA, CNRS and INRIA at the following URL
% "https://cecill.info/licences/Licence_CeCILL-B_V1-en.txt"



\section{Orientation}

Nous examinons ici les différentes sortes de modules présents dans
{\brTeX} et les façons de les agencer pour obtenir un format. La
conception de {\brTeX} encourage la création de formats personnalisés
pour les individus, les associations, les sociétés, et de façon
générale toute sorte de projet faisant intervenir~{\TeX}. De plus il n'y
a pas dans~{\brTeX} de format «principal».

Énumérons les ingrédients de la recette permettant d'obtenir un format à
partir des modules fournis par {\brTeX}.  Ces modules sont
\emph{imposés} ou~\emph{optionnels}.
Les modules imposés sont des éléments fixes, communs à
tous les formats produits avec~{\brTeX}: chaque rôle évoqué est tenu par
un et un seul des modules de~{\brTeX}.   En revanche les modules
optionnels remplissent des rôles que d'autres acteurs pourraient
tenir.

Les modules imposés sont les suivants:

\begin{description}

\item[Noyau]
Le noyau offre des services de haut niveau et définit des types de
données pour réaliser ces services. Le noyau de {\brTeX} définit des
facilités pour la programmation, il rend des services pour
l'utilisation de plusieurs types de bibliothèques de procédures et
pour l'utilisation de paramètres régionaux.

\item[Définitions de base]
Elles sont un regroupement de définitions utiles lors de la
préparation d'un document mais ne présumant d'aucun type de document
particulier.  On peut y penser comme à un sous-enesmble des
définitions figurant dans~\emph{plain}.

\item[Plain font selection scheme]
Ce module définit un accès souple aux fontes de caractères,
ressemblant à ce qu'on trouve dans~{\LaTeX}.

\item[Conventions de saisie]
Les conventions de saisie transforment les séquences de la
forme \cs{compguillemteleft}, \cs{puncguillemetleft} et~\cs{euro} en
les commandes adéquates pour produire le symbole souhaité, avec
éventuellement un formatage supplémentaire.

\item[Codage de l'entrée]
Le module~\emph{latin9} prend les dispositions nécessaires
pour que les caractères hors du
champ ASCII7 soient traduits dans le langage de la convention
de saisie.

\end{description}


Les modules optionnels et leurs rôles sont les suivants:

\begin{description}
\item[Fontes pour le texte]
Les modules~\emph{fontlm}, \emph{fontaps}, \emph{fontpx}, \emph{fontpx}
et d'autres, instaurent l'utilisation des fontes Latin Modern, Adobe
Palatino et d'autres, Adobe Palatino et Times
\item[Fontes pour le mode mathématique]
Les modules~\emph{fontcmm}, \emph{fontpxm} et~\emph{fonttxm}
contiennent des définitions de fontes pour le mode mathématique.
\end{description}

Un format définit aussi une liste de bibliothèque optionnels qu'il
charge et les paramètres régionaux en vigueur après l'initialisation
de {\TeX}.

%%% End of file `orientation.tex'
