% Copyright (C) 2006-2014 Michael Grünewald
% All rights reserved.
%
% This file is part of Cadet TeX.
%
% Cadet TeX is free software: you can redistribute it and/or modify
% it under the terms of the GNU General Public License as published by
% the Free Software Foundation, either version 3 of the License, or
% (at your option) any later version.
%
% Cadet TeX is distributed in the hope that it will be useful,
% but WITHOUT ANY WARRANTY; without even the implied warranty of
% MERCHANTABILITY or FITNESS FOR A PARTICULAR PURPOSE.  See the
% GNU General Public License for more details.
%
% You should have received a copy of the GNU General Public License
% along with Cadet TeX.  If not, see <http://www.gnu.org/licenses/>.

\documentclass{article}
\usepackage{cadetman}
\title{Description de Cadet~\TeX}
\author{Michael Grünewald}
\begin{document}
\maketitle
Le logiciel~{\brTeX} est un outil de préparation de documents. Combiné
à d'autres programmes, il peut enregistrer les résultats de son
traitement au format~\emph{PostScript}, \emph{PDF}, \emph{DVI} ou
d'autres. La partie centrale de ce logiciel est une famille de modules
pour le programme~{\TeX}, cette partie est décrite par ce document. Les
modules dont il est question sont des éléments logiciels dont on peut
combiner un petit nombre pour obtenir \emph{un format}, comme il
l'est expliqué dans la section~\emph{Orientation}.

La description des modules présentée ici est vraiment très détaillée,
elle fait apparaître tous les détails de programmation. Une telle
précision est utile à celui qui veut écrire des macros, une
bibliothèque ou un modèle de document pour~{\brTeX}, mais celui qui
souhaite utiliser~{\brTeX} pour préparer un document particulier
a intérêt se tourner vers d'autres manuels.\footnote{Qui demandent
encore à être écrits!}

Une description générale de {\brTeX} et une présentation de
l'agencement de ce fichier peuvent être lues dans la
section~\emph{Présentation de {\brTeX}}.

Ce document a été produit à l'aide du logiciel~\emph{noweb}, un outil
de programmation lettrée, et de~{\LaTeX}.

\tableofcontents
\part{Présentation de \brTeX}
%%% orientation.tex -- Orientation dans le jeu de modules

% Author: Michael Grünewald <michael.lebarbier@laposte.net>
% Date: Ven 20 jul 2007 23:57:47 CEST

% Copyright (C) 2006, 2013 Michael Grünewald
% All rights reserved.
%
% This file is part of Bhrìd TeX.
%
% Bhrìd TeX is free software: you can redistribute it and/or modify
% it under the terms of the GNU General Public License as published by
% the Free Software Foundation, either version 3 of the License, or
% (at your option) any later version.
%
% Bhrìd TeX is distributed in the hope that it will be useful,
% but WITHOUT ANY WARRANTY; without even the implied warranty of
% MERCHANTABILITY or FITNESS FOR A PARTICULAR PURPOSE.  See the
% GNU General Public License for more details.
%
% You should have received a copy of the GNU General Public License
% along with Bhrìd TeX.  If not, see <http://www.gnu.org/licenses/>.


\section{Orientation}

Nous examinons ici les différentes sortes de modules présents dans
\brTeX\ et les façons de les agencer pour obtenir un format. La
conception de \brTeX\ encourage la création de formats personnalisés
pour les individus, les associations, les sociétés, et de façon
générale toute sorte de projet faisant intervenir~\TeX. De plus il n'y
a pas dans~\brTeX\ de format «principal».

Énumérons les ingrédients de la recette permettant d'obtenir un format à
partir des modules fournis par \brTeX.

\begin{description}

\item[Noyau]
Le noyau offre des services de haut niveau et définit des types de
données pour réaliser ces services. Le noyau de \brTeX\ définit des
facilités pour la programmation, il rend des services pour
l'utilisation de plusieurs types de bibliothèques de procédures et
pour l'utilisation de paramètres régionaux.

\item[Définitions de base]
Elles sont un regroupement de définitions disparates, un
sous-enesmble des définitions figurant dans~\emph{plian}.

\item[Plain font selection scheme]
Ce module définit un accès souple aux fontes de caractères,
ressemblant à ce qu'on trouve dans~\LaTeX.

\item[Conventions de saisie]
Les conventions de saisie transforment les séquencesde la
forme \cs{compguillemteleft}, \cs{puncguillemetleft} et~\cs{euro} en
les commandes adéquates pour produire le symbole souhaité, avec
éventuellement un formatage supplémentaire.

\item[Codage de l'entrée]
Le module~\emph{latin9} s'arrange pour que les caractères hors du
champ ASCII7 soient traduits dans le langage de la convention
de saisie.

\end{description}

Les éléments énumérés ci-dessus sont des éléments fixes, communs à
tous les formats produits avec~\brTeX: chaque rôle évoqué est tenu par
un et un seul des modules de~\brTeX. Les rôles suivants peuvent être
endossés par plusieurs modules:

\begin{description}
\item[Fontes pour le texte]
Les modules~\emph{fontlm}, \emph{fontaps}, \emph{fontpx}, \emph{fontpx}
et d'autres, instaurent l'utilisation des fontes Latin Modern, Adobe
Palatino et d'autres, Adobe Palatino et Times
\item[]
\end{description}



%%% End of file `orientation.tex'

\part{Noyau}
\input{kernel.latex}
\input{basedef.latex}
\part{Bibliothèques de base}
\input{mdoclist.latex}
%%% paper.cls -- Papiers divers

% Author: Michael Grünewald
% Date: Wed Aug 16 00:30:25 CEST 2006
% Copyright: (C) 2006, 2013 Michael Grünewald

% Copyright (C) 2006, 2013 Michael Grünewald
% All rights reserved.
%
% This file is part of Bhrìd TeX.
%
% Bhrìd TeX is free software: you can redistribute it and/or modify
% it under the terms of the GNU General Public License as published by
% the Free Software Foundation, either version 3 of the License, or
% (at your option) any later version.
%
% Bhrìd TeX is distributed in the hope that it will be useful,
% but WITHOUT ANY WARRANTY; without even the implied warranty of
% MERCHANTABILITY or FITNESS FOR A PARTICULAR PURPOSE.  See the
% GNU General Public License for more details.
%
% You should have received a copy of the GNU General Public License
% along with Bhrìd TeX.  If not, see <http://www.gnu.org/licenses/>.


%%% DESCRIPTION

% Un `papier' présente un texte structuré en section, paragraphes
% formels et petits paragraphes. Les sections et les paragraphes
% formels sont numérotés hiérarchiquement. Un `papier' peut présenter
% un texte mathématique et les équation sont alors numérotées dans
% chaque paragraphe formel. Un papier peut également être un manuel
% d'informatique et des commandes spéciales permettent de présenter
% des fragments de fichiers ou de boucles d'intéraction.


\macro zapfd
\macro verbatim

\enableauxfile

%%% A. Fontes

\def\romanfont{\rmfamily\fontseries{m}\fontshape{n}}
\def\propositionfont{\mainfont\fontshape{o}}
\def\titlefont{\romanfont\fontsize{17pt}}
\def\sectionfont{\romanfont\fontsize{10pt}\fontseries{b}}
\def\formalparfont{\romanfont\fontsize{10pt}\fontshape{N}}
\def\headerfont{\romanfont\fontsize{9pt}}
\def\footerfont{\romanfont\fontsize{9pt}}

%%% B. Enregistrement du document

\def\ifismark#1{\iffalse}

\let\theauthor=\empty
\let\theshortauthor=\empty
\let\thetitle=\empty
\let\theshorttitle=\empty

\sdef\author#1#2#3{%
  \ifstrempty{#2}\def\theshortauthor{#3}\else\def\theshortauthor{#2}\fi
  \def\theauthor{#3}%
}

\sdef\title#1#2#3{%
  \ifstrempty{#2}\def\theshorttitle{#3}\else\def\theshorttitle{#2}\fi
  \def\thetitle{#3}%
}


%%% C. PRÉPARATION DES PAGES

\def\emptypage{\eject\hbox{}\vfill\eject}
\def\nextevenpage{\vfill\supereject\ifodd\the\pageno\emptypage\fi}
\def\nextoddpage{\vfill\supereject\ifodd\the\pageno\else\emptypage\fi}

\def\maketitle{%
  \centerline{\titlefont\fontselect\thetitle}
  \bigskip
  \centerline{\romanfont\fontselect\theauthor}
  \vskip 4em plus 1em minus 1em
  \relax
}

\def\makeheader{}

\def\insertitleoutput{%
  %
  % On insére le titre dans la main vertical list (MVL)
  %  puis on reporte le contenu présenté à la routine de sortie
  %  (box255) dans cette MVL.
  %
  \vbox{\makeheader\maketitle}%
  \unvbox\@cclv\penalty\outputpenalty
  \global\output={\firstpageoutput}%
}

\def\firstpageoutput{%
  \headline={\hfil}%
  \plainoutput
  \global\output={\plainoutput}%
}

\headline={\headerfont\fontselect\sc\hfil\makeheadlinecontent\hfil}
\def\makeheadlinecontent{%
  \begingroup
  \disablemarkup
  \ifodd\pageno
    \rtA=\expandafter{\theshortauthor}%
  \else
    \rtA=\expandafter{\theshorttitle}%
  \fi
  \expandafter\lowercase\expandafter{\the\rtA}%
  \endgroup
}

\footline={%
  \footerfont\fontselect
  \hfil
  \rm\folio
  \hfil
}

\output={\insertitleoutput}


%%% D. STRUCTURE DU PAPELARD

\newtoks\structuretags
\newtoks\structureref
\newtoks\structuretitle

%
% Section
%

\newcount\sectionno
\newhook\sectionhook

\def\sectionbreak{%
  \vskip 0pt plus .3\vsize
  \penalty-250
  \vskip 0pt plus-.3\vsize
  \vskip 2em plus .3em minus .3em
  \relax
}

\sdef\section#1#2#3{%
  \structuretags={#1}%
  \structureref={#2}%
  \structuretitle={#3}%
  \advance\sectionno by 1
  \formalparno=0
  \setxref\makesectionxref{\the\sectionno}%
  \ifstrempty{#2}\else\label[#2]\fi
  \begingroup\makesection{#1}{#2}{#3}\endgroup
}

\def\makesection#1#2#3{%
  \sectionbreak
  \runhook\sectionhook
  \leftline{\sectionfont\fontselect
    \S\espacefine\the\sectionno. \ignorespaces#3}
  \bigskip
  \noindent
  \aftergroup\ignorewhitespace
}

\def\makesectionxref#1{\S\espacefine#1\relax}

%
% Formalpar
%

\newcount\formalparno
\newhook\formalparhook

\addhook
  \global\formalparno=0
\to\sectionhook

\sdef\formalpar#1#2#3{%
  \ifnum\sectionno=0
    \warning{Formal paragraphs must occur in a section}%
  \fi
  \structuretags={#1}%
  \structureref={#2}%
  \ifstrempty{#3}\structuretitle={sans titre}\else\structuretitle={#3}\fi
  \advance\formalparno by 1
  \setxref\makeformalparxref{{\the\sectionno}{\the\formalparno}}%
  \ifstrempty{#2}\else\label[#2]\fi
  \begingroup\makeformalpar{#1}{#2}{#3}\endgroup
}

\def\makeformalpar#1#2#3{%
  \bigbreak
  \runhook\formalparhook
  \noindent
  \hskip\marginleft
  \begingroup
    \formalparfont\mainsize
    \llap{\the\sectionno.\the\formalparno\relax.\enspace}%
    \ifstrempty{#3}\else\ignorespaces #3\space\medskip\nobreak\noindent\fi
  \endgroup
  \aftergroup\ignorewhitespace
}

\def\makeformalparxref#1{\makeformalparxref@A#1}
\def\makeformalparxref@A#1#2{{\formalparfont\fontselect(#1.#2)}}

\sdef\remark#1#2#3{%
  \ifstrempty{#3}%
    \sdef@@formalpar{#1}{#2}{Remarque}%
  \else
    \sdef@@formalpar{#1}{#2}{#3}%
  \fi
}

\sdef\example#1#2#3{%
  \ifstrempty{#3}%
    \sdef@@formalpar{#1}{#2}{Exemple}%
  \else
    \sdef@@formalpar{#1}{#2}{#3}%
  \fi
}


%%% E. ÉLÉMENTS MATHÉMATIQUES


%
% Proposition
%

\newhook\propositionhook

\sdef\proposition#1#2#3#4\par{%
  \sdef@@formalpar{#1}{#2}{#3}%
  \begingroup\propositionfont\fontselect#4\par\endgroup
  \medskip
}

%
% Preuves
%

\def\proof#1\par{\beginproof#1\endproof}

\long\def\beginproof#1\endproof{%
  {\it Preuve}. #1\hfill\endproofsign\par
  \smallskip
}

\zdchardef\endproofsign="76

%
% Équations
%

\newcount\equationno

\everydisplay={\prepareequationlabel}

\def\prepareequationlabel{%
  \let\equation@L=\label
  \let\label=\eqlabel
}

\def\eqlabel[#1]{%
  \global\advance\equationno by 1
  \setxref\makeequationxref{%
    {\the\sectionno}%
    {\the\formalparno}%
    {\the\equationno}%
  }%
  \equation@L[#1]%
  \relax\leqno\hbox{(\the\sectionno.\the\formalparno.\the\equationno)}%
}

\addhook
  \global\equationno=0
\to\formalparhook

\def\makeequationxref#1{\makeequationxref@A#1}
\def\makeequationxref@A#1#2#3{(#1.#2.#3)}


%%% F. ÉLÉMENTS INFORMATIQUES

\def\li#1{{\tt #1}}		% Literal

\leftappenditem\li\to\markup@L

%
% Code Source
%

\def\begincode{%
  \begingroup
  \def\verbatim@begin{\displayopenskip}%
  \def\verbatim@leave{%
    \displaycloseskip
    \endgroup\endgroup
    \noindent\ignorewhitespace
  }%
  \edef\verbatim@stop{\verbatim@esc endcode}%
  \verbatim@enter
}

%
% Écrans d'intéractions
%

% Les caractèrs spéciaux de TeX sont moins courants dans les écrans
% d'intéraction, comme par exemple avec le shell, aussi le caractère \
% permet d'utiliser de CS, et les accolades déterminent des groupes.

\def\beginscreen{%
  \begingroup
  \def\${\slshape\char"24\space}%
  \def\#{\slshape\char"23\space}%
  \def\%{\char"25 }%
  \def\\{\char"5C }%
  \def\{{\char"7B }%
  \def\}{\char"7D }%
  \def\user##1{\slshape ##1}%
  \def\verbatim@begin{%
    \medskip\nobreak
    \catcode`\%=0		% On ne peut pas utiliser \\ sinon
    \catcode`\{=1		%  \endscreen ne fonctionne pas
    \catcode`\}=2
  }%
  \def\verbatim@end{\medskip\endgroup\noindent}%
  \edef\verbatim@stop{\verbatim@esc endscreen}%
  \verbatim@enter
}

%%% G. PDF GOODIES

\def\enablepdfgoodies{%
\addhook
  \begingroup
  \edef\rmA{\noexpand\outline0{\the\structuretitle}}%
  \expandafter\endgroup\rmA
\to\sectionhook
\addhook
  \begingroup
  \iftoksempty\structuretitle\def\rmA{}\else
    \edef\rmA{\noexpand\outline1{\the\structuretitle}}%
  \fi
  \expandafter\endgroup\rmA
\to\formalparhook
}

\ifpdfoutput\enablepdfgoodies\fi
\let\enablepdfgoodies\undefined

%%% H. MACROS DE L'UTILISATEUR

\ifexists{paperrc}\input paperrc \fi

%%% End of file `paper.cls'

\input{output.latex}
\input{pfss.latex}
\input{pfssadj.latex}
\input{pfssmath.latex}
\part{Convention de saisie}
\input{convcomp.latex}
\input{convent.latex}
\input{convmath.latex}
\input{convpunc.latex}
\input{convtext.latex}
\part{Bibliothèques de fontes}
\input{fontcmm.latex}
\input{fontcmmt.latex}
\input{fontlm.latex}
\input{fontaps.latex}
\input{fonteufm.latex}
\input{fontpx.latex}
\input{fonttx.latex}
\part{Formats}
\input{cadetcore.latex}
\input{drvdvips.latex}
\input{drvpdf.latex}
% First, let's select a very small paper size.  It is suitable to our
% experiments, that only shall go on-screen.
%
% If you want to play with magnification, it has to be the first
% setting in the file, i.e it should go before the `input' statement
% getting this file processed.
\setpaperformat{A5}
\setmargin{10mm}{10mm}
\selectlayout
\nopagenumbers
\enableprivatenames

\def\maketitle#1{%
  \centerline{\largesize\bf#1}%
}

\def\title{%
  \beginnext
  \readline\to\rtA\then
    \edef\next{\noexpand\maketitle{\the\rtA}}%
    \endnext
  \done
}

\input{latin9.latex}
\end{document}
